Um die Experimente ausführen zu können, wurde eine Multi-Agenten-Simulationsumgebung entwickelt. Auf eine bereits existierende Simulationsumgebung wurde nicht zurückgegriffen, da dort die Kommunikationsebene bereits vorgegeben ist. Entweder existieren die Agenten als Objekte auf einem großen Server und rufen die Methoden bei anderen Agenten direkt auf (zum Beispiel \cite{web:marsGroup}), oder die Simulationsumgegbung gibt bestimmte Nachrichtenformate und Verhaltensmuster für Agenten vor (zum Beispiel \cite{web:mass}). Da es in dieser Arbeit nicht nur darum geht Experimente für einen verteilten Algorithmus zu wiederholen, sondern auch Software für ein Multi-Roboter-System zu entwickeln, sind diese Einschränkungen groß genug, um eine eigene Simulationsumgebung zu entwickeln.

Der Simulator wurde dabei rudimentär entwickelt. Der Simulator startet die JADE Umgebung und erzeugt alle Agenten. Außerdem stellt die Umgebung den Agenten die Karte für das jeweilige Experiment zur Verfügung und beobachtet, wann ein Experiment erfolgreich beendet wurde. Zusätzlich bietet der Simulator die Möglichkeit, die von den Agenten erzeugten Daten als Dateien zu speichern und visualisiert die Bewegungen der Agenten.

Da JADE jedoch über Threads skaliert \cite{book:jade}, tut dies der Simulator auch. Für ein verteiltes System ist dieses Verhalten kein Problem, für einen einzelnen Rechner bedeutet dies aber, dass eine hohe Anzahl Agenten nicht simuliert werden kann. Hier wäre ein potenter Server nötig oder die Simulationsumgebung müsste dahingehend erweitert werden, dass diese verteilt ausgeführt werden kann. Dies stellt auch die größte Schwachstelle des Simulators dar.