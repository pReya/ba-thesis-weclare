Um die Implementation des CoDy Algorithmus bewerten zu können, ist es notwendig qualitative Merkmale zu definieren, gegen die getestet werden kann. Da in \cite{book:regele} schon Experimente durchgeführt wurden, werden diese Merkmale zum größten Teil übernommen. Jedes Experiment wird 30 Mal wiederholt. Da für die Wegstrecke und den Zeitbedarf nur Durchschnittswerte angegeben sind, kommt für die eigenen Messungen ein \(95\%\)-Konfidenzintervall mit t-Verteilung zum Einsatz, um die Stichprobe besser einordnen zu können.

\begin{itemize}
\item \textbf{Erfolg:} Ein Durchlauf gilt dann als erfolgreich, wenn alle Agenten auf ihrer Zielposition parken.
\item \textbf{Prioritätsüberlauf:} Wenn es in einer Wiederholung zu mindestens einem Prioritätsüberlauf kommt, wird die Wiederholung entsprechend markiert.
\item \textbf{Wegstrecke:} Die Anzahl der Schritte, die ein Agent braucht um seine Zielposition zu erreichen. Auf einer Position zu verharren, erhöht diesen Wert also nicht.
\begin{itemize}
    \item \textbf{\(s\textsubscript{opt}\):} Die durchschnittliche Wegstrecke für die optimale Lösung.
    \item \textbf{\(s\textsubscript{CoDy}\):} Die durchschnittliche Wegstrecke für die Messung aus \cite{book:regele}.
    \item \textbf{[\(\bar{s}\textsubscript{u}\), \(\bar{s}\textsubscript{o}\)]:} Die untere und obere Grenze für das Konfidenzintervall der durchschnittlichen Wegstrecke.
\end{itemize}
\item \textbf{Zeitbedarf}: Gibt den Zeitpunkt wieder, zu dem der Agent final seine Zielposition erreicht hat.
\begin{itemize}
    \item \textbf{\(t\textsubscript{opt}\):} Der durchschnittliche Zeitbedarf für die optimale Lösung.
    \item \textbf{\(t\textsubscript{CoDy}\):} Der durchschnittliche Zeitbedarf für die Messung aus \cite{book:regele}.
    \item \textbf{[\(\bar{t}\textsubscript{u}\), \(\bar{t}\textsubscript{o}\)]:} Die untere und obere Grenze für das Konfidenzintervall des durchschnittlichen Zeitbedarfs.
\end{itemize}
\item \textbf{Prioritätsverlauf:} Zeigt die Prioritäten der einzelnen Agenten im Zeitverlauf der Experimente.
\end{itemize}

