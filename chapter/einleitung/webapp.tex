Neben den spezifischen, fachlichen Problemen, die \ac{sturesy} mit sich bringt und die im weiteren Verlauf dieser Arbeit noch genauer benannt werden, steckt auch eine idealistische Motivation in dieser Arbeit.

Seit wenigen Jahren etabliert sich eine moderne Variante der Software-Distribution: die Web-Applikation. Der Internet-Browser, der ursprünglich nur dazu gedacht war, statische Texte und Bilder darzustellen wurde immer leistungsfähiger und hat sich zu einer beliebten Anwendungsplattform entwickelt (vgl. \cite[S. 1]{art:web1to4}). Dank der Skriptsprache JavaScript sind Webbrowser heute mächtige, clientseitige Laufzeitumgebungen, wie die folgenden Beispiele zeigen.

Gerade Consumer-Software wie Officeanwendungen (Google Docs\footnote{\url{https://docs.google.com}}, Microsoft Office 365\footnote{\url{https://www.office.com}}), Grafikprogramme (Figma\footnote{\url{https://www.figma.com}}, Draw.io\footnote{\url{https://www.draw.io}}), Musik- und Videoplayer (Spotify\footnote{\url{https://www.spotify.com}}, Netflix\footnote{\url{https://www.netflix.com}}) oder Messenger (Telegram Web\footnote{\url{https://web.telegram.org}}, WhatsApp Web\footnote{\url{https://web.whatsapp.com}}) sind Anwendungen, die heute üblicherweise im Browser ausgeführt werden, ohne vorher (sichtbar) heruntergeladen und installiert werden zu müssen. Die Liste dieser Anwendungen wächst stetig und mit WebAssembly steht eine neue Technologie in den Startlöchern, die dafür sorgen könnte, dass zukünftig auch besonders leistungshungrige und komplexe Anwendungen wie etwa 3D- oder CAD-Software, Videobearbeitung und Spiele im Browser ausgeführt werden.

Webbrowser stellen also bereits heute eine Abstraktionsebene dar, die es ermöglicht, plattformunabhängige Software zu entwickeln. Damit nehmen sie immer öfter eine Funktion ein, die früher zum Beispiel von virtuellen Maschinen (etwa der Java Virtual Machine) oder Browser-Plugins (Adobe Flash, Java-Applets) erledigt wurde.

Im Gegensatz zu diesen älteren Technologien ist ein Browser aber auf den meisten Rechnern bereits installiert, oder wird gar mit dem Betriebssystem ausgeliefert (zum Beispiel Microsoft Edge bei Windows 10 oder Safari bei Apple Mac OS X). Die gesamte, sogenannte User Experience eines Browsers ist besser als die der bisherigen Lösungen.
Die Benutzung und das Funktionsprinzip von Adobe Flash oder der Java Virtual Machine sind für Laien schwer nachzuvollziehen. Der Gebrauch eines Browsers hingegen ist heute den meisten Menschen vertraut.

Ein weiterer Vorteil der Webbrowser ist, dass sie offene Web-Standards interpretieren, und es sich nicht um die proprietäre Technologie eines einzigen Herstellers handelt. Damit haben Browser und Web-Technologien gute Voraussetzungen, um zukünftig eine immer größere Rolle bei der Entwicklung plattformunabhängiger Software zu spielen.

Die konzeptionellen Vorteile von Web-Anwendungen für die Nutzer sind offensichtlich: das Auffinden der Software ist gleichbedeutend mit dem Merken einer Internetadresse. Das Herunterladen der Software wird vor dem Nutzer versteckt (sichtbar ist nur das Aufrufen einer Webseite, auch wenn im Hintergrund weiterhin Code heruntergeladen wird) und eine Installation entfällt. Software-Updates werden ebenfalls vor dem Nutzer verborgen (bei jedem Aufruf der Applikation wird automatisch die neuste Programmversion geladen).

Aber natürlich gibt es auch Nachteile: Web-Applikationen benötigen meistens eine dauerhafte und schnelle Internetverbindung, je nachdem wie viel der Anwendung beim Client im Browser ausgeführt wird, und wie viel der Server vorab berechnet. Außerdem ist die Leistungsfähigkeit von Web-Applikationen gegenüber nativen Anwendungen an einigen Stellen noch unterlegen, zum Beispiel durch den eingeschränkten Zugriff auf Betriebssystem-Schnittstellen (wie sich das ändern könnte, wird in Kapitel \ref{chap:ausblick} geschildert).

Ob und wie ein CRS sinnvoll als moderne Web-Applikation realisiert und weiterentwickelt werden kann, soll in den folgenden Kapiteln überprüft werden.

