Neben den spezifischen, fachlichen Problemen, die \ac{sturesy} mit sich bringt und die im weiteren Verlauf dieser Arbeit noch genauer benannt werden, steckt auch eine idealistische Motivation in dieser Arbeit.

Seit wenigen Jahren etabliert sich eine moderne Variante der Software-Distribution: die Web-Applikation. Der Internet-Browser, der ursprünglich nur dazu gedacht war, statische Texte darzustellen, hat sich zu einer beliebten Plattform Anwendungsplattform entwickelt. Dank der Programmiersprache JavaScript, sind Webbrowser heute mächtige clientseitige Laufzeitumgebungen und können in vielen Belangen genau so viel leisten wie klassische, native Anwendungen.

Gerade Consumer-Software wie Office (Google Docs, Microsoft Office 365), Grafikprogramme (Figma, Draw.io), Musik- und Videoplayer (Spotify, Netflix) oder Messenger (Telegram, WhatsApp) sind prominente Beispiel für Anwendungen die heute üblicherweise im Browser ausgeführt werden, ohne vorher heruntergeladen werden zu müssen. Die Liste dieser Anwendungen wächst stetig und mit WebAssembly steht eine neue Technologie in den Startlöchern, die dafür sorgen könnte, das zukünftig auch besonders leistungshungrige und komplexe Anwendungen wie etwa 3D- oder CAD-Software, Videobearbeitung und Spiele im Browser laufen.

Webbrowser stellen also bereits heute eine Abstraktionsebene dar, die es ermöglicht plattformunabhängige Software zu entwickeln. Damit nehmen sie immer öfter eine Funktion ein, die früher zum Beispiel von virtuellen Maschinen (etwa der Java Virtual Machine) oder Browser-Plugins (Adobe Flash, Java-Applets) erledigt wurde.

Im Gegensatz zu diesen, alten Technologien ist ein Browser aber auf den meisten Rechnern bereits installiert, oder wird gar mit dem Betriebssystem ausgeliefert. Die gesamte, sogenannte User Experience eines Browsers ist deutlich besser als die der bisherigen Lösungen. Einem Laien die Funktion der einer virtuellen Maschine oder des Flash-Plugins deutlich zu machen ist schwer – die Funktion eines Browsers ist heute fast jedem bekannt. Außerdem sind die bisherigen, alten Abstraktionsebenen für plattformübergreifende Software vielen Nutzern vorallem durch Sicherheitslücken und ständige Updates aufgefallen.

Ein weiterer Vorteil der Webbrowser ist, dass sie offene Web-Standards interpretieren und es sich nicht um die proprietäre Technologie eines einzigen Herstellers handelt. Damit haben Web-Technologien gute Voraussetzungen um zukünftig eine immer größere Bedeutung für plattformunabhängige Software zu spielen.

Die konzeptionellen Vorteile von Web-Anwendungen für die Nutzer sind offensichtlich: Das Auffinden der Software ist gleichbedeutend mit dem Merken einer Internetadresse. Das Herunterladen der Software wird vor dem Nutzer versteckt (sichtbar ist nur das Aufrufen einer Webseite, auch wenn im Hintergrund weiterhin Code heruntergeladen wird), eine Installation entfällt. Software-Updates werden ebenfalls vor dem Nutzer verborgen (bei jedem Aufruf der Applikation wird automatisch die neuste Programmversion geladen).

Aber natürlich gibt es auch Nachteile: Web-Applikationen benötigen meistens eine dauerhafte und schnelle Internetverbindung, je nachdem wie viel der Anwendung beim Client im Browser ausgeführt wird, und wie viel der Server vorab berechnet. Außerdem ist die Leistungsfähigkeit von Web-Applikationen gegenüber nativen Anwendungen an einigen Stellen noch eingeschränkt (wie sich das ändern könnte, wird in Kapitel \ref{chap:ausblick} geschildert).

Ob und wie ein CRS als Web-Applikation realisiert werden kann, soll in den folgenden Kapiteln überprüft werden.

