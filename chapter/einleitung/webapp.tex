Die Disziplin der Software-Entwicklung hat in den letzten Jahrzehnten enorme Sprünge gemacht. Die rasante Leistungssteigerung von Hardware, die zunehmende Mobilität von Computersystemen bei gleichzeitiger Preisreduzierung und die Zunahme und Qualität der Vernetzung haben Software-Entwicklung radikal beschleunigt und verändert. Ein fundamentales Prinzip hat sich allerdings nur wenig verändert: Die Distribution von Software.

Seit den Anfängen der PC-Ära in den 1980er Jahren bis in die heutige Zeit wird Software zu einem bestimmten Zeitpunkt auf ein Zielsystem übertragen und dann lokal auf dem System gestartet und ausgeführt. Nur einige Details an diesem Prozess haben sich im Laufe der Jahre geändert:

Während früher ein physischer Datenträger die Quelle der Software war, ist es heute meistens das Internet. Früher mussten die Quellen manuell im Internet gefunden werden, heute gibt es Software-Kataloge oder App-Stores, welche die Suche nach Software erleichtern. Diese Entwicklung 

Seit wenigen Jahren etabliert sich eine alternative Variante der Software-Distribution: die Web-Applikation. Das Internet, das ursprünglich nur dazu gedacht war, statische Texte darzustellen, hat sich zu einer beliebten Art der Software-Distribution verwandelt. Dank der Programmiersprache JavaScript, sind Webbrowser heute mächtige Laufzeitumgebungen und können in vielen Belangen genau so viel leisten wie native Anwendungen.

Webbrowser stellen heute eine Abstraktionsebene dar, die es ermöglicht plattformunabhängig zu entwickeln. Browser nehmen heute immer öfter die Funktion ein, die früher zum Beispiel von virtuellen Maschinen (etwa der Java Virtual Machine) erledigt wurde.

Die konzeptionellen Vorteile von Web-Anwendungen sind offensichtlich: Das Auffinden der Software ist gleichbedeutend mit dem Merken einer Internetadresse. Das Herunterladen der Software wird vor dem Nutzer versteckt (sichtbar ist nur das Aufrufen einer Webseite), eine Installation entfällt. Software-Updates werden ebenfalls vor dem Nutzer verborgen (bei jedem Aufruf der Applikation wird jeweils die neuste Programmversion geladen).

Natürlich gibt es auch Nachteile: Web-Applikationen benötigen meistens eine dauerhafte und schnelle Internetverbindung. Außerdem ist die Leistungsfähigkeit gegenüber nativen Anwendungen an einigen Stellen noch eingeschränkt (Ahead-of-time-Compile vs. INterpretiert).

