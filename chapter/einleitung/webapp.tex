Neben den spezifischen, fachlichen Problemen, die \ac{sturesy} mit sich bringt und die im weiteren Verlauf noch genauer benannt werden, steckt auch eine idealistische Motivation in dieser Arbeit.

Seit wenigen Jahren etabliert sich eine moderne Variante der Software-Distribution: die Web-Applikation. Der Internet-Browser, der ursprünglich nur dazu gedacht war, statische Texte darzustellen, hat sich zu einer beliebten Plattform der Software-Distribution entwickelt. Dank der Programmiersprache JavaScript, sind Webbrowser heute mächtige clientseitige Laufzeitumgebungen und können in vielen Belangen genau so viel leisten wie klassische, native Anwendungen.

Gerade Consumer-Software wie Office (Google Docs, Microsoft Office 365), Grafikprogramme (Figma, Draw.io), Musik- und Videoplayer (Spotify, Netflix) oder Messenger (Telegram, WhatsApp) sind prominente Beispiel für Anwendungen die heute üblicherweise im Browser ausgeführt werden, ohne vorher heruntergeladen werden zu müssen. Die Liste dieser Anwendungen wächst stetig und mit WebAssembly steht eine neue Technologie in den Startlöchern, die dafür sorgen könnte, das auch besonders leistungshungrige Anwendungen wie etwa 3D- oder CAD-Software, Videobearbeitung und Spiele bald im Browser laufen.

Webbrowser stellen also bereits heute eine Abstraktionsebene dar, die es ermöglicht plattformunabhängige Software zu entwickeln. Damit nehmen sie immer öfter eine Funktion ein, die früher zum Beispiel von virtuellen Maschinen (etwa der Java Virtual Machine) oder Browser-Plugins (Adobe Flash, Java-Applets) erledigt wurde.

Im Gegensatz zu den besagten Technologien ist ein Browser aber auf den meisten Rechnern bereits installiert, oder wird gar mit dem Betriebssystem ausgeliefert. Die sogenannte User Experience eines Browsers ist deutlich besser als die der bisherigen Lösungen. Einem Laien die Funktion der JVM/JRE oder des Flash-Plugins deutlich zu machen ist schwer. Außerdem sind beide Varianten in der Vergangenheit vorallem durch Sicherheitslücken und ständige Updates bei Nutzern in Erinnerung geblieben. Außerdem sind Browser Programme, die offene Web-Standards interpretieren. So ist das Ökosystem deutlich diverser als bei den genannten Projekten aus der Vergangenheit.

Die konzeptionellen Vorteile von Web-Anwendungen sind offensichtlich: Das Auffinden der Software ist gleichbedeutend mit dem Merken einer Internetadresse. Das Herunterladen der Software wird vor dem Nutzer versteckt (sichtbar ist nur das Aufrufen einer Webseite, auch wenn im Hintergrund weiterhin Code heruntergeladen wird), eine Installation entfällt. Software-Updates werden ebenfalls vor dem Nutzer verborgen (bei jedem Aufruf der Applikation wird automatisch die neuste Programmversion geladen).

Aber natürlich gibt es auch Nachteile: Web-Applikationen benötigen meistens eine dauerhafte und schnelle Internetverbindung, je nachdem wie viel der Anwendung beim Client im Browser ausgeführt wird, und wie viel der Server vorab berechnet. Außerdem ist die Leistungsfähigkeit von Web-Applikationen gegenüber nativen Anwendungen an einigen Stellen noch eingeschränkt (häufig bedingt durch den höheren Grad der Optimierung bei Sprachen, die vor der Ausführung optimiert und kompiliert werden).

