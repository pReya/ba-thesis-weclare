\textit{Internet of Things} (IoT) ist ein sich rasant entwickelndes Themengebiet. Über 50 Milliarden IoT-Geräte werden bis zum Jahr 2022 erwartet \cite{whitepaper:iot}. 
Viel Geld wird in z.B. 5G Netzen investiert => Hauptsächlicher Nutzen für IoT
Die Kommunikation mit IoT-Geräten funktioniert auf Basis eines Anfrage-Antwort-Schemas (TODO Quelle) und ist damit ereignisgesteuert. Des weiteren sind IoT-Geräte proaktiv, da sie Daten analysieren und diese beim menschlichen Endbenutzer melden (TODO Quelle).\newline
Proaktivität ist in Softwareparadigmen wie z.B. der \textit{objektorientierten Programmierung} (OOP) nicht als Konzept mit inbegriffen.\newline
\textit{Agentenorientierte Softwareentwicklung} (AoSE) hingegen ist ein Softwareparadigma, das, wie der Name vermuten lässt, Agenten als Kernkonzept hat. Agenten werden von vielen als der nächste evolutionäre Schritt von Objekten bezeichnet \cite{article:objectsVsAgents}\cite{article:flexibleSoftware} und grenzen sich zu Objekten durch ihre Proaktivität und Autonomie ab [\cite{article:wool,book:wool2002} in \cite{book:padgham}]. Vor dem Hintergrund, dass Autonomie immer (TODO Quelle) auf Ereignissen fußt, lässt sich vermuten, dass AoSE, obwohl es in der Industrie und im akademischen Umfeld zwar noch kaum zum Einsatz kommt \cite{article:agentsWhyAndHow}, potentiell eine langfristige Lösung für die neuen Ansprüche von IoT-Netzen darstellt. TODO \cite{article:flexibleSoftware} passt hier gut