Die \textit{Communication and Distributed System} (CaDS)-Arbeitsgruppe der \textit{Hochschule für angewandte Wissenschaften} (HAW) präsentiert ihre \textit{Smart Chairs}\footnote{Ein Bürostuhl, der mit verschiedenen (Druck-, Abstands- und Temperatur-) Sensoren \cite{web:smartChair}) sowie einem Antrieb ausgestattet ist.} auf Veranstaltungen, um Aufmerksamkeit für sich und die HAW zu generieren. In erster Linie werden die \textit{Smart Chairs} als IoT-Geräte betrachtet und dienen der Erforschung dieses Themengebiets.\newline
Die CaDS-Arbeitsgruppe möchte, dass die \textit{Smart Chairs} sich autonom bewegen, um eine Funktion für den kürzlich hinzugefügten Antrieb zu haben. Dabei sollen die \textit{Smart Chairs} bestimmte Positionen in einem Raum anfahren, um einen realen Anwendungsfall darstellen zu können. Dies könnte zum Beispiel das Fahren an Schreibtische sein. Je näher man einem realem beziehungsweise alltäglichem Szenario ist, desto interessanter und greifbarer wirkt die Anwendung.Diese grobe Definition wird im Rahmen dieser Arbeit nicht weiter vertieft, da der Fokus im Entwickeln eines \textit{Proof of Concepts} (PoC) liegt und hierfür ein detailliert ausformulierter Anwendungsfall nicht erforderlich ist. Lediglich die groben Rahmenbedingungen müssen bekannt sein, welche im folgenden Kapitel festgehalten werden.