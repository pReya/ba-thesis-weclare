In diesem Kapitel werden die Anforderungen an den Anwendungsfall beziehungsweise an das Softwaresystem aufgezählt.\newline
\begin{enumerate}
\item \textbf{Skalierbarkeit} ist eine wichtige Anforderung an das System, da der Anwendungsfall die Anzahl der Teilnehmer nicht klar vorgibt. Das System soll also skalieren, um mit einem einzigen sowie auch mit mehreren Teilnehmern umgehen können.
\item Das System soll \textbf{robust} sein. Die Positionen, die die \textit{Smart Chairs} anfahren, sollen, wenn sie erreicht werden können, in endlicher Zeit erreicht werden. Dabei müssen die Wege nicht optimal bezüglich ihrer Zeit oder Strecke gewählt werden.
\item Das System soll in \textbf{Echtzeit} agieren. Damit soll vermieden werden, dass lange Pausen oder Initialisierungsphasen eintreten.
\item Das System soll mit Hilfe von AoSE umgesetzt werden. Die \textit{Smart Chairs} sollen jeweils als \textbf{Agent} abgebildet werden.
\item Das System soll Hindernisse erkennen und ihnen ausweichen. Es soll also \textbf{nicht} zu \textbf{Kollisionen} kommen.
\end{enumerate}
Zwar sind alle Anforderungen wichtig, sie können, im Rahmen dieser Arbeit, aber nicht alle mit gleichem Gewicht bearbeitet werden. Der Fokus liegt auf der Skalierbarkeit und dem AoSE, da sie Teile der Forschungsfrage sind.