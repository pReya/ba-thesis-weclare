% Diese Arbeit gliedert sich in folgende weitere Kapitel.\newline\newline
% %
% Kapitel \hyperref[chap:theorie]{2} - Theoretischer Hintergrund - gibt einen Einblick in die Theorie verwendeter Methodiken, Konzepte und Algorithmen. Außerdem wird die Wahl der jeweiligen Methodiken, Konzepte und Algorithmen begründet und eine Auswahl an Alternativen dargestellt.\newline\newline
% %
% Kapitel \hyperref[chap:jadeDesign]{3}- Designen mit JADE -  beschreibt den allgemeinen Designprozess eines Multi-Agenten Systems unter der Anleitung von \textit{JADE}. Der Prozess wird an Beispielen explizit aufgezeigt.\newline\newline
% %
% Kapitel \hyperref[chap:methodik]{4} - Methodik - stellt die begründete Auswahl der Methode dar.\newline\newline
% %
% Kapitel \hyperref[chap:datenerhebung]{5} - Datenerhebung - beschreibt zum Einen die Simulationsumgebung und zum Anderen die Metriken, die in Kapitel 6 verwendet werden. Darüber hinaus wird die Strategie zur Validierung der Experimente erklärt.\newline\newline
% %
% Kapitel \hyperref[chap:experimente]{6} - Experimente - listet Experimente nach folgendem Schema auf: Fragestellung, Versuchsaufbau und Beobachtungen.\newline\newline
% %
% Kapitel \hyperref[chap:diskussion]{7} - Diskussion - diskutiert die Beobachtungen der Experimente im Hinblick auf die Fragestellung.\newline\newline
% %
% Kapitel \hyperref[chap:fazit]{8} - Fazit - fasst den Kern der Arbeit zusammen und gibt einen Ausblick.

TODO redo!