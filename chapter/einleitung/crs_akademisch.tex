Interaktivität ist ein Schlüsselfaktor für gute Lehrveranstaltungen. Der klassische "Frontalunterricht" gilt nicht mehr als sonderlich effektiv. Gleichzeitig ist es für Lehrende immer schwieriger, besagte Interaktivität herzustellen, je größer die Anzahl der Teilnehmer einer Veranstaltung ist. In kleinen Gruppen kann durch mündliche Kommunikation noch ein gewisser Grad an Interaktivität garantiert werden. Im Lehrbetrieb von großen Hochschulen und Universitäten ist das aber aufgrund der hohen Teilnehmerzahlen kaum noch realisierbar.

Eine Möglichkeit, gegen den Mangel an Interaktivität vorzugehen, ist der Einsatz sogenannter \acp{ars}. Im akademischen und schulischen Umfeld werden diese Systeme auch noch spezifischer als \acp{crs} bezeichnet. Dabei handelt es sich um Hardware- oder Software-Lösungen, die es einem Dozenten ermöglichen, während einer Veranstaltung beliebig viele Fragen nacheinander an das Publikum zu stellen. Diese können dann entweder mithilfe dedizierter Hardware-Geräte (sogenannte Clicker) oder mit einem Smartphone, Laptop oder Tablet-PC beantwortet werden. Üblicherweise werden dabei verschiedene Fragetypen unterstützt, zum Beispiel Simple- und Multiple-Choice-Varianten.

Ein entscheidendes Merkmal dieser Systeme ist die Anonymität der Antworten, denn nur so können die Teilnehmer ehrlich antworten und lernen, ihren Wissensstand tatsächlich besser einzuschätzen. CRS sind explizit keine Werkzeuge, um digitale Tests oder Prüfungen durchzuführen.

Die Nutzung solcher Systeme im akademischen Umfeld ist bereits relativ weit verbreitet. Weltweit werden diese Lösungen an vielen Hochschulen eingesetzt. Studien und Umfragen attestieren diesen Systemen eine Steigerung der Interaktivität, eine Verbesserung der Selbsteinschätzung und eine Steigerung der Aufmerksamkeit in Lehrveranstaltungen, zum Beispiel an der University of Wisconsin-Milwaukee\cite[S. 5]{web:wisconsin}:

\begin{quote}
„Students similarly reported that the use of clickers increased their engagement, involvement, and interaction, and help students pay attention in class.“
\end{quote}

Oder auch nach zehnjärigem Einsatz an der Harvard University\cite[S. 6]{web:tenyears}:

\begin{quote}
„We find that, upon first implementing Peer Instruction, our students’ scores on the Force Concept Inventory and the Mechanics Baseline Test improved dramatically, and their performance on traditional quantitative problems improved as well.“
\end{quote}


Eine Bewertung des didaktischen Konzepts von CRS ist daher nicht Bestandteil dieser Arbeit. Die Wirksamkeit und Relevanz wird als gegeben angenommen.

Auch an der HAW Hamburg werden CRS eingesetzt. Eine der verwendeten Software-Lösungen, die dort im Department Informatik zum Einsatz kommt, ist das \ac{sturesy} – eine Software, die im Rahmen einer Bachelorarbeit an der Universität Hamburg entwickelt wurde\cite{sturesy}. Wegen der positiven Erfahrungen wird StuReSy im Lauf dieser Arbeit als funktionales Vorbild für die Entwicklung eines neuen CRS dienen, welches das System konzeptionell verbessert, bestehende Probleme behebt und das Anwendungsgebiet stärker auf den Spezialfall der Programmierlehre zuspitzt.