Interaktivität gilt als ein Schlüsselfaktor für gute Lehrveranstaltungen. Der klassische "Frontalunterricht" gilt als nicht sonderlich effektiv. Gleichzeitig ist es für Lehrende immer schwerer, Interaktivität herzustellen, je höher die Anzahl der Teilnehmer einer Veranstaltung ist. In kleinen Gruppen kann durch mündliche Kommunikation noch ein gewisser Grad an Interaktivität hergestellt werden. Im Lehrbetrieb von Hochschulen und Universitäten ist das aber aufgrund der Teilnehmerzahlen kaum mehr möglich.

Eine Möglichkeit, gegen den Mangel an Interaktivität vorzugehen, sind sog. \acp{ars}. Im akademischen Umfeld werden diese Systeme auch noch spezifischer als \acp{crs} bezeichnet. Dabei handelt es sich um Hardware- oder Software-Lösungen, die es dem Dozenten ermöglichen, Fragen an die Veranstaltungs-Teilnehmer zu stellen. Diese können die Fragen dann entweder mithilfe dedizierter Hardware-Geräte (sogenannte Clicker) oder mit einem Smartphone, Laptop oder Tablet-PC beantworten. Ein entscheidendes Merkmal ist die Anonymität dieser Lösungen, denn nur dann können die Teilnehmer ehrlich antworten und ihren Wissensstand besser einschätzen.\ac{crs} sind keine Werkzeuge um Prüfungen auf digitalem Weg durchzuführen.


Die Nutzung solcher Systeme im akademischen Umfeld ist bereits weit verbreitet. Sowohl in Deutschland als auch international werden diese Lösungen an vielen Hochschulen eingesetzt. Viele Studien und Umfragen attestieren diesen Maßnahmen eine Steigerung der Interaktivität, eine Verbesserung der Selbsteinschätzung und eine Steigerung der Aufmerksamkeit in Lehrveranstaltungen.

Auch an der HAW Hamburg werden \ac{crs} eingesetzt. Eine der verwendeten Software-Lösungen die dort zum Einsatz kommt ist das \ac{sturesy}, eine Software die aus einer Abschlussarbeit an der Universität Hamburg von Wolf Posdorfer entwickelt wurde. \ac{sturesy} wird im Lauf der Arbeit als funktionales Vorbild für die Entwicklung eines neuen \ac{crs} dienen.