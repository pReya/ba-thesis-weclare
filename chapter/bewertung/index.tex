Gerade im Rahmen der Programmierlehre scheint der Einsatz von CRS sehr sinnvoll zu sein, denn der Umgang mit Programmiersprachen lässt sich gut objektiv bewerten und leicht in Fragenform bringen. Dennoch gibt es bei bestehenden CRS häufig große Hürden, sie im Kontext der Programmierlehre einzusetzen. Deswegen werden hier zwei bestehende CRS aus dem akademischen Bereich miteinander verglichen: Einerseits die bisher an der HAW Hamburg eingesetzte Lösung StuReSy und zum Vergleich eine populäre, kommerziellere Lösung namens Pingo.

Im Gespräch mit Herrn Professor Schmolitzky konnten schnell die Kriterien und Schwachstellen für einen Einsatz von CRS in der Programmierlehre in Erfahrung gebracht werden.

\section{StuReSy}
\label{chap:sturesy}
\ac{sturesy} ist der Name einer Software, die im Rahmen der Bachelorarbeit von Wolf Posdorfer im Jahr 20012 an der Universität Hamburg entstanden ist. Der Name StuReSy ist ein Akronym für „Student Response System“.

StuReSy besteht aus zwei Komponenten:
\begin{itemize}
    \item Server-Komponente: In PHP geschrieben, agiert gleichzeitig auch als Client-Komponente für die Abstimmungs-Teilenehmer. Inkludiert eine relationale SQL-Datenbank.
    \item Admin-Komponente: Um Fragen zu erstellen und zu bearbeiten wird ein Client als Java-Anwendung benötigt.
\end{itemize}

StuReSy wurde erfolgreich und viele Jahre an der Universität Hamburg und HAW Hamburg eingesetzt. Die Qualität und der Umfang der Software sind für eine Bachelorarbeit beeindruckend.

Dennoch verfügt StuReSy über einige Nachteile und Probleme:
\begin{itemize}
    \item Software-Download und JVM notwendig: Um StuReSy administrativ einsetzen zu können, muss eine Java-Software heruntergeladen werden und eine JVM muss auf dem jeweiligen System vorhanden sein. Eine Administration vom Tablet oder Smartphone ist damit nur schwer möglich.
    \item Server-Komponente: Um StuReSy betreiben zu können, wird eine Server-Instanz benötigt. Diese muss von der jeweiligen Institution oder einem Dozenten aufgesetzt und gewartet werden.
    \item Mangelnde Formatierungsmöglichkeiten für Software-Quelltext: In der Praxis wird StuReSy vor allem in Informatik-Veranstaltungen eingesetzt. Dort werden oft Fragen zu Quelltexten gestellt. Die Darstellung dieser Quelltexte ist schwierig: Zentrierte Text-Ausrichtung .... sorgen für unübersichtliche Darstellung.
\end{itemize}


\newpage
\section{Pingo}
\label{chap:pingo}
Pingo ist eine Software-Lösung, die bereits seit dem Jahr 2011 an der Universität Paderborn entwickelt wird. Der Name ist ebenfalls ein Akronym und steht für „\textbf{P}eer \textbf{In}struction for Very Large \textbf{G}r\textbf{o}ups“. Im Gegensatz zu StuReSy ist Pingo bereits weiter verbreitet und wird an vielen deutschen Hochschulen eingesetzt – im September 2018 gab es 22.000 angemeldete Nutzer (vgl. \cite{web:pingo_zukunft}). Dahinter stand außerdem ein ganzes Team akademischer Mitarbeiter (vgl. \cite{web:pingo_team}). Seit 2019 wird Pingo von der universitätsnahen Coactum GmbH betrieben und weiterentwickelt (vgl. \cite{web:pingo_coactum}). Das Projekt ist damit deutlich professioneller ausgerichtet als StuReSy.

Im Gegensatz zu StuReSy ist Pingo eine reine Web-Applikation, die öffentlich unter \url{http://trypingo.com/} auffindbar ist und kostenlos genutzt werden kann. Sowohl Administratoren als auch Teilnehmer können alle Arbeiten im Browser erledigen, ein Software-Download ist nicht notwendig. Für die administrative Nutzung muss jedoch ein Benutzerkonto erstellt werden. Einzelne Sitzungen werden durch numerische IDs im Namensraum eines Pingo-Servers identifiziert.


Eine Direktverbindung zwischen Umfrage-Teilnehmern wird nicht unterstützt, stattdessen wird eine Pingo-Server vorausgesetzt. Der Betreiber stellt einen öffentlichen Pingo-Server kostenlos zur Verfügung. Pingo steht aber unter einer Open-Source-Lizenz, so dass Nutzer auch eine eigene Instanz betreiben können. Pingo ist in der Programmiersprache Ruby und mithilfe des Web-Frameworks „Ruby on Rails“ implementiert (vgl. \cite{web:pingo_github}). Entsprechend dazu muss ein potenzieller Server auch über einen Ruby-Interpreter und über eine NoSQL-Datenbank verfügen, um Pingo ausführen zu können.

In ihren Kernfunktionen sind sich Pingo und StuReSy sehr ähnlich. Trotzdem fehlt eine kritische Funktionen für den Einsatz in der Programmierlehre:

Fragen innerhalb der Pingo-Plattform können überhaupt nicht formatiert werden. Damit können selbst simple Formatierungen wie Fettschreibungen, Unterstreichungen oder Zeilenumbrüche nicht verwendet werden (erkennbar in Abbildung \ref{abb:pingo_frage}). Dementsprechend ist auch die übersichtliche Darstellung von Quelltext unmöglich und Pingo für den Einsatz in der Programmierlehre ungeeignet.


\begin{figure}[H]
    \includegraphics[width=12cm]{chapter/bewertung/bilder/pingo_editor.png}
    \centering
    \caption[Fragen-Editor in Pingo ohne Formatierungsmöglichkeiten]{Der Fragen-Editor von Pingo verfügt über keine Formatierungsmöglichkeiten.}
    \label{abb:pingo_editor}
\end{figure}


\begin{figure}[H]
    \includegraphics[width=12cm]{chapter/bewertung/bilder/pingo_problem1.png}
    \centering
    \caption[Darstellung von Quelltexten in Pingo]{Eine ordentliche Darstellung von Quelltexten ohne Textformatierungen ist nicht möglich.}
    \label{abb:pingo_frage}
\end{figure}
