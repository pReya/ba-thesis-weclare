\ac{sturesy} ist der Name einer Software, die im Rahmen der Bachelorarbeit von Wolf Posdorfer im Jahr 20012 an der Universität Hamburg entstanden ist. Der Name StuReSy ist ein Akronym für „Student Response System“.

StuReSy besteht aus zwei Komponenten:
\begin{itemize}
    \item Server-Komponente: In PHP geschrieben, agiert gleichzeitig auch als Client-Komponente für die Abstimmungs-Teilenehmer. Inkludiert eine relationale SQL-Datenbank.
    \item Admin-Komponente: Um Fragen zu erstellen und zu bearbeiten wird ein Client als Java-Anwendung benötigt.
\end{itemize}

StuReSy wurde erfolgreich und viele Jahre an der Universität Hamburg und HAW Hamburg eingesetzt. Die Qualität und der Umfang der Software sind für eine Bachelorarbeit beeindruckend.

Dennoch verfügt StuReSy über einige Nachteile und Probleme:
\begin{itemize}
    \item Software-Download und JVM notwendig: Um StuReSy administrativ einsetzen zu können, muss eine Java-Software heruntergeladen werden und eine JVM muss auf dem jeweiligen System vorhanden sein. Eine Administration vom Tablet oder Smartphone ist damit nur schwer möglich.
    \item Server-Komponente: Um StuReSy betreiben zu können, wird eine Server-Instanz benötigt. Diese muss von der jeweiligen Institution oder einem Dozenten aufgesetzt und gewartet werden.
    \item Mangelnde Formatierungsmöglichkeiten für Software-Quelltext: In der Praxis wird StuReSy vor allem in Informatik-Veranstaltungen eingesetzt. Dort werden oft Fragen zu Quelltexten gestellt. Die Darstellung dieser Quelltexte ist schwierig: Zentrierte Text-Ausrichtung .... sorgen für unübersichtliche Darstellung.
\end{itemize}
