Pingo ist eine Software-Lösung, die bereits seit dem Jahr 2011 an der Universität Paderborn entwickelt wird. Der Name ist ebenfalls ein Akronym und steht für „\textbf{P}eer \textbf{In}struction for Very Large \textbf{G}r\textbf{o}ups“. Im Gegensatz zu StuReSy ist Pingo bereits deutlich weiter verbreitet und wird an vielen deutschenn Hochschulen eingesetzt. Dahinter steht außerdem ein ganzes Team von Mitarbeitern. Seit 2019 wird Pingo von der universitätsnahen Coactum GmbH betrieben und weiterentwickelt.\newline

Prinzipiell handelt es sich bei Pingo um eine reine Web-Applikation, die öffentlich und kostenlos zugänglich ist. Für die Nutzung muss jedoch ein Benutzerkonto eröffnet werden. Pingo steht unter einer Open-Source-Lizenz und somit können Nutzer auch eine eigene Pingo-Instanz betreiben.

Pingo ist in der Programmiersprache Ruby und mithilfe des serverseitigen Web-Frameworks Ruby on Rails implementiert.