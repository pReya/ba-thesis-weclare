Die \textit{Foundation for Intelligent Physical Agents} (FIPA) der \textit{IEEE Computer Society} hat im Bereich des Multi-Agenten-Systems mehrere Standards erarbeitet. Diese Standards beschreiben alle grundlegenden Elemente und Funktionen, die für eine Multi-Agenten-Plattform benötigt werden. \cite{article:flexibleSoftware}

Es gibt einige Projekte, die diesen Standard implementieren \cite{web:fipaList}. Unter diesen eignet sich das \textit{Java Agent Development Framework} (JADE) am besten für dieses Projekt. Zum einen ist das JADE Framwork in \textit{Java} implementiert und somit plattformunabhängig \cite{web:java}. Es ist open-source. Es kann also eigenständig verändert und erweitert werden. Zudem ist die Benutzung des Frameworks nicht mit Lizenzkosten verbunden. Zuletzt ist JADE in Form von \cite{book:jade} detailliert beschrieben.

In den folgenden Kapiteln wird eine Auswahl wichtiger Grundkonzepte des JADE Framworks vorgestellt. 