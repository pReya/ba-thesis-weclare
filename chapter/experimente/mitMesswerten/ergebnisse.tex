\begin{table}[H]
    \centering
    \begin{tabular}{c|c|c|c|c|c|c}
       \textbf{Experiment} & \textbf{Gelöst} & \textbf{Prioritäts-Überlauf}
        & \(\textbf{s\textsubscript{opt}}\) & \(\textbf{s\textsubscript{CoDy}}\)
        & \(\textbf{t\textsubscript{opt}}\) & \(\textbf{t\textsubscript{CoDy}}\)\\ \hline
       \textbf{6-vs-6-locker} & 30 von 30 & 0 von 30
       & 19.5 & 20.97 & 22 & 24.33 \\ \hline
       \textbf{6-vs-6-eng} & 26 von 30 & 1 von 30
       & 5.45 & 11.5 & 10.35 & 21.89 \\ \hline
       \textbf{3-vs-3-eng} & 30 von 30 & 3 von 30
       & 5.5 & 6.35 & 6.33 & 7.68 \\ \hline
       \textbf{4-vs-4-eng} & 27 von 30 & 4 von 30
       & 5 & 10.85 & 10.5 & 24.96
    \end{tabular}
    \caption{Messwerte der Experimente aus \cite{book:regele}}
    \label{tab:resultsCoDy}
\end{table}

\begin{table}[H]
    \centering
    \begin{tabular}{c|c|c|c|c}
       \textbf{Experiment} & \textbf{Gelöst} & \textbf{Prioritäts-Überlauf}
        & [\textbf{\(\bar{s}\textsubscript{u}\)}, \textbf{\(\bar{s}\textsubscript{o}\)}]
        & [\textbf{\(\bar{t}\textsubscript{u}\)}, \textbf{\(\bar{t}\textsubscript{o}\)}]\\ \hline
       \textbf{6-vs-6-locker} & 30 von 30 & 0 von 30 & [20.01, 22.33] & [23.92, 26.4]\\ \hline
       \textbf{6-vs-6-eng} & 18 von 30 & 18 von 18 & [8.78, 9.81] & [19.52, 23.09]\\ \hline
       \textbf{3-vs-3-eng} & 27 von 30 & 0 von 27 & [7.3, 7.58] & [9.53, 9.97]\\ \hline
       \textbf{4-vs-4-eng} & 20 von 30 & 20 von 20 & [13.72, 25.2] & [23.04, 41.04]
    \end{tabular}
    \caption{Messwerte der selbst durchgeführten Experimente}
    \label{tab:myResults}
\end{table}

Die eigenen Messwerte weichen teils stark von den vorgegeben Messwerten ab. Lediglich das Experiment "'\ref{chap:6x6_locker} Sechs gegen sechs: Lockere Vorbeifahrt"' trifft die Erwartungen.

Für die eigenen Messungen ist anzumerken, dass für den Prioritätsüberlauf nicht immer "'von 30"' angegeben ist. Das hat damit zu tun, dass die Simulationsumgebung nur Daten erzeugt, wenn ein Experiment erfolgreich war.

\begin{figure}[H]
    \includegraphics[height=40mm]{images/6vs6_tight_full_block.png}
    \centering
    \caption{Beispielhafte Totalblockade für das Experiment "'\ref{chap:6x6_eng} Sechs gegen sechs: Enge Vorbeifahrt"'}
    \label{fig:6x6EngFullBlock}
\end{figure}
Für das Experiment "'\ref{chap:6x6_eng} Sechs gegen sechs: Enge Vorbeifahrt"' ist der zurückgelegte Strecke etwas besser als erwartet. Der Zeitbedarf trifft die Erwartung. Auffällig ist jedoch, dass nur 18 statt 26 Wiederholungen erfolgreich sind und es in allen, statt nur bei einer Wiederholung, zu Prioritätsüberläufen kam. Bei den zwölf ungelösten Wiederholungen ist eine Variation der in Abbildung \ref{fig:6x6EngFullBlock} gezeigten Situation eingetreten. 

Die Werte für "'\ref{chap:3x3_eng} Drei gegen drei: Enge Vorbeifahrt"'
weichen nur leicht von den erwarteten ab. Statt 30 werden nur 27 Wiederholungen gelöst. Dafür kommt es in den gelösten Wiederholungen aber nicht zu Prioritätsüberläufen. Die drei erwarteten Wiederholungen, die einen Prioritätsüberläufen haben, entstehen aus Situationen, in denen sich die Agenten in breiter Front aufeinander zu bewegen. In den eigenen Experimenten führten genau diese Situation zu den drei ungelösten Wiederholungen.

Die gemessenen Werte weichen für das Experiment "'\ref{chap:4x4_eng} Vier gegen vier: Enge Vorbeifahrt"' am stärksten ab. Es werden nur in 20 statt 27 Wiederholungen Lösungen gefunden. In jeder Wiederholung kommt es zu Prioritätsüberläufen. Auch das war nicht erwartet. Die Konfidenzintervalle für die durchschnittlich zurückgelegte Strecke und den durchschnittlichen Zeitbedarf sind im Vergleich besonders groß. Die erwartete Strecke ist nicht mal im Intervall enthalten. Wie Abbildungen \ref{fig:4x4_prio_2} deutlich zeigt, wird der erwartete Prioritätsverlauf verfehlt. Statt das die Priorität der Agenten über die Zeit stetig ansteigt und teilweise wieder zurückgeht, pendeln bei den eigenen Experimenten die Prioritätswerte zwischen minimalen und maximalen Werten hin und her.

\begin{figure}[H]
    \includegraphics[width=\textwidth]{images/4vs4_tight_prio_ref.png}
    \centering
    \caption{Prioritätsverlauf aller Agenten für das Experiment "'\ref{chap:4x4_eng} Vier gegen vier: Enge Vorbeifahrt"' aus \cite{book:regele}}
    \label{fig:4x4_prio_1}
\end{figure}

\begin{figure}[H]
    \includegraphics[width=\textwidth]{images/4vs4_tight_prio.png}
    \centering
    \caption{Prioritätsverlauf aller Agenten für das Experiment "'\ref{chap:4x4_eng} Vier gegen vier: Enge Vorbeifahrt"' der selbst durchgeführten Experimente}
    \label{fig:4x4_prio_2}
\end{figure}