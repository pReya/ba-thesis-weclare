Nach der Analyse zwei bestehender Lösungen sollen nun Kern-Anforderungen formuliert werden, die ein modernes Audience-Response-System für den Einsatz in der Informatiklehre ausmachen:

\section{Webbasiert}
\label{chap:webbasiert}
Eine modernes Audience-Response-System sollte vollständig webbasiert daherkommen. Die Zugänglichkeit einer Web-Applikation die vollständig im Browser und ohne Downloads verwendet werden kann schlägt viele ...
Ein Classroom-Response-System stellt für heutige Computersysteme keine große Lastsituation dar. Performance-Probleme müssen bei einer Web-Lösung nicht befürchtet werden.
\section{Peer 2 Peer}
\label{chap:p2p}
Um die Langlebigkeit eines CRS-Systems zu erhöhen sollte der Wartungsaufwand einer solchen Software möglichst gering ausfallen. Der Betrieb eines eigenen Servers widerspricht dieser Philosophie. Daher sollte ein modernes CRS komplett auf Peer-2-Peer-Basis funktionieren.

\section{Code Formatierungen}
\label{chap:codeformatierung}

\section{Code Ausführung}
\label{chap:codeausfuehrung}
