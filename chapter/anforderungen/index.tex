Nach der Analyse zwei bestehender Lösungen sollen nun Kern-Anforderungen formuliert werden, die ein modernes Audience-Response-System für den Einsatz in der Informatiklehre ausmachen:

\section{Webbasiert}
\label{chap:webbasiert}
Eine modernes Audience-Response-System sollte vollständig webbasiert daherkommen. Die Zugänglichkeit einer Web-Applikation, die vollständig im Browser und ohne Downloads verwendet werden kann ist gerade für den Einsatz in der Universität wichtig. Der Download oder die Installation von zusätzlicher Software hebt die Hürde zum Einsatz der Software unnötigerweise an.

Außerdem stellt ein Classroom-Response-System im Einsatz mit üblichen Gruppengrößen für heutige Computersysteme keine große Lastsituation mehr dar. Eventuelle Performance-Probleme sollten in diesem Fall keine Rolle spielen.

\section{Variable Client-Server-Architektur}
\label{chap:p2p}
Um die Langlebigkeit eines CRS-Systems zu erhöhen sollte der Wartungsaufwand einer solchen Software möglichst gering ausfallen. Außerdem sollte die Hürde gegenüber Dozenten zum Einsatz der Software möglichst weit gesenkt werden. Der Betrieb eines eigenen, dedizierten Servers widerspricht dieser Philosophie. Daher sollte ein modernes CRS größtenteils ohne dedizierten Server funktionieren, und stattdessen auf Direktverbindungen unter den Teilnehmern setzen.

\section{Code-Formatierung}
\label{chap:codeformatierung}
Um die Darstellung von Programm-Ausschnitten zu verbessern sollte ein CRS für den Informatik-Bereich gut mit Code-Beispielen umgehen können. Dazu gehört, dass Code möglichst übersichtlich dargestellt wird. Das spricht für die Unterstützung von Syntax-Highlighting für längere Ausdrücke und Abschnitte sowie die Ermöglichung von Monospace-Formatierung um technische Begriffe im Fließtext ordentlich kennzeichnen zu können.

\section{Interaktive Code-Ausführung}
\label{chap:codeausfuehrung}
Um die Benutzung eines CRS zu erleichtern, sollte der Dozent nicht parallel eine Entwicklungsumgebung geöffnet haben müssen, um das Ergebnis von Code-Ausführungen zu demonstrieren. Die Ausführung direkt im Browser sollte ermöglicht werden.

Gleichzeitig soll das neue System kompatibel zu StuReSy sein. Aus diesem Grund wird ein Kommandozeilen-Werkzeug entwickelt, welches Fragensätze vom XML-basierten StuReSy-Formt in das JSON-basierte Weclare-Format konvertiert. https://github.com/pReya/weclare-sturesy-converter
