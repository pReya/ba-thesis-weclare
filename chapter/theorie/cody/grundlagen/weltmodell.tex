Das Weltmodell ist zeitlich und räumlich diskret. Der CoDy Algorithmus kann mit jedem Weltmodell arbeiten, das sich als ungerichteter Graph darstellen lässt. Um die Berechnungen und Anschauungen aber nicht unnötig komplex zu gestalten, beschränkt sich das Weltmodell auf ein klassisches zweidimensionales Gitter. Die Zellen des Gitters sind quadratisch. Die Nachbarschaftsbeziehungen der Zellen beschränken sich auf vier Himmelsrichtungen. Sprich Zellen, die diagonal zu einander liegen, gelten nicht als benachbart. Eine Zelle kann verschiedene Zustände einnehmen. Die Zustände variieren jedoch zwischen Kartentypen. Sie werden deshalb in den folgenden Kapiteln erörtert. Ein weiteres Attribut der Zellen ist, dass sie eine bestimmte räumliche Größe haben. Für diese Arbeit wird jedoch angenommen, dass die Zellen ein wenig größer als die Agenten sind. Dies ist das einfachst anzunehmende Modell. \cite{book:regele}