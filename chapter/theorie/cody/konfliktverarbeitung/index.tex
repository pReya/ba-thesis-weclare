Die Agenten des \textit{CoDy Algorithmus} sind kooperativ. Sie sind also in der Lage die eigenen "'Bedürfnisse"' zurückzustellen. Man kann zwischen passiver und aktiver Kooperation unterscheiden. Bei der aktiven Kooperation kann ein Agent die Planung übernehmen und bei anderen Agenten nach Zustimmung oder Ablehnung fragen. Bei der passiven Kooperation müssen alle in einem Konflikt beteiligten Agenten die gesamte Situation berechnen und dann anschließend in einer Verhandlung ihre Lösungen diskutieren. Die passive Kooperation hat gegenüber der aktiven Kooperation vor allem den Vorteil, dass der Kommunikationsaufwand deutlich geringer ist. Sie ist aber auch robuster und der Berechnungsaufwand, so wie sie im CoDy Algorithmus umgesetzt ist, geringer. Die echte passive Kooperation kann sehr ineffizient sein. Deshalb benutzt der CoDy Algorithmus eine abgewandelte Form. Agenten, die potenziell in einen Konflikt geraten können, kommunizieren in einer festen Reihenfolge (siehe Kapitel \ref{chap:kommunikation}) und planen nur ihre eigenen Wege. \cite{book:regele}\newline Folgende Beispielsituation:
Agent "'0"' überschreibt einen Teil des geplanten Weges von Agent "'1"'. Dann schickt Agent "'0"' seinen neuen Plan an alle Agenten. Wenn Agent "'1"' wieder an der Reihe ist, plant dieser mit den neuen Daten. Wann welcher Agent, welche Wege überschreiben kann, wird in Kapitel \ref{chap:prioritaeten} beschrieben.

\subsection{Kommunikation}
\label{chap:kommunikation}
Wie bereits erwähnt, kommunizieren die Agenten in einer festen Reihenfolge. In diesem Kapitel wird beschrieben, wie sich dieser verkette Ablauf ergibt.
Jeder Agent merkt sich mit einem Zeitstempel, wann er mit seinem Berechnungsschritt begonnen hat. Nachdem ein Agent seinen Weg geplant hat, verschickt er den geplanten Weg und den Zeitstempel an alle Agenten und wartet auf Nachrichten der anderen Agenten. Wenn ein Agent eine Nachricht erhält, wird zunächst geprüft, ob sich die Umgebungskarten überschneiden. Wenn dies nicht der Fall ist, wird die Nachricht verworfen, da die Agenten sich nicht beeinflussen. Wenn aber eine Überschneidung festgestellt wird, werden die Zeitstempel mit einander verglichen. Hat der andere Agent früher mit seiner Wegplanung begonnen, muss auf diesen gewartet werden, bevor mit dem Planen der nächsten Wegschritte begonnen wird. Sind beide Zeitstempel gleich, entscheidet die numerische Identifikationsnummer der Agenten. Kleine Nummern sind besser. Somit ergibt sich ein zufälliger, aber fester verketteter Ablauf. Agenten reihen sich also in diesen Ablauf ein, wenn sich die Umgebungskarten überschneiden. Wenn sich die Agenten aber soweit von einander entfernen, dass keine Überschneidung mehr vorliegt, werden sie aus der Kette gelöscht. Durch dieses Vorgehen ist die Entwicklung der Prioritäten, das Hauptwerkzeug der Agenten, um Konflikte zu lösen, vorhersehbar. Dieser Vorteil wiegt mehr, als der Nachteil für diese Bereiche die parallele Berechnung aufzugeben. \cite{book:regele}
%
\subsection{Prioritäten}
\label{chap:prioritaeten}
Agenten sind in der Lage geplante Wege anderer Agenten zu überschreiben. Dies wird durch die Prioritätswerte der Agenten ermöglicht. Wenn die Erreichbarkeitskarte erstellt wird, werden Zellen die von Agenten mit niedrigerer Priorität als die eigene reserviert sind als frei betrachtet. Bei Planen des Wegs wird aber darauf geachtet, dass möglichst wenig Wege überschrieben werden. Wenn zum Beispiel zwei Zellen die Entfernung zum Ziel gleicher maßen verringern und eine von diesen Zellen tatsächlich frei ist, dann wird diese gewählt. \cite{book:regele}

Die dynamische Anpassung der Prioritäten passiert genau dann, wenn ein Konflikt beziehungsweise eine Blockade erkannt wird. Die Konflikterkennung ist dabei recht simpel. Um einen Konflikt zu erkennen, betrachtet man den vom Agenten geplanten Weg. Wenn jeder geplante Schritt den Abstand zum Ziel verringert, liegt \textbf{keine Blockade} vor. Der Prioritätswert wird um den Wert \(PrioNoBlock\) dekrementiert. Der Prioritätswert eines Agenten kann dabei den Wert \(BasePrio\) nicht unterschreiten. Das Dekrementieren des Prioritätswert geschieht, da der Agent eventuell andere Agenten blockiert. Wenn der geplante Weg Schritte enthält, die die Entfernung zum Ziel nicht verbessern, der Agent aber bis zum Schluss in Bewegung bleibt, handelt es sich um einen \textbf{Umweg}. In diesem Fall wird die Priorität nicht verändert, da die Agenten ja kooperativ arbeiten und Umwege in Kauf genommen werden sollen. Wenn der geplante Weg in einer Wartephase endet, handelt es sich um eine \textbf{Blockade}. Da es Agenten aber möglich sein soll, andere Agenten verdrängen zu können, wird eine Blockade nur als solche erkannt, wenn in zwei aufeinander folgenden Berechnungsschritten die gleiche Blockade erkannt wird. In diesem Fall wird die Priorität des Agenten um \(PrioBlock\) erhöht. Mit dieser neuen Priorität plant der Agent erneut seinen Weg. Wichtig ist, dass die Priorität sich nur einmal erhöht. Diese erneute Planung ist für den Berechnungsschritt also die Finale. Wenn kein Weg gefunden wird, also nicht einmal das Verharren auf der eigenen Position möglich ist, dann liegt eine \textbf{total Blockade vor}. In diesem Fall wird die Priorität um \(PrioFullBlock\) erhöht und die Wegplanung erneut durchgeführt. Wenn immer noch kein gültiger Wegeplan berechnet werden konnte, wird ein Notweg erzwungen. Der Notweg ist das Verharren auf der eigenen Position. Zusätzlich wird die Priorität maximal erhöht, damit der Notweg nicht direkt überschrieben werden kann. Für das Überschreiben der Wege ist noch wichtig, dass die ersten drei Zeitschritte geschützt sind. In diesen ersten Zeitschritten dürfen keine Wege anderer Agenten überschrieben werden. Diese Regel gilt jedoch nicht für das Erzwingen eines Notweges. \cite{book:regele}


