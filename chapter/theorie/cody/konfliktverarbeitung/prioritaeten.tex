Agenten sind in der Lage geplante Wege anderer Agenten zu überschreiben. Dies wird durch die Prioritätswerte der Agenten ermöglicht. Wenn die Erreichbarkeitskarte erstellt wird, werden Zellen die von Agenten mit niedrigerer Priorität als die eigene reserviert sind als frei betrachtet. Bei Planen des Wegs wird aber darauf geachtet, dass möglichst wenig Wege überschrieben werden. Wenn zum Beispiel zwei Zellen die Entfernung zum Ziel gleicher maßen verringern und eine von diesen Zellen tatsächlich frei ist, dann wird diese gewählt. \cite{book:regele}

Die dynamische Anpassung der Prioritäten passiert genau dann, wenn ein Konflikt beziehungsweise eine Blockade erkannt wird. Die Konflikterkennung ist dabei recht simpel. Um einen Konflikt zu erkennen, betrachtet man den vom Agenten geplanten Weg. Wenn jeder geplante Schritt den Abstand zum Ziel verringert, liegt \textbf{keine Blockade} vor. Der Prioritätswert wird um den Wert \(PrioNoBlock\) dekrementiert. Der Prioritätswert eines Agenten kann dabei den Wert \(BasePrio\) nicht unterschreiten. Das Dekrementieren des Prioritätswert geschieht, da der Agent eventuell andere Agenten blockiert. Wenn der geplante Weg Schritte enthält, die die Entfernung zum Ziel nicht verbessern, der Agent aber bis zum Schluss in Bewegung bleibt, handelt es sich um einen \textbf{Umweg}. In diesem Fall wird die Priorität nicht verändert, da die Agenten ja kooperativ arbeiten und Umwege in Kauf genommen werden sollen. Wenn der geplante Weg in einer Wartephase endet, handelt es sich um eine \textbf{Blockade}. Da es Agenten aber möglich sein soll, andere Agenten verdrängen zu können, wird eine Blockade nur als solche erkannt, wenn in zwei aufeinander folgenden Berechnungsschritten die gleiche Blockade erkannt wird. In diesem Fall wird die Priorität des Agenten um \(PrioBlock\) erhöht. Mit dieser neuen Priorität plant der Agent erneut seinen Weg. Wichtig ist, dass die Priorität sich nur einmal erhöht. Diese erneute Planung ist für den Berechnungsschritt also die Finale. Wenn kein Weg gefunden wird, also nicht einmal das Verharren auf der eigenen Position möglich ist, dann liegt eine \textbf{total Blockade vor}. In diesem Fall wird die Priorität um \(PrioFullBlock\) erhöht und die Wegplanung erneut durchgeführt. Wenn immer noch kein gültiger Wegeplan berechnet werden konnte, wird ein Notweg erzwungen. Der Notweg ist das Verharren auf der eigenen Position. Zusätzlich wird die Priorität maximal erhöht, damit der Notweg nicht direkt überschrieben werden kann. Für das Überschreiben der Wege ist noch wichtig, dass die ersten drei Zeitschritte geschützt sind. In diesen ersten Zeitschritten dürfen keine Wege anderer Agenten überschrieben werden. Diese Regel gilt jedoch nicht für das Erzwingen eines Notweges. \cite{book:regele}


