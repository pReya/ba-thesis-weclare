Wie bereits erwähnt, kommunizieren die Agenten in einer festen Reihenfolge. In diesem Kapitel wird beschrieben, wie sich dieser verkette Ablauf ergibt.
Jeder Agent merkt sich mit einem Zeitstempel, wann er mit seinem Berechnungsschritt begonnen hat. Nachdem ein Agent seinen Weg geplant hat, verschickt er den geplanten Weg und den Zeitstempel an alle Agenten und wartet auf Nachrichten der anderen Agenten. Wenn ein Agent eine Nachricht erhält, wird zunächst geprüft, ob sich die Umgebungskarten überschneiden. Wenn dies nicht der Fall ist, wird die Nachricht verworfen, da die Agenten sich nicht beeinflussen. Wenn aber eine Überschneidung festgestellt wird, werden die Zeitstempel mit einander verglichen. Hat der andere Agent früher mit seiner Wegplanung begonnen, muss auf diesen gewartet werden, bevor mit dem Planen der nächsten Wegschritte begonnen wird. Sind beide Zeitstempel gleich, entscheidet die numerische Identifikationsnummer der Agenten. Kleine Nummern sind besser. Somit ergibt sich ein zufälliger, aber fester verketteter Ablauf. Agenten reihen sich also in diesen Ablauf ein, wenn sich die Umgebungskarten überschneiden. Wenn sich die Agenten aber soweit von einander entfernen, dass keine Überschneidung mehr vorliegt, werden sie aus der Kette gelöscht. Durch dieses Vorgehen ist die Entwicklung der Prioritäten, das Hauptwerkzeug der Agenten, um Konflikte zu lösen, vorhersehbar. Dieser Vorteil wiegt mehr, als der Nachteil für diese Bereiche die parallele Berechnung aufzugeben. \cite{book:regele}