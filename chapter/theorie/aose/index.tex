Ein Agent ist für die AoSE, dass was ein Objekt für die OOP ist. AoSE ist, so wie die OOP auch, ein Programmierparadigma. Um AoSE zu verstehen, wird folgend der Begriff des Agenten definiert.

\subsection{Agent}
\label{chap:aose_agent}
Es existiert keine universelle Definition für Agenten \cite{book:padgham}. Jedoch gibt es Eigenschaften die einen Agenten beschreiben, die von diversen Definitionen aufgegriffen werden. Ein Agent ist:
\begin{itemize}
\item \textbf{zielorientiert.} Er hat ein oder mehrere Ziele. Sein Verhalten ist darauf ausgelegt, diese zu erfüllen.
\item \textbf{autonom.} Ein Agent kontrolliert seinen internen Status und entscheidet selbständig, ob und welche Aktion er ausführt.
\item \textbf{reaktiv.} Auf Änderungen der Umwelt kann reagiert werden.
\item \textbf{proaktiv.} Der Agent ergreift Eigeninitiative. Es sind also keine Befehle von außen notwendig, um zu handeln.
\item \textbf{sozial.} Er interagiert "'menschenähnlich"' mit anderen Agenten. Das bedeutet, dass Agenten Verhandeln, Koordinieren, Kooperieren und so weiter.
\end{itemize}\cite{book:padgham}\cite{book:jade}\cite{article:flexibleSoftware}

Eingangs wurde der Vergleich mit OOP gemacht. Objekte und Agenten teilen sich tatsächlich viele Merkmale. Objekte kapseln ihre Identität und ihren Status. Sie kontrollieren aber nicht ihr Verhalten. Die Methoden der Objekte werden von außen und potenziell zu jeder Zeit aufgerufen. Agenten kapseln dies jedoch zusätzlich. Sie verfügen selbständig über ihre Methoden. Sie treten aber auch in komplexe Verhandlungen mit anderen Agenten. Zwar können die Konzepte des AoSE auch mit OOP implementieren, jedoch fehlen der OOP wichtige Konzepte. \cite{article:flexibleSoftware} Das Verhältnis von AoSE zur OOP ist so ähnlich, wie das Verhältnis zwischen der iterativen und der OOP. Weshalb die AoSE auch als nächster evolutionärer Schritt bezeichnet wird \cite{article:objectsVsAgents}.


Über die Zeit wurde es immer wichtiger lose Kopplung und hohe Kapselung in Software zu erreichen. AoSE führt diesen Trend konsequent weiter, indem eine Entität seinen eigen Kontrollfluss kontrolliert und damit seine Absichten kapselt. Diese Attribute gewinnen an Gewicht, je komplexer ein Softwaresystem ist. \cite{article:flexibleSoftware}.

In dieser Arbeit wird mit Hilfe der AoSE ein Multi-Agenten-System entwickelt.