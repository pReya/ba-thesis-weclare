Es existiert keine universelle Definition für Agenten \cite{book:padgham}. Jedoch gibt es Eigenschaften die einen Agenten beschreiben, die von diversen Definitionen aufgegriffen werden. Ein Agent ist:
\begin{itemize}
\item \textbf{zielorientiert.} Er hat ein oder mehrere Ziele. Sein Verhalten ist darauf ausgelegt, diese zu erfüllen.
\item \textbf{autonom.} Ein Agent kontrolliert seinen internen Status und entscheidet selbständig, ob und welche Aktion er ausführt.
\item \textbf{reaktiv.} Auf Änderungen der Umwelt kann reagiert werden.
\item \textbf{proaktiv.} Der Agent ergreift Eigeninitiative. Es sind also keine Befehle von außen notwendig, um zu handeln.
\item \textbf{sozial.} Er interagiert "'menschenähnlich"' mit anderen Agenten. Das bedeutet, dass Agenten Verhandeln, Koordinieren, Kooperieren und so weiter.
\end{itemize}\cite{book:padgham}\cite{book:jade}\cite{article:flexibleSoftware}

Eingangs wurde der Vergleich mit OOP gemacht. Objekte und Agenten teilen sich tatsächlich viele Merkmale. Objekte kapseln ihre Identität und ihren Status. Sie kontrollieren aber nicht ihr Verhalten. Die Methoden der Objekte werden von außen und potenziell zu jeder Zeit aufgerufen. Agenten kapseln dies jedoch zusätzlich. Sie verfügen selbständig über ihre Methoden. Sie treten aber auch in komplexe Verhandlungen mit anderen Agenten. Zwar können die Konzepte des AoSE auch mit OOP implementieren, jedoch fehlen der OOP wichtige Konzepte. \cite{article:flexibleSoftware} Das Verhältnis von AoSE zur OOP ist so ähnlich, wie das Verhältnis zwischen der iterativen und der OOP. Weshalb die AoSE auch als nächster evolutionärer Schritt bezeichnet wird \cite{article:objectsVsAgents}.
