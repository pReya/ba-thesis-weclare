In diesem Kapitel wird eine Strategie formuliert, mit der die Vereinigung des JADE Framework und des CoDy Algorithmus validiert werden kann.

In \cite{book:regele} ist eine Vielzahl an Experimenten definiert. Diese werden in einer geeigneten Simulationsumgebung wiederholt und mit den Ergebnissen aus \cite{book:regele} verglichen. Wenn diese Experimente wiederholt werden können, dann erfüllt auch das JADE Framework seinen Zweck. Denn nur wenn das Scheduling und die Kommunikation der Agenten korrekt funktioniert, kann auch der CoDy Algorithmus funktionieren. Dass die eigenen Messwerte oder Beobachtungen exakt mit denen aus \cite{book:regele} übereinstimmen, ist relativ unwahrscheinlich. Das liegt daran, dass für einige wichtige Funktionen des CoDy Algorithmus keine exakten Informationen der Implementierung vorliegen (siehe TODO link auf CoDy Kapitel mit den Abweichungen). Zusätzlich wurde auch nur eine relativ kleine Stichprobe von 30 Wiederholungen pro Experiment gemessen. Aus den möglichen hunderten von Millionen Kommunikationsreihenfolgen \cite{book:regele}, die 30 gleichen oder ähnlichen zu erwischen, ist sehr unwahrscheinlich. Für die Messwerte bedeutet das, dass eine Annäherung der Werte ausreicht, um eine fundierte Aussage über die Qualität der Implementierung zu treffen. Von den Experimenten, die durchgeführt werden, sind aber für die Mehrheit keine Messwerte dokumentiert. Hier liegen dann allgemeine Beobachtungen vor. Für die Validierung der Implementation sollen diese genauso beobachtet werden. Falls es hier zu Abweichungen kommt, bedarf es einer tieferen Analyse beziehungsweise einer genauen Erklärung.