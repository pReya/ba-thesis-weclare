Der Weclare-Prototyp zeigt, dass es möglich ist, dank einer JVM-Implementierung in JavaScript einen Java-Quelltext in der Browser-Umgebung zu kompilieren und anschließend auszuführen, ohne dafür einen Compiler-Server zu verwenden. Gleichzeitig zeigen sich bei dieser Vorgehensweise auch deutliche Probleme:

\begin{itemize}
    \item \textbf{Größe der Java Runtime:} Die Java-Laufzeitumgebung ist etwa 62 Megabyte groß und muss erst vom Client heruntergeladen werden. Trotz schneller Internetverbindungen ist das eine beträchtliche Datenmenge, die auch die mobile Nutzung der Code-Ausführung einschränkt.
    \item \textbf{Performance:} Die DoppioJVM ist um ein Vielfaches langsamer als eine native JVM. Das hat mehrere Gründe, zum Beispiel die Qualität der Implementierung oder der geringere Grad der Optimierung (Ahead-Of-Time-kompilierte vs. interpretierte Sprache).
\end{itemize}

Wäre diese Arbeit zwei oder drei Jahre später entstanden, hätte es vermutlich eine ernstzunehmende Alternative zur Realisierung der Code-Ausführbarkeit gegeben: WebAssembly.

WebAssembly ist ein Standard der bereits von allen großen Browser-Herstellern unterstützt wird (Microsoft, Mozilla, Apple, Google). Es handelt sich bei WebAssembly um ein sogenanntes Compile-Target. Also eher nicht um eine Programmiersprache, die manuell von Entwicklern geschrieben wird (obwohl das möglich ist), sondern um ein Ausgabeformat eines Compilers. Viele andere, beliebte Sprachen können nach WebAssembly kompiliert werden. Gut unterstützt wird das momentan zum Beispiel von den Sprachen C. C++ und Rust. WebAssembly erfüllt damit die gleiche Aufgabe wie Java-Bytecode: Es ist ein plattformunabhängiges und kompaktes Binärformat, das in einer virtuellen Maschine ausgeführt werden kann.

Damit ermöglicht WebAssembly zukünftig ganz neue Möglichkeiten: Software-Entwickler können Web-Anwendungen in beliebigen Sprachen verfassen und sind nicht mehr an JavaScript gebunden. Außerdem können bestehende Anwendungen relativ einfach in die Browser-Umgebung portiert werden und die Leistungsfähigkeit von Web-Anwendungen gesteigert werden (WASM kann als Ahead-of-Time-kompilierte Sprache tendenziell schneller ausgeführt werden als eine interpretierte und JIT-kompilierte Sprache wie JavaScript).

Es gibt bereits Compiler, die Java-Bytecode nach WebAssembly konvertieren können (zum Beispiel TeaVM\footnote{Offizielle Webseite: \url{http://teavm.org/}}). Allerdings ist der TeaVM-Compiler selbst in Java geschrieben und dafür gedacht, vorab und einmalig ausgeführt zu werden, um eine Java-Anwendung im Browser ausführen zu können. Der Compiler selbst kann also noch nicht im Browser ausgeführt werden. Theoretisch ließe sich dieser Compiler selbst nach WebAssembly kompilieren und könnte dann im Browser funktionieren. Praktisch ist das allerdings aufgrund vieler kleiner Einschränkungen des (noch relativ jungen) WebAssembly-Standards noch nicht so einfach möglich.

Mit steigender Popularität und steigendem Funktionsumfang von WebAssembly ist es sehr wahrscheinlich dass es bald einen Java-zu-WebAssembly-Compiler geben wird, der selbst in WebAssembly implementiert wurde oder problemlos nach WebAssembly übersetzt werden kann. Damit ließe sich dann im Browser Java-Quellcode in WebAssembly konvertieren und ausführen ohne das lästige Laden der Java Laufzeitumgebung und mit tendenziell besserer Performance als DoppioJVM (vgl. \cite{art:wasm_speed}).

Mit WebAssembly wäre es außerdem leichter möglich, das Feature der Code-Ausführung im Browser um weitere Sprachen zu erweitern und damit den potenziellen Nutzerkreis für Weclare zu erweitern.