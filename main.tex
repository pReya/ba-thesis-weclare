\documentclass[draft=false
              ,paper=a4
              ,twoside=false
              ,fontsize=11pt
              ,headsepline
              ,BCOR=10mm
              ]{scrbook}
\usepackage[ngerman,english]{babel}
%% see http://www.tex.ac.uk/cgi-bin/texfaq2html?label=uselmfonts
%% Fotns
\usepackage[T1]{fontenc}
\usepackage[utf8]{inputenc}
\usepackage{libertine}
\usepackage{pifont}
\usepackage{microtype}
\usepackage{inconsolata}
%%
\usepackage{textcomp}
\usepackage[german,refpage]{nomencl}
\usepackage[ngerman,colorlinks=true]{hyperref}
\usepackage{setspace}
\usepackage{makeidx}
\usepackage{listings}
\usepackage{csquotes}
\usepackage{acronym}
\usepackage{subcaption}
\usepackage{amsfonts}
\usepackage{tikz}
\usepackage[
    backend=biber,
    style=ieee,
    sorting=nyt,
]{biblatex}
\addbibresource{literature.bib}
\usepackage{soul}
\usepackage{hawstyle}
\usepackage{scrhack}
\usepackage{subfiles}
\usepackage{graphicx}
\usepackage{float}
\usepackage{parskip}

%% define some colors
\colorlet{BackgroundColor}{gray!20}
\colorlet{KeywordColor}{blue}
\colorlet{CommentColor}{black!60}
%% for tables
\colorlet{HeadColor}{gray!60}
\colorlet{Color1}{blue!10}
\colorlet{Color2}{white}

%% configure colors
\HAWifprinter{
  \colorlet{BackgroundColor}{gray!20}
  \colorlet{KeywordColor}{black}
  \colorlet{CommentColor}{gray}
  % for tables
  \colorlet{HeadColor}{gray!60}
  \colorlet{Color1}{gray!40}
  \colorlet{Color2}{white}
}{}
\definecolor{lightgray}{rgb}{.9,.9,.9}
\definecolor{darkgray}{rgb}{.4,.4,.4}
\definecolor{purple}{rgb}{0.65, 0.12, 0.82}
\lstdefinelanguage{JavaScript}{
  keywords={do, if, in, for, let, new, try, var, case, else, enum, eval, null, this, true, void, with, await, break, catch, class, const, false, super, throw, while, yield, delete, export, import, public, return, static, switch, typeof, default, extends, finally, package, private, continue, debugger, function, arguments, interface, protected, implements, instanceof},
  morecomment=[l]{//},
  morecomment=[s]{/*}{*/},
  morestring=[b]',
  morestring=[b]",
  ndkeywords={class, export, boolean, throw, implements, import, this},
  keywordstyle=\color{blue}\bfseries,
  ndkeywordstyle=\color{darkgray}\bfseries,
  identifierstyle=\color{black},
  commentstyle=\color{purple}\ttfamily,
  stringstyle=\color{red}\ttfamily,
  sensitive=true
}

\lstset{
   language=JavaScript,
   backgroundcolor=\color{BackgroundColor},
   extendedchars=true,
   basicstyle=\footnotesize\ttfamily,
   showstringspaces=false,
   showspaces=false,
   numbers=left,
   numberstyle=\footnotesize,
   numbersep=9pt,
   tabsize=2,
   breaklines=true,
   showtabs=false,
   captionpos=b
}
\ifpdfoutput{
  \hypersetup{bookmarksopen=false,bookmarksnumbered,linktocpage}
}{}

%% more fancy C++
\DeclareRobustCommand{\cxx}{C\raisebox{0.25ex}{{\scriptsize +\kern-0.25ex +}}}

\clubpenalty=10000
\widowpenalty=10000
\displaywidowpenalty=10000
\setcounter{secnumdepth}{3}

% unknown hyphenations
\hyphenation{
}

%% recalculate text area
\typearea[current]{last}

\makeindex
\makenomenclature

\begin{document}
\selectlanguage{ngerman}

%%%%%
%% customize (see readme.pdf for supported values)
\HAWThesisProperties{Author={Moritz Stückler}
                    ,Title={Entwurf und prototypische Implementierung eines webbasierten Classroom-Reponse-Systems für die Programmierlehre}
                    ,EnglishTitle={Design and prototypical Implementation of a web based Classroom Response System for Computer Science Education}
                    ,ThesisType={Bachelorarbeit}
                    ,ExaminationType={Bachelorprüfung}
                    ,DegreeProgramme={Bachelor of Science Technische Informatik}
                    ,ThesisExperts={Prof. Dr. Axel Schmolitzky \and Prof. Dr. Martin Becke}
                    ,ReleaseDate={9. Mai 2019}
                  }

%% title
\frontmatter

%% output title page
\maketitle

\onehalfspacing

%% add abstract pages
\HAWAbstractPage
{Audience Response Systeme, Classroom Response Systeme, StuReSy, Webentwicklung, React, Redux, WebRTC, JavaScript}
{Sogenannte Classroom-Response-Systeme (CRS) werden inzwischen an vielen Universitäten eingesetzt. Studierende können damit während einer Veranstaltung mit ihrem Smartphone oder anderen internetfähigen Geräten Fragen zum Inhalt der Veranstaltung beantworten. Im Kontext der Programmierlehre sind bisherige CRS-Lösungen allerdings häufig nicht ideal. Im Rahmen dieser Arbeit sollen bestehende CRS-Systeme auf ihre Eignung für den Einsatz in der Programmierlehre bewertet werden. Anschließend wird eine Software konzipiert und als Prototyp implementiert, die sich von bestehenden CRS in vier Kern-Aspekten unterscheidet:\newline
Sie ist vollständig \textbf{webbasiert} und läuft als JavaScript-Anwendung im Browser der beteiligten Parteien. Der Download von Software oder die Installation von Plugins ist nicht erforderlich.\newline
Die Verbindung zwischen Studenten und Dozent wird \textbf{direkt zwischen den beteiligten Browsern} hergestellt. Ein dediziertes Server-System ist nicht notwendig.\newline
Die Anwendung ist optimiert auf die \textbf{Darstellung und Einbettung von Quelltexten} (zum Beispiel durch Monospace-Formatierung und Syntax Highlighting).\newline
Außerdem können Java-Quelltexte, welche in die Fragestellungen eingebettet werden, direkt \textbf{im Browser ausgeführt werden}. Dazu wird eine Java Virtual Machine (implementiert in JavaScript) direkt im Browser ausgeführt.\newline
Abschließend wird bewertet wie gut die Umsetzung dieser Anforderungen gelungen ist, und an welchen Stellen für einen Praxiseinsatz noch nachgebessert werden sollte.}
{Audience Response Systems, Classroom Response Systems, StuReSy, Web development, React, Redux, WebRTC, JavaScript}
{So-called classroom response systems (CRS) are being used in many universities. Students can use their smartphone or other connected devices to answer questions about the class' contents. However many CRS are not ideally suited for computer science education. Within this thesis, existing CRS will be compared towards their computer science compatibility. After that, a new software will be devised and implemented as a prototype, which sets itself apart from other CRS in four main criteria:\newline
It is completely \textbf{web based} and runs as a JavaScript application within the browser of the involved parties. Downloading software or installing plugins is not necessary.\newline
The connection between the students and the instructor will be created in \textbf{directly between the browsers}. A dedicated server system will not be used.\newline
The application is optimised to display and embed Java source code by using \textbf{syntax highlighting} and monospace fonts.\newline
Additionally, Java source code which is embedded in the content of a question will be able tu run right in the browser. To achieve this, a Java Virtual Machine (implemented in JavaScript) will be run in the browser.}
\newpage
\singlespacing

\setcounter{tocdepth}{3}
\tableofcontents
\newpage
%% enable if these lists should be shown on their own page
\listoftables
\listoffigures
\lstlistoflistings
\begin{acronym}[StuReSy]
    \acro{crs}[CRS]{Classroom-Response-System}
    \acroplural{crs}[CRS]{Classroom-Response-Systeme}
    \acro{ars}[ARS]{Audience-Response-System}
    \acroplural{ars}[ARS]{Audience-Response-Systeme}
    \acro{sturesy}[StuReSy]{Student Response System}
    \acro{spa}[SPA]{Single-Page-Application}
    \acro{jvm}[JVM]{Java Virtual Machine}
    \acro{weclare}[Weclare]{Web Classroom Response (System)}
    \acro{wasm}[WASM]{WebAssembly}
\end{acronym}


%% main
\mainmatter
\onehalfspacing

%%%%%% Chapters
\chapter{Einleitung}
\section{Was sind Classroom-Response-Systeme?}
\label{chap:was_sind_crs}
Interaktivität ist ein Schlüsselfaktor für gute Lehrveranstaltungen (vgl. \cite[S. 1]{art:ieee}). Der klassische Frontalunterricht gilt nicht mehr als besonders effektiv. Gleichzeitig ist es für Lehrende immer schwieriger, besagte Interaktivität herzustellen, je größer die Anzahl der Teilnehmer einer Veranstaltung ist. In kleinen Gruppen kann durch mündliche Kommunikation meistens noch ein gewisser Grad an Interaktivität garantiert werden. Im Lehrbetrieb von großen Hochschulen und Universitäten ist das aber aufgrund der hohen Teilnehmerzahlen kaum noch realisierbar. (vgl. \cite[S. 389]{art:crs_informatik_spektrum})


Eine Möglichkeit, die Interaktivität zu steigern, ist der Einsatz sogenannter \acp{ars} (vgl. \cite[S. 5]{art:ieee}). Im akademischen und schulischen Umfeld werden diese Systeme auch noch spezifischer als \acp{crs} bezeichnet. Dabei handelt es sich um Hardware- oder Software-Lösungen, die es einem Dozenten ermöglichen, während einer Veranstaltung beliebig viele Fragen nacheinander an das Publikum zu stellen. Diese können entweder mithilfe dedizierter Hardware-Geräte (sogenannte Clicker) oder mit einem Smartphone, Laptop oder Tablet-PC beantwortet werden. Üblicherweise werden dabei verschiedene Fragetypen unterstützt, zum Beispiel Simple- und Multiple-Choice-Varianten.

Ein Merkmal dieser Systeme ist die Anonymität der Antworten, denn nur so können die Teilnehmer ehrlich antworten und lernen, ihren Wissensstand tatsächlich besser einzuschätzen (vgl. \cite[S. 106]{art:crs_literature_review}). CRS sind explizit keine Werkzeuge, um digitale Tests oder Prüfungen durchzuführen.

Die Nutzung solcher Systeme im akademischen Umfeld ist bereits relativ weit verbreitet (vgl. \cite[S. 392]{art:crs_informatik_spektrum}. Weltweit werden diese Lösungen an vielen Hochschulen eingesetzt (vgl. \cite{web:elanwiki}). Studien und Umfragen attestieren diesen Systemen eine Steigerung der Interaktivität, eine Verbesserung der Selbsteinschätzung und eine Steigerung der Aufmerksamkeit in Lehrveranstaltungen, zum Beispiel an der University of Wisconsin-Milwaukee \cite[S. 5]{paper:wisconsin_clicker}:

\begin{quote}
„Students similarly reported that the use of clickers increased their engagement, involvement, and interaction, and help students pay attention in class.“
\end{quote}

Oder auch nach zehnjärigem Einsatz an der Harvard University \cite[S. 6]{paper:harvard_tenyears}:

\begin{quote}
„We find that, upon first implementing Peer Instruction, our students’ scores on the Force Concept Inventory and the Mechanics Baseline Test improved dramatically, and their performance on traditional quantitative problems improved as well.“
\end{quote}


Eine Bewertung des didaktischen Konzepts von CRS ist daher nicht Bestandteil dieser Arbeit. Die Wirksamkeit und Relevanz wird als gegeben angenommen.

Auch an der \ac{haw} Hamburg werden CRS eingesetzt. Eine der verwendeten Software-Lösungen, die dort im Department Informatik zum Einsatz kommt, ist das \ac{sturesy} – eine Software, die im Rahmen einer Bachelorarbeit an der Universität Hamburg entwickelt wurde \cite{sturesy}. Wegen der positiven Erfahrungen, die der Autor und Herr Axel Schmolitzky, einer der Lehrenden an der HAW Hamburg, mit der Software gemacht haben, wird StuReSy im Lauf dieser Arbeit als funktionales Vorbild für die Entwicklung eines neuen CRS dienen, welches das System konzeptionell verbessert, bestehende Probleme behebt und das Anwendungsgebiet stärker auf den Spezialfall der Programmierlehre zuspitzt.

\newpage
\section{Motivation: Vom Software-Download zur Web-Applikation}
\label{chap:webapp}
Die Disziplin der Software-Entwicklung hat in den letzten Jahrzehnten enorme Sprünge gemacht. Die rasante Leistungssteigerung von Hardware, die zunehmende Mobilität von Computersystemen bei gleichzeitiger Preisreduzierung und die Zunahme und Qualität der Vernetzung haben Software-Entwicklung radikal beschleunigt und verändert. Ein fundamentales Prinzip hat sich allerdings nur wenig verändert: Die Distribution von Software.

Seit den Anfängen der PC-Ära in den 1980er Jahren bis in die heutige Zeit wird Software zu einem bestimmten Zeitpunkt auf ein Zielsystem übertragen und dann lokal auf dem System gestartet und ausgeführt. Nur einige Details an diesem Prozess haben sich im Laufe der Jahre geändert:

Während früher ein physischer Datenträger die Quelle der Software war, ist es heute meistens das Internet. Früher mussten die Quellen manuell im Internet gefunden werden, heute gibt es Software-Kataloge oder App-Stores, welche die Suche nach Software erleichtern. Diese Entwicklung 

Seit wenigen Jahren etabliert sich eine alternative Variante der Software-Distribution: die Web-Applikation. Das Internet, das ursprünglich nur dazu gedacht war, statische Texte darzustellen, hat sich zu einer beliebten Art der Software-Distribution verwandelt. Dank der Programmiersprache JavaScript, sind Webbrowser heute mächtige Laufzeitumgebungen und können in vielen Belangen genau so viel leisten wie native Anwendungen.

Webbrowser stellen heute eine Abstraktionsebene dar, die es ermöglicht plattformunabhängig zu entwickeln. Browser nehmen heute immer öfter die Funktion ein, die früher zum Beispiel von virtuellen Maschinen (etwa der Java Virtual Machine) erledigt wurde.

Die konzeptionellen Vorteile von Web-Anwendungen sind offensichtlich: Das Auffinden der Software ist gleichbedeutend mit dem Merken einer Internetadresse. Das Herunterladen der Software wird vor dem Nutzer versteckt (sichtbar ist nur das Aufrufen einer Webseite), eine Installation entfällt. Software-Updates werden ebenfalls vor dem Nutzer verborgen (bei jedem Aufruf der Applikation wird jeweils die neuste Programmversion geladen).

Natürlich gibt es auch Nachteile: Web-Applikationen benötigen meistens eine dauerhafte und schnelle Internetverbindung. Außerdem ist die Leistungsfähigkeit gegenüber nativen Anwendungen an einigen Stellen noch eingeschränkt (Ahead-of-time-Compile vs. INterpretiert).


\label{chap:einleitung}
%
\chapter{Bewertung bestehender CRS-Lösungen}
\label{chap:bewertung}
Zwei bestehende CRS aus dem akademischen Bereich können nun in Bezug auf die ermittelten Anforderungen bewertet und miteinander verglichen werden: Einerseits die bisher an der HAW Hamburg eingesetzte Lösung StuReSy und zum Vergleich eine populäre, professionellere Lösung namens Pingo.

\section{StuReSy}
\label{chap:sturesy}
\ac{sturesy} ist der Name einer Software, die im Rahmen der Bachelorarbeit von Wolf Posdorfer im Jahr 20012 an der Universität Hamburg entstanden ist. Der Name StuReSy ist ein Akronym für „Student Response System“.

StuReSy besteht aus zwei Komponenten:
\begin{itemize}
    \item Server-Komponente: In PHP geschrieben, agiert gleichzeitig auch als Client-Komponente für die Abstimmungs-Teilenehmer. Inkludiert eine relationale SQL-Datenbank.
    \item Admin-Komponente: Um Fragen zu erstellen und zu bearbeiten wird ein Client als Java-Anwendung benötigt.
\end{itemize}

StuReSy wurde erfolgreich und viele Jahre an der Universität Hamburg und HAW Hamburg eingesetzt. Die Qualität und der Umfang der Software sind für eine Bachelorarbeit beeindruckend.

Dennoch verfügt StuReSy über einige Nachteile und Probleme:
\begin{itemize}
    \item Software-Download und JVM notwendig: Um StuReSy administrativ einsetzen zu können, muss eine Java-Software heruntergeladen werden und eine JVM muss auf dem jeweiligen System vorhanden sein. Eine Administration vom Tablet oder Smartphone ist damit nur schwer möglich.
    \item Server-Komponente: Um StuReSy betreiben zu können, wird eine Server-Instanz benötigt. Diese muss von der jeweiligen Institution oder einem Dozenten aufgesetzt und gewartet werden.
    \item Mangelnde Formatierungsmöglichkeiten für Software-Quelltext: In der Praxis wird StuReSy vor allem in Informatik-Veranstaltungen eingesetzt. Dort werden oft Fragen zu Quelltexten gestellt. Die Darstellung dieser Quelltexte ist schwierig: Zentrierte Text-Ausrichtung .... sorgen für unübersichtliche Darstellung.
\end{itemize}


\newpage
\section{Pingo}
\label{chap:pingo}
Pingo ist eine Software-Lösung, die bereits seit dem Jahr 2011 an der Universität Paderborn entwickelt wird. Der Name ist ebenfalls ein Akronym und steht für „\textbf{P}eer \textbf{In}struction for Very Large \textbf{G}r\textbf{o}ups“. Im Gegensatz zu StuReSy ist Pingo bereits weiter verbreitet und wird an vielen deutschen Hochschulen eingesetzt – im September 2018 gab es 22.000 angemeldete Nutzer (vgl. \cite{web:pingo_zukunft}). Dahinter stand außerdem ein ganzes Team akademischer Mitarbeiter (vgl. \cite{web:pingo_team}). Seit 2019 wird Pingo von der universitätsnahen Coactum GmbH betrieben und weiterentwickelt (vgl. \cite{web:pingo_coactum}). Das Projekt ist damit deutlich professioneller ausgerichtet als StuReSy.

Im Gegensatz zu StuReSy ist Pingo eine reine Web-Applikation, die öffentlich unter \url{http://trypingo.com/} auffindbar ist und kostenlos genutzt werden kann. Sowohl Administratoren als auch Teilnehmer können alle Arbeiten im Browser erledigen, ein Software-Download ist nicht notwendig. Für die administrative Nutzung muss jedoch ein Benutzerkonto erstellt werden. Einzelne Sitzungen werden durch numerische IDs im Namensraum eines Pingo-Servers identifiziert.


Eine Direktverbindung zwischen Umfrage-Teilnehmern wird nicht unterstützt, stattdessen wird eine Pingo-Server vorausgesetzt. Der Betreiber stellt einen öffentlichen Pingo-Server kostenlos zur Verfügung. Pingo steht aber unter einer Open-Source-Lizenz, so dass Nutzer auch eine eigene Instanz betreiben können. Pingo ist in der Programmiersprache Ruby und mithilfe des Web-Frameworks „Ruby on Rails“ implementiert (vgl. \cite{web:pingo_github}). Entsprechend dazu muss ein potenzieller Server auch über einen Ruby-Interpreter und über eine NoSQL-Datenbank verfügen, um Pingo ausführen zu können.

In ihren Kernfunktionen sind sich Pingo und StuReSy sehr ähnlich. Trotzdem fehlt eine kritische Funktionen für den Einsatz in der Programmierlehre:

Fragen innerhalb der Pingo-Plattform können überhaupt nicht formatiert werden. Damit können selbst simple Formatierungen wie Fettschreibungen, Unterstreichungen oder Zeilenumbrüche nicht verwendet werden (erkennbar in Abbildung \ref{abb:pingo_frage}). Dementsprechend ist auch die übersichtliche Darstellung von Quelltext unmöglich und Pingo für den Einsatz in der Programmierlehre ungeeignet.


\begin{figure}[H]
    \includegraphics[width=12cm]{chapter/bewertung/bilder/pingo_editor.png}
    \centering
    \caption[Fragen-Editor in Pingo ohne Formatierungsmöglichkeiten]{Der Fragen-Editor von Pingo verfügt über keine Formatierungsmöglichkeiten.}
    \label{abb:pingo_editor}
\end{figure}


\begin{figure}[H]
    \includegraphics[width=12cm]{chapter/bewertung/bilder/pingo_problem1.png}
    \centering
    \caption[Darstellung von Quelltexten in Pingo]{Eine ordentliche Darstellung von Quelltexten ohne Textformatierungen ist nicht möglich.}
    \label{abb:pingo_frage}
\end{figure}


\newpage
\section{Tabellarische Gegenüberstellung}
\label{chap:tabelle}
 Nach dem Aufstellen der Anforderungen und der Betrachtung beider CRS lassen sich diese Ergebnisse wie folgt tabellarisch gegenüberstellen: 
 \begin{table}[ht]
     \centering
     
     \label{tab:vergleich}
     \begin{tabular}{|l|c|c|c|}
     \hline
      & \textbf{StuReSy} & \textbf{Pingo} & \textbf{Gewünscht}  \\
      \hline
      komplett webbasiert & x & \checkmark & \checkmark \\
      kommt ohne Server aus & x & x & \checkmark \\
      Code-Formatierungsoptionen & \checkmark & x & \checkmark \\
      Code-Ausführung & x & x & \checkmark \\
      \hline
     \end{tabular}
     \caption{Tabellarischer Vergleich verschiedener CRS-Systeme.}
 \end{table}
%
\chapter{Kern-Anforderungen an ein modernes CRS für die Programmierlehre}
\label{chap:anforderungen}
Nach der Analyse zweier bestehender Lösungen sollen nun Kern-Anforderungen abgeleitet werden, die ein modernes CRS für den Einsatz in der Programmierlehre mitbringen sollte:

\section{Vollständige Umsetzung als Web-Applikation}
\label{chap:webbasiert}
Eine modernes CRS im akademischen Einsatz sollte vollständig webbasiert daherkommen. Die Zugänglichkeit einer Web-Applikation, die vollständig im Browser und ohne Downloads verwendet werden kann, ist gerade für den Einsatz in der Universität wichtig. Der Download oder die Installation zusätzlicher Software stellt eine unnötige Hürde für den Einsatz der Software dar. Während andere CRS durch spezielle Anforderungen (zum Beispiel die Unterstützung von Hardware-Clickern) dazu gezwungen sind, als native Anwendung zu laufen, gibt es im vorliegenden Fall keine solchen Gründe. Die Umsetzung als Web-Applikation ermöglicht außerdem die Nutzung auf verschiedenen Geräten, unabhängig von der technischen Plattform.

\section{Peer-to-Peer-Verbindungen zwischen den Nutzern}
\label{chap:anforderung_p2p}
Um die Langlebigkeit eines CRS zu erhöhen sollte der Wartungsaufwand einer solchen Software so gering wie möglich ausfallen. Außerdem sollte die Hürde zum Einsatz gerade gegenüber den Dozenten möglichst weit gesenkt werden. Der Betrieb eines eigenen, dedizierten Servers mit individuellen Anforderungen (z.B. vorhandene Interpreter oder Datenbanksysteme) widerspricht dieser Philosophie. Daher sollte ein modernes CRS ohne dedizierten Anwendungs-Server funktionieren, und stattdessen auf Direktverbindungen unter den Teilnehmern setzen.

\section{Formatierungsmöglichkeiten für Quelltext}
\label{chap:codeformatierung}
Um die Darstellung von Quelltext-Ausschnitten zu verbessern, sollte ein CRS für die Programmierlehre gut mit Code-Beispielen umgehen können. Dazu gehört, dass Code möglichst übersichtlich dargestellt werden kann. Das beinhaltet die Unterstützung von simplen Formatierungsmöglichkeiten (zum Beispiel Absätze und Fettschreibung), von Syntax-Highlighting für längere Abschnitte sowie die Unterstützung von Monospace-Formatierung, um technische Begriffe in Fließtext ordentlich abheben zu können.

\section{Code-Ausführung im Browser}
\label{chap:codeausfuehrung}
Um ein CRS noch spezifischer auf das Einsatzgebiet der Programmierlehre zuschneiden zu können, soll außerdem evaluiert werden, ob sich die Abhängigkeit von weiteren Programmen wie etwa Entwicklungsumgebungen reduzieren lässt, indem die Ausführung von Quelltexten direkt im CRS ermöglicht wird.

Um die Benutzung eines CRS zu erleichtern, sollte der Dozent keine weiteren Anwendungen neben dem CRS benötigen. Ein Parallelbetrieb von CRS und Entwicklungsumgebung, um zwischen Code-Ausführung und Fragestellung hin- und herzuschalten, ist unübersichtlich und verhindert den administrativen Einsatz eines CRS auf einem fremden Computer. Die Ausführung von Quelltext im Browser könnte diesen Nachteil beheben.

\newpage
\section{Tabellarische Gegenüberstellung}
\label{chap:tabelle}
 Nach Betrachtung beider CRS und Ableitung der Anforderungen lassen sich diese Ergebnisse wie folgt tabellarisch zusammenfassen: 
 \begin{table}[ht]
     \centering
     
     \label{tab:vergleich}
     \begin{tabular}{|l|c|c|c|}
     \hline
      & \textbf{StuReSy} & \textbf{Pingo} & \textbf{Gewünscht}  \\
      \hline
      kompl. webbasiert & x & \checkmark & \checkmark \\
      kommt ohne Server aus & x & x & \checkmark \\
      Code-Formatierungsoptionen & \checkmark & x & \checkmark \\
      Code-Ausführung & x & x & \checkmark \\
      \hline
     \end{tabular}
     \caption{Tabellarischer Vergleich verschiedener CRS-Systeme.}
 \end{table}

%
\chapter{Entwurf eines modernen CRS für die Programmierlehre}
\label{chap:entwurf}
Im Rahmen dieser Arbeit wird ein CRS entworfen und als Protptyp implementiert, welches die besagten Kern-Anforderungen erfüllt. Die entstehende Software heißt Weclare (ein Akronym für „\textbf{We}b \textbf{Cla}ssroom \textbf{Re}sponse (System)“). Der gesamte Quelltext zu dem Projekt befindet sich in einem öffentlichen GitHub-Repository\footnote{\url{https://github.com/pReya/weclare}} und eine öffentlich zugängliche Version der Software kann unter \url{https://weclare.de} aufgerufen werden. Insgesamt hat die entstandene Implementierung einen Umfang von rund 6.000 Zeilen Code (vergleichbar mit dem Vorbild StuReSy). Die Software wird als freie Software unter einer GPL-3.0-Lizenz veröffentlicht (vgl. \cite{web:github_weclare_license}).

Eine weitere, beiläufige Anforderung an das neue System: Zur Evaluierung soll Weclare auch mit bestehenden Fragesätzen aus StuReSy getestet werden. Aus diesem Grund wird außerdem ein Werkzeug entwickelt, welches Fragensätze vom XML-basierten StuReSy-Format in das JSON-basierte Weclare-Format konvertiert. Dieser Konverter wird ebenfalls in JavaScript geschrieben und ist als Kommandozeilen-Werkzeug entworfen. Es benötigt daher die Node.js-Laufzeitumgebung, um JavaScript außerhalb des Browsers ausführen zu können. Auch dieses Hilfsprogramm kann in einem öffentlichen GitHub-Repository\footnote{\url{https://github.com/pReya/weclare-sturesy-converter}} gefunden werden. 

Die Umsetzung der wichtigsten Kern-Aspekte von Weclare wird auf den kommenden Seiten exemplarisch beschrieben.

\newpage
\section{Implementierung als Single-Page-Application mit dem React-Framework}
\label{chap:react_einfuehrung}
Um eine webbasierte Anwendung zu erstellen, die ohne einen zentralen Server auskommt, muss die Software komplett clientseitig in einem Browser ausgeführt werden und damit auch zwangsläufig in JavaScript implementiert werden. Ein klassisches Backend, also eine Datenschicht (üblicherweise handelt es sich bei den meisten Web-Applikationen mindestens um eine Zwei-Schichten-Architektur) existiert nicht, beziehungsweise ist in die Präsentationsschicht integriert. Das bedingt die Kategorisierung der entstehenden Anwendung als „Fat Client“.

Durch die vollständig clientseitige Ausrichtung bietet es sich an, die Software als sogenannte Single-Page-Application (SPA) zu implementieren. Herkömmliche Webseiten, sogenannte Multi-Page-Applications (MPA) laden während ihrer Lebenszeit mehrfach neue Seiten vom Server, zum Beispiel jedes Mal, wenn innerhalb der Anwendung eine neue Ansicht dargestellt werden soll. Eine SPA dagegen lädt nur eine einzige Webseite sowie das zugehörige JavaScript-Programm und verändert diese eine Seite dann im weiteren Verlauf dynamisch. Neue Daten werden bei Bedarf asynchron (ohne Blockieren der Seite) nachgeladen, es wird jedoch keine komplette Seite geladen. Damit wird auch eine maximale Autarkie gegenüber dem Webserver erreicht, da dieser nur mit wenigen Requests (im einfachsten Fall ein Request pro Nutzer) umgehen muss. Die Anforderungen an den Webserver, auf dem die Anwendung bereitgestellt wird, sind sehr gering – er muss lediglich statische Dateien (HTML, JavaScript, Bilder, Schriften, etc.) bereitstellen.

Als Framework für die Implementierung einer solchen SPA wird das React-Framework\cite{web:react} ausgewählt. Das Open-Source-Projekt existiert seit 2013 und wird von Facebook finanziert und gefördert. Es gehört zu den populärsten Frameworks zum Erstellen von Benutzeroberflächen und Web-Applikationen\cite{web:stackoverflow_umfrage} und ist dem Autor der Arbeit bereits vertraut.

Um einige Implementationsdetails nachzuvollziehen, erfolgt an dieser Stelle eine  Einführung in einige Grundkonzepte von React.

\subsection{Komponenten und State}
Die elementaren Bausteine einer React-Anwendung sind \texttt{Komponenten}. Die React-Dokumentation beschreibt die Aufgabe von Komponenten wie folgt\cite{web:react}:
\begin{quotation}
„Build encapsulated components that manage their own state, then compose them to make complex UIs.“
\end{quotation}

Ein Komponente ist eine autarke und wiederverwendbare Einheit und kapselt meistens sowohl Struktur, Aussehen als auch Logik. Komponenten werden häufig als Klasse implementiert (können aber in einfachen Fällen auch als Funktion implementiert werden) und erben von der Klasse \texttt{React.Component}. Valide Komponenten müssen über eine \texttt{render()}-Funktion verfügen, die HTML oder andere React-Komponenten zurückliefert. Eine sehr simple, statische Komponente könnte so aussehen:

\begin{minipage}{\linewidth}
\begin{lstlisting}[caption={Einfache React-Komponente ohne JSX-Syntax.}]
import React from "react";

class Greeting extends React.Component {
  render() {
    return React.createElement("div", null, "Hello there!");
  }
}
\end{lstlisting}
\end{minipage}

Da die Syntax des Aufrufs \texttt{React.createElement()} nicht so schön zu lesen ist, wie die Notation eines XML-Tags (\texttt{<div>...</div>}) wird im React-Umfeld üblicherweise eine Syntax-Erweiterung namens JSX (JavaScript Syntax Extension) verwendet, um React-Komponenten einfacher beschreiben zu können. JSX wird mit einem Compiler während des Build-Prozesses in herkömmliches JavaScript umgewandelt. Äquivalent zum letzten Beispiel wäre daher die folgende Variante unter Einbeziehung von JSX-Syntax:

\begin{minipage}{\linewidth}
\begin{lstlisting}[caption={Einfache React-Komponente mit JSX-Syntax.}]
import React from 'react';

class Greeting extends React.Component {
    render() {
        return <div>Hello there!</div>;
    }
}
\end{lstlisting}
\end{minipage}

Um Komponenten dynamisch zu machen, können über sogenannte \texttt{Properties} Daten an Komponenten übergeben werden. Auf diese kann mit dem \texttt{props}-Objekt zugegriffen werden:

\begin{minipage}{\linewidth}
\begin{lstlisting}[caption={Komponenten erhalten Daten über ihre Properties.}]
import React from "react";

class Greeting extends React.Component {
  render() {
    return <div>Hello, {this.props.name}!</div>;
  }
}
\end{lstlisting}
\end{minipage}

Properties werden wie andere HTML-Attribute auch einfach hinter den Namen einer Komponente innerhalb des zugehörigen Tags in die Instanziierung einer Komponente integriert:

\begin{minipage}{\linewidth}
\begin{lstlisting}[caption={Properties werden wie normale HTML-Attribute verwendet.}]
import React from "react";

class GreetAllFriends extends React.Component {
  render() {
    return (
      <div>
        <Greeting name="Michael" />
        <Greeting name="Karla" />
      </div>
    );
  }
}
\end{lstlisting}
\end{minipage}

\texttt{Properties} werden also verwendet, um Daten in Komponenten hinein zu reichen. React kümmert sich dann automatisch um das Aktualisieren der Ansicht, sobald sich eine Property ändert (dieses „reaktive“ Prinzip ist auch der Namensgeber für das Framework). Dabei verwendet React sehr schnelle und intelligente Algorithmen, um immer nur diejenigen Elemente einer Seite zu aktualisieren, die sich auch tatsächlich geändert haben. Dadurch können Single-Page-Applications schneller agieren als Multi-Page-Applications.

Properties können nicht modifiziert werden – sie sind wie auch die meisten anderen Objekte innerhalb von React „immutable“. Wenn Daten innerhalb einer Komponente modifiziert werden sollen, gehören sie in das interne, modifizierbare \texttt{state}-Objekt dieser Komponente:

\begin{minipage}{\linewidth}
\begin{lstlisting}[caption={Jede Komponente kann über einen modifizierbaren state verfügen.}]
import React from "react";

class BusyOrNot extends React.Component {
  state = {
    busy: false
  };

  toggleBusy() {
    this.setState(prevState => ({
      busy: !prevState.busy
    }));
  }

  render() {
    return (
      <div>
        <div>This user is {busy ? "busy" : "not busy"}!</div>
        <button type="button" onClick={this.toggleBusy}>
          Change Busy State
        </button>
      </div>
    );
  }
}
\end{lstlisting}
\end{minipage}

Bedingt durch die statische Natur der Properties, forciert React einen unidirektionalen Datenfluss. Daten können nur von „oben nach unten“ (in Bezug auf die Baumstruktur im Document-Object-Model einer Seite) durch eine Anwendung fließen. Möchten zwei Komponenten an unterschiedlichen Stellen auf die gleichen Daten zugreifen, dann sollten diese oberhalb der beiden Komponenten in einer gemeinsamen Eltern-Komponente gehalten werden.\newline

\begin{minipage}{\linewidth}
\begin{lstlisting}[caption={„Lifting state up“: Mehrere Komponenten greifen auf die gleichen Daten zu.}]
import React from "react";

function Son(props) {
  return <p>I am the son of {props.parent}</p>
}

function Daughter(props) {
  return <p>I am the daughter of {props.parent}</p>
}

class Parent extends React.Component {
  state = {
    parentName: "Peter Parent"
  };

  render() {
    return (
      <div>
        <Son parent={this.state.parentName} />
        <Daughter parent={this.state.parentName} />
      </div>
    );
  }
}
\end{lstlisting}
\end{minipage}

Da dieses Muster bei großen Anwendungen aber schnell zu sehr aufwendigem „Durchstecken“ von Properties (sogenanntes „Prop Drilling“) durch mehrere Komponenten-Ebenen führt, gibt es eine populäre Erweiterung für React zur Verwaltung eines einzigen, globalen States in der gesamten Anwendung: das Flux-Muster\cite{web:flux_doku}.


\subsection{State Management mit Redux}
\label{chap:redux_state_management}
Das Flux-Entwurfsmuster ist ebenfalls eine Entwicklung von Facebook. Prinzipiell handelt es sich um ein abstraktes Entwurfsmuster, das in vielen Sprachen angewendet werden kann. Etabliert hat es sich jedoch gerade in Kombination mit React-Anwendungen. Die bekannteste Implementation, die auch in dieser Arbeit verwendet wird, hört auf den Namen Redux\cite{web:redux}.

Im Flux-Muster geht es darum, eine zentrale Zustandsverwaltung für eine Anwendung einzurichten, eine sogenannte „Single Source of Truth“. Dieser zentrale Ort wird als \texttt{Store} bezeichnet. Ein Store beinhaltet typischerweise solche Daten, die für die gesamte Anwendung relevant sind. Parallel dazu kann es aber weiterhin Komponenten geben, die einen eigenen, lokalen State verwalten, wenn dieser nicht für die gesamte Anwendung relevant ist.

\begin{figure}[H]
    \includegraphics[width=12cm]{chapter/entwurf/bilder/BA_redux.png}
    \centering
    \caption{Der Redux-Store verwaltet den globalen Zustand einer React-Anwendung und ist die „Single Source of Truth“. Quelle: https://css-tricks.com/learning-react-redux/ (aufgerufen am 24.4.19)}
    \label{abb:redux_flow}
\end{figure}


Einzelne React-Komponenten können mit einem Store durch einen Publish/Subscribe-Mechanismus verbunden werden. In Redux wird diese Verbindung durch die \texttt{connect}-Funktion zur Verfügung gestellt. Die gewünschten Daten aus dem Store können in einer \texttt{mapStateToProps}-Funktion angegeben werden und stehen der Komponente dann als Properties zur Verfügung. Ändern sich die Daten im Store, dann wird die verbundene Komponente sofort benachrichtigt und bei einer Änderung der Properties auch neu gerendert (reaktives Prinzip). Um eine möglichst lose Kopplung zwischen den Komponenten zu realisieren, wird deswegen empfohlen, die Verbindung zu einem Store in einer (nicht sichtbaren) Container-Komponente zu realisieren. Im folgenden Beispiel wird die eigentliche, sichtbare \texttt{Header}-Komponente mit einem unsichtbaren Container versehen, der die notwendigen Daten aus dem Store innerhalb der Komponente unter der \texttt{status}-Property verfügbar macht.

\begin{minipage}{\linewidth}
\begin{lstlisting}[caption={Über den connect-Aufruf beim Exportieren der Komponente wird sie mit dem Store verbunden. (aus:  src/client/components/ClientHeaderContainer.js)}]
import { connect } from "react-redux";
import Header from "./Header";

const mapStateToProps = state => ({
  status: state.connection.status
});

export default connect(mapStateToProps)(Header);
\end{lstlisting}
\end{minipage}

Änderungen in einem Store müssen mithilfe von \texttt{Actions} realisiert werden. Bei einer Action handelt es sich lediglich um ein Objekt, welches die Art der Änderung in einem Store beschreibt. Um das wiederholte Schreiben solcher aufwändigen Objekt-Literale zu erleichtern, werden die Actions üblicherweise von einer einfachen \texttt{ActionCreator}-Funktion erzeugt:

\begin{minipage}{\linewidth}
\begin{lstlisting}[caption={Ein Action-Objekt ist lediglich die Beschreibung einer Änderungsoperation und wird in einem ActionCreator erzeugt.}]
export function addQuestion(newQuestion) {
  return {
    type: "ADD_QUESTION",
    payload: {
      newQuestion
    }
  };
}
\end{lstlisting}
\end{minipage}

Die eigentliche Implementierung einer Änderungsoperation erfolgt in dem zugehörigen \texttt{reducer}. Die Operationen in einem reducer sind stets pure Funktionen, das heißt, sie liefern immer das gleiche Ergebnis bei gleichen Eingabe-Parametern und sie haben keine Seiteneffekte. Die Parameter des reducers sind immer der aktuelle State des Stores und die eingehende Action, der Rückgabewert ist der neue State des Stores. Ein einfacher Reducer zum Hinzufügen einer Frage zu einem Fragekatalog könnte so aussehen:

\begin{minipage}{\linewidth}
\begin{lstlisting}[caption={In einem Reducer werden die Änderungsoperationen eines Stores als pure Funktion implementiert.}]
const questionEditor = (state = [], action) => {
  switch (action.type) {
    case "ADD_QUESTION": {
      return [... state, createNewQuestion()];
    }
  }
};
\end{lstlisting}
\end{minipage}

\subsection{Konkrete Umsetzung am Beispiel des Fragen-Editors}
Um das neue CRS zu implementieren, muss zunächst eine sinnvolle Aufteilung in React-Komponenten erfolgen. Exemplarisch soll an dieser Stelle einer der Hauptbestandteile der Anwendung besprochen werden: der Fragen-Editor.

Das zweiteilige Design, bestehend aus einer Fragenliste in der Seitenspalte und einem Fragen-Inhaltsbereich, das sowohl bei StuReSy als auch bei Pingo zum Einsatz kommt, soll beibehalten werden.

Eine mögliche Komponenten-Hierarchie für den Fragen-Editor wird in Abbildung \ref{abb:komponenten_hierarchie} veranschaulicht. Die äußerste Komponente, der \texttt{QuestionEditorContainer}, ist nicht sichtbar. Es handelt sich dabei lediglich um eine Container-Komponente, welche die notwendigen Daten aus dem Store holt und dann an ihre Kinder-Komponenten mittels Properties weitergibt.

\begin{figure}[H]
    \includegraphics[width=\textwidth]{chapter/entwurf/bilder/Component_Hierarchy.png}
    \centering
    \caption{Mögliche Komponenten-Hierarchie für den Fragen-Editor.}
    \label{abb:komponenten_hierarchie}
\end{figure}

Die tatsächliche Implementierung entspricht nahezu vollständig diesem Bild (einzig die Komponente \texttt{QuestionEditorListItem} wurde aus Gründen der Drag-and-Drop-Funktionalität in der Seiteleiste nicht genau so implementiert).


\begin{figure}[H]
    \centering
    \setlength{\fboxsep}{0pt}
    \setlength{\fboxrule}{0.5pt}
    \fbox{
    \includegraphics[width=\textwidth-1pt]{chapter/entwurf/bilder/weclare_editor.png}}
    \caption{Fertige Implementierung des Fragen-Editors im React-Framework.}
    \label{abb:weclare_editor}
\end{figure}

Alle Daten, mit denen der Fragen-Editor arbeitet, werden in einem einzigen Redux-Store gehalten. Alle möglichen Änderungen finden sich in einem zugehörigen Reducer wieder, der auszugsweise wie folgt aussieht:


\begin{minipage}{\linewidth}
\begin{lstlisting}[caption={Auszug aus dem Reducer für den Fragen-Editor (aus: src/server/reducers/questions.js)}]
export const questionEditor = (state = [], action) => {
  switch (action.type) {
    case ADD_QUESTION: {...}
    case EDIT_QUESTION_TEXT: {...}
    case EDIT_QUESTION_CODE: {...}
    case EDIT_QUESTION_MODE: {...}
    case EDIT_QUESTION_TYPE: {...}
    case DELETE_QUESTION: {...}
    case DELETE_ANSWER: {...}
    case ADD_ANSWER: {...}
    case EDIT_ANSWER_TEXT: {...}
    case SET_CORRECT_SINGLE_ANSWER: {...}
    case SET_CORRECT_MULTI_ANSWER: {...}
    case LOAD_QUESTIONS: {...}
    case SORT_QUESTION: {...}
    case SORT_ANSWER: {...}
    default: {
      return state;
    }
  }
};
\end{lstlisting}
\end{minipage}

\newpage
\section{Peer-to-Peer-WebRTC-Verbindungen mit PeerJS}
\label{chap:p2p}
Um Unabhängigkeit von einem dedizierten, zentralen Anwendungsserver zu erlangen, der einerseits Wartungsaufwand bedeutet, und außerdem einen „Single Point of Failure“ darstellt, werden Verbindungen direkt zwischen den Teilnehmern aufgebaut. Jeder Weclare-Nutzer kann zum Start einer Sitzung, also zur Laufzeit, entscheiden, welche Rolle er in der aktuellen Sitzung einnehmen will (Server oder Client). Der Rechner des Dozenten agiert meist als Server, die Rechner der Studenten sind Clients und somit ergibt sich eine klassische, zentralisierte Architektur des verteilten Systems in Form einer Stern-Topologie. Unabhängig von der Wahl wird jedoch stets der gleiche Code vom Server geladen. Das Programm ist in dieser Hinsicht also omnipotent.

\begin{figure}[H]
    \centering
    \setlength{\fboxsep}{0pt}
    \setlength{\fboxrule}{0.5pt}
    \fbox{\includegraphics[width=\textwidth-1pt]{chapter/entwurf/bilder/weclare_start.png}}
    \caption{Am Start einer Sitzung kann der Nutzer entscheiden, ob er als Client oder als Server teilnehmen möchte. Unabhängig von der Wahl wird das gleiche Programm geladen.}
    \label{abb:weclare_start}
\end{figure}

Mit diesem Verhalten erfüllt die Software die Definition einer Peer-to-Peer Architektur, zum Beispiel gemäß Tannenbaum (S. 62) (TODO):

\begin{quotation}
„Aus einem übergeordneten Blickwinkel heraus sind die Prozesse, die ein Peer-to-Peer-System bilden, alle gleich. Die Funktionen, die ausgeführt werden müssen, werden also von jedem Prozess des verteilten Systems dargestellt. Folglich ist die meiste Interaktion zwischen Prozessen symmetrisch: Jeder Prozess agiert gleichzeitig als Client und als Server [...].“
\end{quotation}

Jedoch trifft dies eigentlich nur auf den Zeitraum vor dem Start einer Sitzung zu. Sobald eine Sitzung begonnen hat, handelt es sich um eine klassische Client-Server-Struktur. Die Bezeichnung als Peer-to-Peer-Architektur entspricht daher nicht der gängigen Vorstellung eines vollvermaschten Peer-to-Peer-Netzwerks, bei dem jeder Teilnehmer mit jedem anderen Teilnehmer verbunden ist.

Die Tatsache, dass jeder Teilnehmer entscheiden kann, ob er als Server oder Client teilnehmen will, bedeutet auch, dass Aspekte wie Network Address Translation (NAT) und Firewalls beim Austausch der IP-Adressen (Signalling) berücksichtigt werden müssen.

Der Zugriff auf die Schnittstellen des Betriebssystems (wie etwa das Netzwerk) durch Browser-Skripte ist aus Sicherheitsgründen stark eingeschränkt. Viele bestehende CRS verwenden das WebSocket-Protokoll zur Kommunikation zwischen Client und Server. Ein Browser kann jedoch nur ausgehende Verbindungen über WebSockets initiieren und nicht empfangen. Eine Peer-to-Peer-Architektur mit WebSockets ist deswegen nicht realisierbar.

Die einzige Möglichkeit, eine omnidirektionale Verbindung zwischen zwei Browsern zu realisieren, ist der relativ neue und offene WebRTC-Standard (Web Real-Time Communication)\cite{web:webrtc}. WebRTC wird hauptsächlich für Multimedia-Echtzeit-Anwendungen eingesetzt und seit 2017 von allen großen Browsern (Google Chrome, Mozilla Firefox, Opera, Apple Safari und Microsoft Edge) unterstützt. Viele Video- und Audiotelefonie-Lösungen (zum Beispiel Skype oder Discord) basieren inzwischen auf dem WebRTC-Protokoll. Neben Audio- und Videoinhalten können aber auch beliebige andere Daten über sogenannte \texttt{RTCDataChannel} übertragen werden. WebRTC beinhaltet aber keine Anweisungen für den Austausch der IP-Adressen zwischen beteiligten Parteien. Dieser Teil, das sogenannte Signalling, ist nicht Teil des Standards und muss selbst implementiert werden. Der Aufwand dafür ist relativ groß, weil viele verschiedene Netzwerk-Situationen berücksichtigt werden müssen. Deswegen wird beim Weclare-Prototypen eine OpenSource-Bibliothek namens PeerJS\cite{web:peerjs} verwendet, die WebRTC in eine sehr einfache API kapselt und ein etabliertes Signalling-Verfahren beisteuert.

Die mehrfach formulierte Anforderung, keinen dedizierten Server zu benötigen, kann aufgrund des notwendigen Signallings also nicht vollständig erfüllt werden und muss an dieser Stelle präzisiert werden: Ein Server wird lediglich zum Austausch der IP-Adressen der Teilnehmer benötigt. Nach dem Austausch ist kein Server mehr notwendig. Anwendungsdaten werden nie über einen zentralen Server, sondern immer nur zwischen den einzelnen Teilnehmern verschickt (Die TURN-Funktionalität von WebRTC wird nicht verwendet). Der Signalling-Server muss nicht vom Weclare-Anwender betrieben werden – ein öffentlicher Server kann verwendet werden.

Die PeerJS-Library ermöglicht den Aufbau einer Datenverbindung zwischen zwei Browsern mit sehr simplen Aufrufen: Zunächst müssen Client und Server ein Peer-Objekt erzeugen. Als Parameter kann an dieser Stelle eine (auf dem Signalling-Server noch nicht verwendete) alphanumerische ID übergeben werden, unter welcher der zugehörige Peer beim Signalling-Server bekannt gemacht wird. Optional kann außerdem ein eigener Signalling-Server angegeben werden. Die Standard-Einstellung verwendet den kostenlosen und öffentlichen Signalling-Server, der von den PeerJS-Autoren betrieben wird.

Anschließend können an dem neuen Peer-Objekt diverse Callback-Methoden registriert werden, die den weiteren Gebrauch regeln. So kann zum Beispiel der Server seine neu erstellte ID kundtun und auf eingehende Verbindungen und Daten reagieren:

\begin{minipage}{\linewidth}
\begin{lstlisting}[caption={Verbindungsaufbau mit der PeerJS-Bibliothek auf der Server-Seite. (aus: src/server/actions/server.js)}]
import Peer from "peerjs";

const peer = new Peer("server-id");

peer.on("open", id => {
  console.log(`Successfully created peer: ${id}`);
});

peer.on("connection", conn => {
  console.log(`New client connected: ${conn.peer}`);
  conn.on("data", data => {
    switch (data.type) {
      case "answer":
        // Do something
        break;
      default:
      // Noop
    }
  });
});
\end{lstlisting}
\end{minipage}

Auf der Gegenseite, beim Client kann die Verbindung einfach über die \texttt{connect()}-Methode aufgebaut werden, die als Parameter die ID des gewünschten Peers erhält. Anschließend kann über das zurückgelieferte \texttt{Connection}-Objekt eine Nachricht verschickt werden:

\begin{minipage}{\linewidth}
\begin{lstlisting}[caption={Verbindungsaufbau mit der PeerJS-Bibliothek auf der Client-Seite. (aus: src/client/actions/client.js)}]
import Peer from "peerjs";

const peer = new Peer();
const connection = peer.connect("server-id");
connection.send("Hello world!");
\end{lstlisting}
\end{minipage}

Um dem Flux-Muster (siehe \ref{chap:redux_state_management}) treu zu bleiben, wird die gesamte Netzwerk-Kommunikation innerhalb von ActionCreator-Funktionen implementiert. Da Netzwerk-Aufrufe keine puren Funktionen sind und asynchron erfolgen müssen, können sie nicht in einen reducer integriert werden. Um solche asynchronen Actions zu realisieren, wird Redux um eine sehr simple Middleware namens \texttt{redux-thunk}\cite{web:redux_thunk} erweitert. Diese Middleware erlaubt es, asynchrone Aufrufe in ActionCreator-Methoden unterzubringen. Da viele dieser Netzwerkaufrufe eigentlich keine Änderungen im Store bewirken, wird der ActionCreator streng genommen missbraucht, denn am Ende wird entgegen seinem Zweck keine Action mehr zurückgegeben.

\begin{minipage}{\linewidth}
\begin{lstlisting}[caption={ActionCreator zum Versenden von Antworten vom Client zum Server. (aus: src/client/actions/client.js)}]
export function sendAnswers(answerIdxArray) {
  return (dispatch, getState) => {
    const {
      client: { connection = null, currentQuestion = null }
    } = getState();

    if (
      connection &&
      currentQuestion &&
      typeof answerIdxArray !== "undefined"
    ) {
      const msg = {
        type: "answer",
        payload: {
          questionIdx: currentQuestion.questionIdx,
          answerIdxArray,
          userId: connection.provider.id
        }
      };
      connection.send(msg);
    }
  };
}
\end{lstlisting}
\end{minipage}

\newpage
\section{Text- und Code-Formatierung mit Quill und CodeMirror}
\label{chap:formatierung}
Quelltexte können in zwei verschiedenen Kontexten in Fragestellungen auftauchen: Zum Einen gibt es unvollständige, kurze Quelltext-Fragmente, die in einen Fließtext eingebunden werden sollen, wie in diesem Beispiel:

\begin{quote}
„Wie viele String-Parameter hat die folgende Java-Methode?\newline
public String m(int i, int s, boolean b) \{ ... \}“
\end{quote}

Dabei handelt es sich um ein syntaktisch unvollständiges und nicht ausführbares Java-Fragment. Um die Lesbarkeit dieses Fragments zu erhöhen, sollte es dennoch optisch deutlich vom Fließtext unterscheidbar sein. Es bietet sich an, den Code in einer Monospace-Schriftart zu formatieren und (wenn möglich) eine syntaktische Einfärbung (Syntax Highlighting) anzuwenden.

Da die Entwicklung einer eigenen Editor-Komponente komplex ist, und es in diesem Bereich bereits eine große Auswahl an Bibliotheken gibt, wird auf eine bestehende Implementierung zurückgegriffen. Dabei müssen folgende Anforderungen von einer solchen Bibliothek erfüllt werden:

\begin{itemize}
    \item \textbf{Integration in React:}  Muss sich leicht in das deklarative React-Framework einbinden lassen.
    \item \textbf{WYSIWYG (What You See Is What You Get):} Der Editor muss stets eine Vorschau aller Formatierungen darstellen, und es soll nicht zwischen einem Markup- und einem Vorschau-Modus gewechselt werden müssen. Formatierungen sollen über Buttons eingefügt werden können.
    \item \textbf{Text-Formatierungen:} Der Editor muss über simple Text-Formatierungen verfügen (zum Beispiel Fettschrift und Kursivierung).
    \item \textbf{Monospace-Schriftart:} Eine Möglichkeit zum Verwenden einer Monospace-Schriftart muss vorhanden sein.
    \item \textbf{Syntax-Highlighting:} Syntaktische, farbliche Hervorhebungen von Code-Fragmenten müssen unterstützt werden (üblicherweise durch ein Plugin/Erweiterung mit einer zusätzlichen Bibliothek wie Highlight.js\footnote{Offizielle Webseite: \url{https://highlightjs.org/}}).
\end{itemize}

Nach dem Erwägen und Ausprobieren diverser Bibliotheken (z.B. Draft.js\footnote{Offizielle Webseite: \url{https://draftjs.org/}}, Slate\footnote{Offizielle Webseite: \url{https://www.slatejs.org/}}, Prosemirror\footnote{Offizielle Webseite: \url{https://prosemirror.net/}}) fiel die Wahl auf den Quill-Editor\footnote{Offizielle Webseite: \url{https://quilljs.com/}}, der alle genannten Anforderungen erfüllt. Quill ist nicht primär für den Einsatz mit React vorgesehen. Daher wird der Editor in Form der Wrapper-Bibliothek \texttt{react-quill}-Bibliothek\footnote{Offizielle Webseite: \url{https://github.com/zenoamaro/react-quill}} verwendet, die Quill in React-Komponenten bündelt.

\begin{figure}[H]
    \centering
    \setlength{\fboxsep}{0pt}
    \setlength{\fboxrule}{0.5pt}
    \fbox{\includegraphics[width=\textwidth-1pt]{chapter/entwurf/bilder/sturesy_editor.png}}
    \caption[StuReSy-Fragen-Editor]{Bearbeitung einer Fragestellung im StuReSy-Fragen-Editor.}
    \label{abb:sturesy_editor}
\end{figure}

\begin{figure}[H]
    \centering
    \setlength{\fboxsep}{0pt}
    \setlength{\fboxrule}{0.5pt}
    \fbox{\includegraphics[width=\textwidth-1pt]{chapter/entwurf/bilder/sturesy_fragment.png}}
    \caption[Darstellung eines Code-Fragments in StuReSy]{Darstellung eines Code-Fragments in StuReSy (ohne manuell vorgenommene Formatierungen).}
    \label{abb:sturesy_code_fragment}
\end{figure}


\begin{figure}[H]
    \centering
    \setlength{\fboxsep}{0pt}
    \setlength{\fboxrule}{0.5pt}
    \fbox{\includegraphics[width=\textwidth-1pt]{chapter/entwurf/bilder/weclare_quill.png}}
    \caption[Quill-Editor von Weclare]{Bearbeitung einer Fragestellung im Quill-Editor von Weclare.}
    \label{abb:weclare_quill}
\end{figure}

\begin{figure}[H]
    \centering
    \setlength{\fboxsep}{0pt}
    \setlength{\fboxrule}{0.5pt}
    \fbox{\includegraphics[width=\textwidth-1pt]{chapter/entwurf/bilder/weclare_fragment.png}}%
    \caption[Darstellung eines Code-Fragments in Weclare]{Darstellung eines Code-Fragments in Weclare (Quelltext als Code-Block ausgezeichnet).}
    \label{abb:weclare_code_fragment}
\end{figure}

Der zweite Kontext von Quelltexten in Fragestellungen ist das Einfügen von vollständigem, ausführbarem Code, in Form ganzer Java-Klassen. Derartiger Code kann in Weclare direkt im Browser ausgeführt werden. Solch ein interaktives Code-Beispiel kann allerdings nur einmal pro Frage vorhanden sein und ist auch nicht Teil des Fließtexts. Eine entsprechende Editor-Komponente zum Einfügen des Codes sollte also auch funktional darauf zugeschnitten sein: Text-Formatierungen sind nicht mehr notwendig, stattdessen rücken Funktionen wie die korrekte Einrückung von Code-Zeilen und Zeilen-Nummerierungen in den Vordergrund.

Der Quill-Editor kann nicht in einem „Code only“-Modus betrieben werden, so dass eine zweite, unabhängige Editor-Komponente für interaktive Code-Abschnitte integriert wird. Mit CodeMirror\footnote{Offizielle Webseite: \url{https://codemirror.net/}} existiert eine beliebte Bibliothek für diesen Zweck, die allerdings nicht explizit für den Einsatz im React-Framework bestimmt ist, so dass auch sie in Form einer Wrapper-Bibliothek namens \texttt{react-codemirror2}\footnote{Offizielle Webseite: \url{https://github.com/scniro/react-codemirror2}} zum Einsatz kommt.



\begin{figure}[H]
    \centering
    \setlength{\fboxsep}{0pt}
    \setlength{\fboxrule}{0.5pt}
    \fbox{\includegraphics[width=\textwidth-1pt]{chapter/entwurf/bilder/weclare_codemirror.png}}
    \caption[Darstellung eines ausführbaren Java-Quelltexts in Weclare]{Darstellung eines ausführbaren Java-Quelltexts in Weclare mittels der CodeMirror-Bibliothek.}
    \label{abb:weclare_codemirror}
\end{figure}

\newpage
\section{Code-Ausführung im Browser via DoppioJVM}
\label{chap:ausfuehrung}
Normalerweise wird Java in einem zweistufigen Prozess ausgeführt. Zunächst wird eine plattformabhängige JVM geladen. Innerhalb dieser JVM wird der Java-Compiler (javac) aufgerufen. Dieser compiliert Java-Quelltext zu Java-Bytecode. Der Java-Bytecode kann dann von der JVM ausgeführt werden. Dieser Ablauf ist in der \ref{Abbildung 4.4} unter a) dargestellt.

Um Java in einem Browser auszuführen ergeben sich konzeptionell mindestens drei verschiedene Möglichkeiten, die in \ref{Abbildung 4.4} mit den Buchstaben b-d gekennzeichnet sind.

\begin{figure}[H]
    \includegraphics[width=14cm]{chapter/entwurf/Java_JavaScript_Execution.png}
    \centering
    \caption{Blubb}
    \label{Abbildung 4.4}
\end{figure}

\begin{itemize}
    \item \textbf{b):} Java-Quelltext wird mit einem Compiler direkt in JavaScript umgewandelt: Diese Variante ist konzeptionell sehr einfach. Zwar gibt es einige Programme, welche Java-Quelltext in JavaScript übersetzen können (zum Beispiel JSweet oder GWT), jedoch sind alle diese Programme entweder in Java selbst oder in anderen Programmiersprachen verfasst. Einen solchen Compiler, der selbst in JavaScript geschrieben wurde, gibt es bisher nicht. Es wäre möglich eines dieser Programme auf einem separaten Server zu verwenden. Dies würde aber eine weitere Abhängigkeit von einem Server bedeuten, und damit der formulierten Kern-Anforderung widersprechen.
    \item \textbf{c):} Java-Quelltext wird mithilfe eines ersten Compilers in Java-Bytecode übersetzt. Anschließend wird der Java-Bytecode mit einem zweiten Compiler in JavaScript übersetzt. Auch hier ergibt sich das gleiche Problem wie in b). Zwar gibt es Programme, die den jeweiligen Teilschritt übernehmen könnten, jedoch ist keins davon in JavaScript verfasst, so dass es keine Lösung dafür im Browser gibt.
    \item \textbf{d):} Es wird eine JVM verwendet, die in JavaScript implementiert wurde. Mithilfe dieser JVM kann der javac-Compiler geladen werden, um Java-Quelltext in Java-Bytecode zu übersetzen und anschließend auszuprobieren. Mit dem Software-Projekt Doppio fand sich eine überzeugende Implementierung, die relativ einfach eingesetzt werden konnte.Der konzeptionelle Nachteil dieser Lösung liegt in der benötigten Datenmenge: Um eine JVM im Browser auszuführen wird die Java Runtime Environment benötigt, deren Größe bei etwa 60 Megabyte liegt.
\end{itemize}
%
\chapter{Ausblick: WebAssembly als Zukunftsvision für portable Software}
\label{chap:ausblick}
Der Weclare-Prototyp zeigt, dass es möglich ist, dank einer JVM-Implementierung in JavaScript einen Java-Quelltext in der Browser-Umgebung zu kompilieren und anschließend auszuführen, ohne dafür einen Compiler-Server zu verwenden. Gleichzeitig zeigen sich bei dieser Vorgehensweise auch deutliche Probleme:

\begin{itemize}
    \item \textbf{Größe der Java Runtime:} Die Java-Laufzeitumgebung ist etwa 62 Megabyte groß und muss erst vom Client heruntergeladen werden. Trotz schneller Internetverbindungen ist das eine beträchtliche Datenmenge, die auch die mobile Nutzung der Code-Ausführung einschränkt.
    \item \textbf{Performance:} Die DoppioJVM ist um ein Vielfaches langsamer als eine native JVM. Das hat mehrere Gründe, zum Beispiel die Qualität der Implementierung oder der geringere Grad der Optimierung (Ahead-Of-Time-kompilierte vs. interpretierte Sprache).
\end{itemize}

Wäre diese Arbeit zwei oder drei Jahre später entstanden, hätte es vermutlich eine ernstzunehmende Alternative zur Realisierung der Code-Ausführbarkeit gegeben: WebAssembly.

WebAssembly ist ein Standard der bereits von allen großen Browser-Herstellern unterstützt wird (Microsoft, Mozilla, Apple, Google). Es handelt sich bei WebAssembly um ein sogenanntes Compile-Target. Also eher nicht um eine Programmiersprache, die manuell von Entwicklern geschrieben wird (obwohl das möglich ist), sondern um ein Ausgabeformat eines Compilers. Viele andere, beliebte Sprachen können nach WebAssembly kompiliert werden. Gut unterstützt wird das momentan zum Beispiel von den Sprachen C. C++ und Rust. WebAssembly erfüllt damit die gleiche Aufgabe wie Java-Bytecode: Es ist ein plattformunabhängiges und kompaktes Binärformat, das in einer virtuellen Maschine ausgeführt werden kann.

Damit ermöglicht WebAssembly zukünftig ganz neue Möglichkeiten: Software-Entwickler können Web-Anwendungen in beliebigen Sprachen verfassen und sind nicht mehr an JavaScript gebunden. Außerdem können bestehende Anwendungen relativ einfach in die Browser-Umgebung portiert werden und die Leistungsfähigkeit von Web-Anwendungen gesteigert werden (WASM kann als Ahead-of-Time-kompilierte Sprache tendenziell schneller ausgeführt werden als eine interpretierte und JIT-kompilierte Sprache wie JavaScript).

Es gibt bereits Compiler, die Java-Bytecode nach WebAssembly konvertieren können (zum Beispiel TeaVM\footnote{Offizielle Webseite: \url{http://teavm.org/}}). Allerdings ist der TeaVM-Compiler selbst in Java geschrieben und dafür gedacht, vorab und einmalig ausgeführt zu werden, um eine Java-Anwendung im Browser ausführen zu können. Der Compiler selbst kann also noch nicht im Browser ausgeführt werden. Theoretisch ließe sich dieser Compiler selbst nach WebAssembly kompilieren und könnte dann im Browser funktionieren. Praktisch ist das allerdings aufgrund vieler kleiner Einschränkungen des (noch relativ jungen) WebAssembly-Standards noch nicht so einfach möglich.

Mit steigender Popularität und steigendem Funktionsumfang von WebAssembly ist es sehr wahrscheinlich dass es bald einen Java-zu-WebAssembly-Compiler geben wird, der selbst in WebAssembly implementiert wurde oder problemlos nach WebAssembly übersetzt werden kann. Damit ließe sich dann im Browser Java-Quellcode in WebAssembly konvertieren und ausführen ohne das lästige Laden der Java Laufzeitumgebung und mit tendenziell besserer Performance als DoppioJVM (vgl. \cite{art:wasm_speed}).

Mit WebAssembly wäre es außerdem leichter möglich, das Feature der Code-Ausführung im Browser um weitere Sprachen zu erweitern und damit den potenziellen Nutzerkreis für Weclare zu erweitern.
%
\chapter{Fazit und Verbesserungsmöglichkeiten}
\label{chap:fazit}
Wie im Rest der Arbeit soll sich auch das Fazit an den drei Kern-Anforderungen entlanghangeln, die zu Beginn festgelegt wurden. Für jeden Aspekt soll einzeln erörtert werden, ob die Zielsetzung erreicht werden konnte, und welche Erfahrungen bei der Implementation gemacht werden konnten.

Webbasiert:
Trotz einiger Nachteile, die der Browser als Laufzeitumgebung mit sich bringt, erscheint es nur logisch, ein CRS webbasiert zu implementieren. Die technischen Anforderungen an das ausführende System sind relativ gering. Die Anforderungen an die User Experience und eine gut gestaltete Benutzeroberfläche im Gegensatz dazu eher hoch. Es erscheint nicht mehr zeitgemäß für eine solch simple (in Bezug auf die technischen Anforderungen) Aufgabe Software extra herunterladen und installieren zu müssen. Bei StuReSy war diese Entwurfs-Entscheidung nachvollziehbar, da dort auch Hardware-Abstimmungsgeräte und ein Plugin-System unterstützt werden sollten, die in einer Web-Anwendung deutlich schwerer zu integrieren gewesen wären.

Das React-Framework hat sich als gute Wahl für diese Aufgabe herausgestellet. Die bereitgestellten Konzepte und Bibliotheken waren relativ einfach zu verstehen und haben eine schnelle Entwicklung ermöglicht. Sicherlich hätte die Implementierung aber auch mit alternativen Frameworks erfolgreich funktioniert. Die Unterschiede der großen JavaScript-Frameworks (React, Vue) scheinen nur marginal zu sein.

Peer-2-Peer-Topologie: Die Implementierung einer P2P-Topologie auf Basis von WebRTC erwies sich als schwierig. Der WebRTC-Standard ist noch nicht fertiggestellt, und die Abweichungen zwischen den Browser-Herstellern sind relativ groß. WebRTC im Produktiveinsatz mit breiter Kompatibilität zu verwenden ist sehr aufwändig. Die Verwendung von WebRTC-Wrappern, die die Schnittstelle vereinfachen und das Signalling regeln erscheint gerade in kleinen Projekten notwendig zu sein. Leider gibt es in diesem Bereich nicht sehr viele, und nicht sehr aktuelle Open-Source-Bibliotheken.

Das fertige System funktioniert meistens einwandfrei. Allerdings kann es schnell zu Komplikationen kommen, zum Beispiel wenn die Verbindung der Teilnehmer unterbrochen wird, wenn Browser nicht unterstützt werden. Für den Produktiveinsatz sollte die Software an dieser Stelle noch robuster werden.

Code-Ausführung im Browser: Weclare zeigt, dass die Ausführung von Java-Quellcode im Browser grundsätzlich funktioniert.




%%%%%% Chapters End

%% bibliography and other stuff
\backmatter

\typeout{===== Section: literature}
%% read the documentation for customizing the style
\printbibliography
\typeout{===== Section: nomenclature}
%% uncomment if a TOC entry is needed
%% \addcontentsline{toc}{chapter}{Glossar}
\renewcommand{\nomname}{Glossar}
\clearpage
\markboth{\nomname}{\nomname} %% see nomencl doc, page 9, section 4.1
\printnomenclature

%% index
\typeout{===== Section: index}
\printindex

\HAWasurency

\end{document}
