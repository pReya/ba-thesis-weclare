\documentclass[draft=false
              ,paper=a4
              ,twoside=false
              ,fontsize=11pt
              ,headsepline
              ,BCOR=10mm
              ]{scrbook}
\usepackage[ngerman,english]{babel}
%% see http://www.tex.ac.uk/cgi-bin/texfaq2html?label=uselmfonts
\usepackage[T1]{fontenc}
\usepackage[utf8]{inputenc}
\usepackage{libertine}
\usepackage{pifont}
\usepackage{microtype}
\usepackage{textcomp}
\usepackage[german,refpage]{nomencl}
\usepackage[ngerman,colorlinks=true]{hyperref}
\usepackage{setspace}
\usepackage{makeidx}
\usepackage{listings}
\usepackage{csquotes}
\usepackage{tikz}
\usepackage[
    backend=biber,
    style=ieee,
    sorting=nyt,
]{biblatex}
\addbibresource{literature.bib}
\usepackage{soul}
\usepackage{hawstyle}
\usepackage{scrhack}
\usepackage{subfiles}
\usepackage{graphicx}
\usepackage{float}
\usepackage{parskip}

%% define some colors
\colorlet{BackgroundColor}{gray!20}
\colorlet{KeywordColor}{blue}
\colorlet{CommentColor}{black!60}
%% for tables
\colorlet{HeadColor}{gray!60}
\colorlet{Color1}{blue!10}
\colorlet{Color2}{white}

%% configure colors
\HAWifprinter{
  \colorlet{BackgroundColor}{gray!20}
  \colorlet{KeywordColor}{black}
  \colorlet{CommentColor}{gray}
  % for tables
  \colorlet{HeadColor}{gray!60}
  \colorlet{Color1}{gray!40}
  \colorlet{Color2}{white}
}{}
\lstset{
  language=Java,
  numbers=left,
  numberstyle=\tiny,
  stepnumber=1,
  numbersep=5pt,
  basicstyle=\ttfamily\small,
  keywordstyle=\color{KeywordColor}\bfseries,
  identifierstyle=\color{black},
  commentstyle=\color{CommentColor},
  backgroundcolor=\color{BackgroundColor},
  captionpos=b,
  fontadjust=true
}
\lstset{escapeinside={(*@}{@*)}, % used to enter latex code inside listings
        morekeywords={uint32_t, int32_t}
}
\ifpdfoutput{
  \hypersetup{bookmarksopen=false,bookmarksnumbered,linktocpage}
}{}

%% more fancy C++
\DeclareRobustCommand{\cxx}{C\raisebox{0.25ex}{{\scriptsize +\kern-0.25ex +}}}

\clubpenalty=10000
\widowpenalty=10000
\displaywidowpenalty=10000
\setcounter{secnumdepth}{3}

% unknown hyphenations
\hyphenation{
}

%% recalculate text area
\typearea[current]{last}

\makeindex
\makenomenclature

\begin{document}
\selectlanguage{ngerman}

%%%%%
%% customize (see readme.pdf for supported values)
\HAWThesisProperties{Author={Dennis Schröder}
                    ,Title={\textit{Agentenorientierte Softwareentwicklung} im Kontext der \textit{Multi-Roboter-Wegplanung}}
                    ,EnglishTitle={\textit{Agent-oriented software engineering} in the context of \textit{mutli-robot path planning}}
                    ,ThesisType={Bachelorarbeit}
                    ,ExaminationType={Bachelorprüfung}
                    ,DegreeProgramme={Bachelor of Science Technische Informatik}
                    ,ThesisExperts={Prof. Dr. Martin Becke \and Prof. Dr.-Ing. Andreas Meisel}
                    ,ReleaseDate={1. April 2019}
                  }

%% title
\frontmatter

%% output title page
\maketitle

\onehalfspacing

%% add abstract pages
\HAWAbstractPage
{Agentenorientierte Softwareentwicklung, AoSE, Multi-Roboter-Wegplanung, Agenten, JADE, CoDy Algorithmus, Smart Chair, CaDS}
{\textit{Internet of Things} (IoT), \textit{Ubiquitous Computing} und \textit{Industrie 4.0} sind Begriffe, die in letzter Zeit immer mehr an Bedeutung gewinnen. Gemeinsam haben diese Themengebiete vor allem ihren Fokus auf der Vernetzung von Rechnern. Die Anforderung an die Skalierbarkeit solcher Systeme wird mit der Anzahl der Geräte und dem Grad der Vernetzung steigen und somit zukünftig immer wichtiger werden. Auch mit dem Blick auf das Jahr 2022, in dem 50 Milliarden IoT-Geräte aktiv sein sollen.\newline
Im Rahmen dieser Bachelorarbeit wird ein solches skalierendes System entwickelt. Eine Kernanforderung an das System ist es, mit unterschiedlichen Anzahlen von Teilnehmern souverän umgehen zu können. Die Spanne reicht von wenigen Einzelnen bis vielen Hunderten. Um diesen Grad der Skalierbarkeit zu erreichen, wird das System mit Hilfe der \textit{agentenorientierten Softwareenticklung} (AoSE) umgesetzt. Als Produkt entsteht ein Multi-Agenten-System, das durch das \textit{Java Agent Development Framework} (JADE) implementiert wird. Zusammen mit dem \textit{Cooperative Dynamic} (CoDy)-Algorithmus entsteht ein kooperatives System aus Agenten die sich in einer zweidimensionalen \textit{Gridworld} bewegen können.\newline
Denn das Projekt, an dem die Skalierbarkeit des AoSE-Ansatzes aufgezeigt wird, beschäftigt sich mit der Erweiterung der \textit{Smart Chairs} der \textit{Communication and Distributed System} (CaDS)-Arbeitsgruppe der \textit{Hochschule für angewandte Wissenschaften Hamburg}, um die Fähigkeit an bestimmte Plätze fahren zu können.\newline
In einer selbst entwickelten Simulationsumgebung werden dann Experimente aus \cite{book:regele} wiederholt. Anhand der Ergebnisse, den zuvor definierten qualitativen Merkmalen und Anforderungen sowie den Kerneigenschaften des Multi-Agenten-Systems, Dezentralität und Autonomie, wird dann die Skalierbarkeit diskutiert und das Projekt bewertet.}
{Agent-oriented software engineering, AoSE, Mutli-robot path planning, Agents, JADE, CoDy Algorithm, Smart Chair, CaDS}
{\textit{Internet of Things} (IoT), \textit{ubiquitous computing} and \textit{Industry 4.0} are terms that are gaining importance lately. These topics share a focus on more complex networked computers. The scalability requirement of such systems will increase with the number of devices and the degree of networking and thus become more important in the future. Experts estimate that in the year 2022 that 50 billion IoT devices will be active.\newline
Such a scaling system will be developed within the scope of this bachelor thesis. A core requirement of the system is to handle different numbers of participants confidently. The range varies from a few to many hundreds individuals. To achieve this level of scalability, the system is implemented using \textit{agent-oriented software engineering} (AoSE). The product is a multi-agent system, which is implemented with the \textit{Java Agent Development Framework} (JADE). Together with the \textit{Cooperative Dynamic} (CoDy) algorithm, it creates a cooperative system of agents that can move in a two-dimensional \textit{gridworld}.\newline
The project, which demonstrates the scalability of the AoSE approach, addresses the extension of the \textit{Smart Chairs} of the \textit{Communication and Distributed System} (CaDS) team from the \textit{Hamburg University of Applied Sciences} to be able to drive to certain places.\newline
Experiments from \cite{book:regele} will be repeated in a self developed simulator. Based on the results, the previously defined qualitative characteristics and requirements as well as the core properties of the multi-agent system, decentralization and autonomy, the scalability is discussed and the project evaluated.}

\newpage
\singlespacing

\setcounter{tocdepth}{3}
\tableofcontents
\newpage
%% enable if these lists should be shown on their own page
\listoftables
\listoffigures
\lstlistoflistings

%% main
\mainmatter
\onehalfspacing

%%%%%% Chapters
\chapter{Einleitung}
\section{Was sind Classroom-Response-Systeme?}
\label{chap:was_sind_crs}
Interaktivität ist ein Schlüsselfaktor für gute Lehrveranstaltungen (vgl. \cite[S. 1]{art:ieee}). Der klassische Frontalunterricht gilt nicht mehr als besonders effektiv. Gleichzeitig ist es für Lehrende immer schwieriger, besagte Interaktivität herzustellen, je größer die Anzahl der Teilnehmer einer Veranstaltung ist. In kleinen Gruppen kann durch mündliche Kommunikation meistens noch ein gewisser Grad an Interaktivität garantiert werden. Im Lehrbetrieb von großen Hochschulen und Universitäten ist das aber aufgrund der hohen Teilnehmerzahlen kaum noch realisierbar. (vgl. \cite[S. 389]{art:crs_informatik_spektrum})


Eine Möglichkeit, die Interaktivität zu steigern, ist der Einsatz sogenannter \acp{ars} (vgl. \cite[S. 5]{art:ieee}). Im akademischen und schulischen Umfeld werden diese Systeme auch noch spezifischer als \acp{crs} bezeichnet. Dabei handelt es sich um Hardware- oder Software-Lösungen, die es einem Dozenten ermöglichen, während einer Veranstaltung beliebig viele Fragen nacheinander an das Publikum zu stellen. Diese können entweder mithilfe dedizierter Hardware-Geräte (sogenannte Clicker) oder mit einem Smartphone, Laptop oder Tablet-PC beantwortet werden. Üblicherweise werden dabei verschiedene Fragetypen unterstützt, zum Beispiel Simple- und Multiple-Choice-Varianten.

Ein Merkmal dieser Systeme ist die Anonymität der Antworten, denn nur so können die Teilnehmer ehrlich antworten und lernen, ihren Wissensstand tatsächlich besser einzuschätzen (vgl. \cite[S. 106]{art:crs_literature_review}). CRS sind explizit keine Werkzeuge, um digitale Tests oder Prüfungen durchzuführen.

Die Nutzung solcher Systeme im akademischen Umfeld ist bereits relativ weit verbreitet (vgl. \cite[S. 392]{art:crs_informatik_spektrum}. Weltweit werden diese Lösungen an vielen Hochschulen eingesetzt (vgl. \cite{web:elanwiki}). Studien und Umfragen attestieren diesen Systemen eine Steigerung der Interaktivität, eine Verbesserung der Selbsteinschätzung und eine Steigerung der Aufmerksamkeit in Lehrveranstaltungen, zum Beispiel an der University of Wisconsin-Milwaukee \cite[S. 5]{paper:wisconsin_clicker}:

\begin{quote}
„Students similarly reported that the use of clickers increased their engagement, involvement, and interaction, and help students pay attention in class.“
\end{quote}

Oder auch nach zehnjärigem Einsatz an der Harvard University \cite[S. 6]{paper:harvard_tenyears}:

\begin{quote}
„We find that, upon first implementing Peer Instruction, our students’ scores on the Force Concept Inventory and the Mechanics Baseline Test improved dramatically, and their performance on traditional quantitative problems improved as well.“
\end{quote}


Eine Bewertung des didaktischen Konzepts von CRS ist daher nicht Bestandteil dieser Arbeit. Die Wirksamkeit und Relevanz wird als gegeben angenommen.

Auch an der \ac{haw} Hamburg werden CRS eingesetzt. Eine der verwendeten Software-Lösungen, die dort im Department Informatik zum Einsatz kommt, ist das \ac{sturesy} – eine Software, die im Rahmen einer Bachelorarbeit an der Universität Hamburg entwickelt wurde \cite{sturesy}. Wegen der positiven Erfahrungen, die der Autor und Herr Axel Schmolitzky, einer der Lehrenden an der HAW Hamburg, mit der Software gemacht haben, wird StuReSy im Lauf dieser Arbeit als funktionales Vorbild für die Entwicklung eines neuen CRS dienen, welches das System konzeptionell verbessert, bestehende Probleme behebt und das Anwendungsgebiet stärker auf den Spezialfall der Programmierlehre zuspitzt.

\newpage
\section{Motivation: Vom Software-Download zur Web-Applikation}
\label{chap:webapp}
Die Disziplin der Software-Entwicklung hat in den letzten Jahrzehnten enorme Sprünge gemacht. Die rasante Leistungssteigerung von Hardware, die zunehmende Mobilität von Computersystemen bei gleichzeitiger Preisreduzierung und die Zunahme und Qualität der Vernetzung haben Software-Entwicklung radikal beschleunigt und verändert. Ein fundamentales Prinzip hat sich allerdings nur wenig verändert: Die Distribution von Software.

Seit den Anfängen der PC-Ära in den 1980er Jahren bis in die heutige Zeit wird Software zu einem bestimmten Zeitpunkt auf ein Zielsystem übertragen und dann lokal auf dem System gestartet und ausgeführt. Nur einige Details an diesem Prozess haben sich im Laufe der Jahre geändert:

Während früher ein physischer Datenträger die Quelle der Software war, ist es heute meistens das Internet. Früher mussten die Quellen manuell im Internet gefunden werden, heute gibt es Software-Kataloge oder App-Stores, welche die Suche nach Software erleichtern. Diese Entwicklung 

Seit wenigen Jahren etabliert sich eine alternative Variante der Software-Distribution: die Web-Applikation. Das Internet, das ursprünglich nur dazu gedacht war, statische Texte darzustellen, hat sich zu einer beliebten Art der Software-Distribution verwandelt. Dank der Programmiersprache JavaScript, sind Webbrowser heute mächtige Laufzeitumgebungen und können in vielen Belangen genau so viel leisten wie native Anwendungen.

Webbrowser stellen heute eine Abstraktionsebene dar, die es ermöglicht plattformunabhängig zu entwickeln. Browser nehmen heute immer öfter die Funktion ein, die früher zum Beispiel von virtuellen Maschinen (etwa der Java Virtual Machine) erledigt wurde.

Die konzeptionellen Vorteile von Web-Anwendungen sind offensichtlich: Das Auffinden der Software ist gleichbedeutend mit dem Merken einer Internetadresse. Das Herunterladen der Software wird vor dem Nutzer versteckt (sichtbar ist nur das Aufrufen einer Webseite), eine Installation entfällt. Software-Updates werden ebenfalls vor dem Nutzer verborgen (bei jedem Aufruf der Applikation wird jeweils die neuste Programmversion geladen).

Natürlich gibt es auch Nachteile: Web-Applikationen benötigen meistens eine dauerhafte und schnelle Internetverbindung. Außerdem ist die Leistungsfähigkeit gegenüber nativen Anwendungen an einigen Stellen noch eingeschränkt (Ahead-of-time-Compile vs. INterpretiert).


\label{chap:einleitung}
%
\chapter{Theoretischer Hintergrund}
\label{chap:theorie}
\section{Agentenorientierte Softwareentwicklung}
\label{chap:aose}
Wie im Rest der Arbeit soll sich auch das Fazit an den drei Kern-Anforderungen entlanghangeln, die zu Beginn festgelegt wurden. Für jeden Aspekt soll einzeln erörtert werden, ob die Zielsetzung erreicht werden konnte, und welche Erfahrungen bei der Implementation gemacht werden konnten.

Webbasiert:
Trotz einiger Nachteile, die der Browser als Laufzeitumgebung mit sich bringt, erscheint es nur logisch, ein CRS webbasiert zu implementieren. Die technischen Anforderungen an das ausführende System sind relativ gering. Die Anforderungen an die User Experience und eine gut gestaltete Benutzeroberfläche im Gegensatz dazu eher hoch. Es erscheint nicht mehr zeitgemäß für eine solch simple (in Bezug auf die technischen Anforderungen) Aufgabe Software extra herunterladen und installieren zu müssen. Bei StuReSy war diese Entwurfs-Entscheidung nachvollziehbar, da dort auch Hardware-Abstimmungsgeräte und ein Plugin-System unterstützt werden sollten, die in einer Web-Anwendung deutlich schwerer zu integrieren gewesen wären.

Das React-Framework hat sich als gute Wahl für diese Aufgabe herausgestellet. Die bereitgestellten Konzepte und Bibliotheken waren relativ einfach zu verstehen und haben eine schnelle Entwicklung ermöglicht. Sicherlich hätte die Implementierung aber auch mit alternativen Frameworks erfolgreich funktioniert. Die Unterschiede der großen JavaScript-Frameworks (React, Vue) scheinen nur marginal zu sein.

Peer-2-Peer-Topologie: Die Implementierung einer P2P-Topologie auf Basis von WebRTC erwies sich als schwierig. Der WebRTC-Standard ist noch nicht fertiggestellt, und die Abweichungen zwischen den Browser-Herstellern sind relativ groß. WebRTC im Produktiveinsatz mit breiter Kompatibilität zu verwenden ist sehr aufwändig. Die Verwendung von WebRTC-Wrappern, die die Schnittstelle vereinfachen und das Signalling regeln erscheint gerade in kleinen Projekten notwendig zu sein. Leider gibt es in diesem Bereich nicht sehr viele, und nicht sehr aktuelle Open-Source-Bibliotheken.

Das fertige System funktioniert meistens einwandfrei. Allerdings kann es schnell zu Komplikationen kommen, zum Beispiel wenn die Verbindung der Teilnehmer unterbrochen wird, wenn Browser nicht unterstützt werden. Für den Produktiveinsatz sollte die Software an dieser Stelle noch robuster werden.

Code-Ausführung im Browser: Weclare zeigt, dass die Ausführung von Java-Quellcode im Browser grundsätzlich funktioniert.




%
\section{JADE}
\label{chap:jade}
Die \textit{Foundation for Intelligent Physical Agents} (FIPA) der \textit{IEEE Computer Society} hat im Bereich des Multi-Agenten-Systems mehrere Standards erarbeitet. Diese Standards beschreiben alle grundlegenden Elemente und Funktionen, die für eine Multi-Agenten-Plattform benötigt werden. \cite{article:flexibleSoftware}

Es gibt einige Projekte, die diesen Standard implementieren \cite{web:fipaList}. Unter diesen eignet sich das \textit{Java Agent Development Framework} (JADE) am besten für dieses Projekt. Zum einen ist das JADE Framwork in \textit{Java} implementiert und somit plattformunabhängig \cite{web:java}. Es ist open-source. Es kann also eigenständig verändert und erweitert werden. Zudem ist die Benutzung des Frameworks nicht mit Lizenzkosten verbunden. Zuletzt ist JADE in Form von \cite{book:jade} detailliert beschrieben.

In den folgenden Kapiteln wird eine Auswahl wichtiger Grundkonzepte des JADE Framworks vorgestellt. 

\subsection{Grundlagen}
\label{chap:jade_grundlagen}
\begin{figure}[H]
    \includegraphics[width=10cm]{images/jade_structure.png}
    \centering
    \caption{\textit{Entity-Relationship}-Diagramm aus \cite{book:jade}}
    \label{fig:jade_structure}
\end{figure}

\paragraph{\textit{Agent Platform} (AP)}
Die AP beschreibt die physikalische Infrastruktur in der die Agenten ausgeführt werden. Hier sind die Rechner, die Netzwerke, die Betriebssysteme, das \textit{Agent Management System}, die Agenten selbst und zusätzliche Software mit inbegriffen.

\paragraph{Agent}
Ein Agent existiert in der AP und bietet ein oder mehrere Services die mit Hilfe einer \textit{Service Description} veröffentlicht sind. Ein Agent hat ein eindeutigen \textit{FIPA Agent Identifier} (AID).  

\paragraph{Diretory Faciliator (DF)}
Der DF ist eine optionale Komponente. Der DF stellt den Agenten einen "'Gelbe Seiten"'-Service zur Verfügung und hält eine komplette Liste aller Agenten. Der "'Gelbe Seiten"'-Service wird von den Agenten genutzt um ihre Services zu registrieren und somit den anderen Agenten verfügbar zu machen. Die AP kann beliebig viele DF starten, die ihre Daten untereinander synchronisieren.

\paragraph{Agent Management System (AMS)}
Das AMS ist für das Erstellen und Löschen der Agenten verantwortlich. Jeder Agent muss sich beim AMS registrieren. Bei diesem Schritt vergibt das AMS dann die AID für die Agenten. Ein Agent beendet sich, wenn er sich vom AMS abmeldet. Das AMS ist als eine Entität zu verstehen und erstreckt sich auch über mehrere Rechner.

\paragraph{Message Transport Service (MTS)} Dieser Service wird von der AP bereitgestellt. Über ihn können die Agenten Nachrichten austauschen.
%
\subsection{Behaviour}
\label{chap:jade_agent}
Ein \textit{Behaviour} ist eine Aufgabe die ein Agent ausführen kann und als erbende Klasse von \texttt{jade.core.behaviours.Behaviour} implementiert ist. Für jeden Agenten startet die JADE Umgebung einen Thread. Ein \textit{Behaviour} wird über einen Scheduler auf diesem Thread ausgeführt. Der Agent fügt ein \textit{Behaviour} mit \texttt{addBehaviour()} der Scheduling-Queue hinzu. Dies kann der Agent während der Initialisierungsphase in der \texttt{steup()} Methode tun oder innerhalb eines \textit{Behaviours}. Jede \textit{Behaviour}-Subklasse muss die \texttt{action()} und \texttt{done()} Methode implementieren. Die \texttt{action()} Methode enthält die Logik, also den auszuführenden Code, des \textit{Behaviours}. Die \texttt{done()} Methode gibt einen \texttt{boolean} zurück, der Aussage darüber trifft, ob das \textit{Behaviour} seine Aufgabe abgeschlossen hat und damit aus Scheduling-Queue entfernt werden soll. Ein Agent kann mehrere \textit{Behavoiur} ausführen. Ein \textit{Behavoiur} ist jedoch nicht preemptive. Wenn die \texttt{action()} Methode vom Scheduler aufgerufen wird, kann diese nicht unterbrochen werden. Das \textit{Behaviour} muss also selbständig diese Ressource wieder freigeben. \cite{book:jade}

Das JADE Framwork stellt jedoch nicht nur den Basistypen zur Verfügung, sondern implementiert weitere abstrakte \textit{Behaviour}. Folgend wird eine Auswahl aus \cite{book:jade} vorgestellt.

\paragraph{One-Shot Behaviour}
Für das \texttt{jade.core.behaviours.OneShotBehaviour} muss eine erbende Klasse nur die \texttt{action()} Methode implementieren. \texttt{done()} liefert standardmäßig \texttt{true} zurück. Ein \textit{One-Shot Berhaviour} wird also nur einmal ausgeführt.

\paragraph{Cyclic Behaviour}
Das \texttt{jade.core.behaviours.CyclicBehaviour} ist dem \textit{One-Shot Behaviour} recht ähnlich. Die \texttt{done()} Methode liefert jedoch standardmäßig \texttt{false} zurück. Ein \textit{Cyclic Behaviour} beendet sich also nie.

\paragraph{Ticker Behaviour}
Das \texttt{jade.core.behaviours.TickerBehaviour} implementiert sowohl \texttt{action()} als auch \texttt{done()} Methoden. Die \texttt{done()} Methode liefert immer \texttt{false} zurück. Die \texttt{action()} Methode führt die \texttt{onTick()} Methode periodisch aus. Das Zeitintervall wird über den Konstruktor definiert. Das \textit{Ticker Behavior} ist selbst eine abstrakte Klasse. Erbende Klassen implementieren die Methode \texttt{onTick()}.
%
\subsection{Kommunikation}
\label{chap:jade_kommunikation}
Agenten interagieren miteinander. Dies geschieht indirekt über das Verändern der Umwelt oder durch direkte Kommunikation. Die direkte Kommunikation ist wahrscheinlich eines des wichtigsten Funktionen des JADE Frameworks. Die Kommunikation zwischen Agenten erfolgt durch den asynchronen Nachrichtenaustausch. Jeder Agent besitzt eine Queue in der alle Nachrichten eines Agenten empfangen werden. Eine Nachricht hält dabei mehrere Felder:
\begin{itemize}
\item Die ID des \textbf{Senders}
\item Eine Liste, die alle \textbf{Empfänger} enthält.
\item Die \textbf{Absicht} der Nachricht. Die FIPA definiert hier eine Liste mit Möglichkeiten. Zum Beispiel "'Inform"'. Der Sender berichtet den Empfänger einen Fakt.
\item Den \textbf{Inhalt}, den der Sender mitteilen möchte.
\item Die \textbf{Kodierung} der Nachricht. So das die Empfänger wissen, wie die Nachricht zu lesen ist.
\end{itemize} \cite{book:jade}

%
\section{CoDy Algorithmus}
\label{chap:cody}
Ein Wegplanungs-Algorithmus kann entweder verteilt oder zentral arbeiten. Ein zentraler Algorithmus hat aber das Problem, dass der Berechnungsaufwand quadratisch von der Anzahl der Agenten abhängig ist (TODO Quelle). Ist für diese Arbeit also nicht interessant, da ein potentiell stark skalierendes Agenten-System zum Einsatz kommt. Ein echtzeitfähiger, verteilter und skalierender Algorithmus ist der \textit{CoDy Algorithmus}. Der CoDy Algorithmus berechnet die Wege für jeden Agenten aus Sicht der jeweiligen Agenten und löst mit Hilfe von heuristischer Prioritätsanpassung auftretende Konflikte. Der \textit{CoDy Algorithmus} hat bestimmte Voraussetzungen an das implementierende System. Es soll verteilt sein, muss den Agenten die Möglichkeit bieten, untereinander zu kommunizieren und homogen sein. Zusätzlich muss ein Agent in der Lage sein seine Umgebung zu erfassen, entweder durch die eigene Sensorik oder durch Kommunikation mit anderen Agenten. Alle Voraussetzungen können erfüllt werden. \cite{book:regele}

Die Funktionsweise des CoDy Algorithmus wird in den folgenden Unterkapiteln genauer beschrieben. Es soll aber nicht Sinn dieser Arbeit sein, die Dissertation wiederzugeben. Die grundlegenden Elemente und Funktionen werden nicht nur zum Grundverständnis wiedergegeben. Vor allem soll aber eine Basis geschaffen werden, auf deren Grund dann im Kapitel (TODO link) beschrieben werden kann, wie der Algorithmus umgesetzt wird und wo es zu Abweichungen kommt.
\subsection{Grundlagen}
\label{chap:grundlagen}
In diesem Kapitel werden die grundlegenden Mechanismen und Elemente des CoDy Algorithmus vorgestellt.

\subsubsection{Das Weltmodell}
\label{chap:weltmodell}
\input{chapter/theorie/cody/grundlagen/weltmodell.tex}
%
\subsubsection{Die Entfernungskarte}
\label{chap:entfernungskarte}
\input{chapter/theorie/cody/grundlagen/entfernungskarte.tex}
%
\subsubsection{Die Umgebungskarte}
\label{chap:umgebungskarte}
\input{chapter/theorie/cody/grundlagen/umgebungskarte.tex}
%
\subsubsection{Die Erreichbarkeitskarte}
\label{chap:erreichbarkeitskarte}
\input{chapter/theorie/cody/grundlagen/erreichbarkeitskarte.tex}
%
\subsubsection{Der Raum-Zeit-Pfad}
\label{chap:raum-zeit-pfad}
\input{chapter/theorie/cody/grundlagen/raum-zeit-pfad.tex}

%
\subsection{Konfliktverarbeitung}
\label{chap:konfliktverarbeitung}
Die Agenten des \textit{CoDy Algorithmus} sind kooperativ. Sie sind also in der Lage die eigenen "'Bedürfnisse"' zurückzustellen. Man kann zwischen passiver und aktiver Kooperation unterscheiden. Bei der aktiven Kooperation kann ein Agent die Planung übernehmen und bei anderen Agenten nach Zustimmung oder Ablehnung fragen. Bei der passiven Kooperation müssen alle in einem Konflikt beteiligten Agenten die gesamte Situation berechnen und dann anschließend in einer Verhandlung ihre Lösungen diskutieren. Die passive Kooperation hat gegenüber der aktiven Kooperation vor allem den Vorteil, dass der Kommunikationsaufwand deutlich geringer ist. Sie ist aber auch robuster und der Berechnungsaufwand, so wie sie im CoDy Algorithmus umgesetzt ist, geringer. Die echte passive Kooperation kann sehr ineffizient sein. Deshalb benutzt der CoDy Algorithmus eine abgewandelte Form. Agenten, die potenziell in einen Konflikt geraten können, kommunizieren in einer festen Reihenfolge (siehe Kapitel \ref{chap:kommunikation}) und planen nur ihre eigenen Wege. \cite{book:regele}\newline Folgende Beispielsituation:
Agent "'0"' überschreibt einen Teil des geplanten Weges von Agent "'1"'. Dann schickt Agent "'0"' seinen neuen Plan an alle Agenten. Wenn Agent "'1"' wieder an der Reihe ist, plant dieser mit den neuen Daten. Wann welcher Agent, welche Wege überschreiben kann, wird in Kapitel \ref{chap:prioritaeten} beschrieben.

\subsection{Kommunikation}
\label{chap:kommunikation}
\input{chapter/theorie/cody/konfliktverarbeitung/kommunikation.tex}
%
\subsection{Prioritäten}
\label{chap:prioritaeten}
\input{chapter/theorie/cody/konfliktverarbeitung/prioritaeten.tex}
%
\subsection{Algorithmus}
\label{chap:algorithmus}
In diesem Kapitel werden die zuvor erläuterten Elemente in Kontext gesetzt. 

Wenn die Agenten initialisiert werden, sind ihnen die statischen Hindernisse und die eigene Position bereits bekannt. Alle Agenten werden mit den gleichen Parametern erzeugt. Weil für den ersten Berechnungsschritt die Agenten die Positionen der anderen nicht kennen, planen alle Agenten stehen zu bleiben. Ein Agent führt periodisch die nächste geplante Bewegung aus. Zwischen den Bewegungen finden die Berechnungsschritte statt. Für diese Arbeit findet aus Gründen der Übersicht, nur ein Berechnungsschritt pro Bewegungsschritt statt. Parallel zu den Bewegungs- und Berechnungsschritten, empfängt ein Agent die Nachrichten der anderen Agenten. \cite{book:regele}

Ein Berechnungsschritt läuft folgendermaßen ab:\newline
Damit ein Berechnungsschritt beginnen kann, müssen von allen Agenten, auf die gewartet wird, Nachrichten empfangen werden. Wenn das erfüllt ist, wird die Umgebungskarte zeitlich und räumlich zentriert. Darauffolgend werden die neuen geplanten Wege in die Umgebungskarte eingetragen. Jetzt plant der Agent mit Hilfe einer Erreichbarkeitskarte, den eigenen Weg. Prioritätsanpassungen und eventuelle Neuberechnungen werden gemäß \ref{chap:prioritaeten} durchgeführt. Der geplante Weg wird an alle Agenten per Nachricht übermittelt. \cite{book:regele}
%
\chapter{Software Architektur}
\label{chap:architektur}
In diesem Kapitel wird erörtert, wie und aus welchen Elementen, sich das einstehende Softwareprojekt gliedert. Dabei wird jedoch nur die grobe Struktur des Projekt und die für den Anwendungsfall relevanten Aspekte beleuchtet.

Das Projekt ist in \textit{Java} implementiert. Um grafische Benutzeroberflächen zu ermöglichen, wird \textit{JavaFX} eingesetzt. Das Projekt richtet sich nach dem MVC-Modell aus \cite{web:mvc}.

\begin{sloppypar}
Kernaspekt des Projekt ist der Agent. Dieser ist in Form der Klasse \texttt{codyAgent.CoDyAgent} implementiert, die die Klasse \texttt{jade.core.Agent} erweitert. Ein \textit{CoDy-Agent} implementiert, abgesehen von der Wegplanung, die Logik des CoDy Algorithmus. Die Wegplanung ist in Objekten gekapselt, die den verschiedenen Kartentypen des CoDy Algorithmus entsprechen. Diese werden jedoch direkt von dem Agenten gesteuert. Wie in \cite{book:jade} empfohlen, implementiert der CoDy Agent sein Verhalten als innere Klassen.

\paragraph{PlanningBehaviour}
Das \textit{PlanningBehaviour} übernimmt mehrere Aufgaben. Zum einen führt es die Wegplanung aus, zum anderen wandelt es den geplanten Weg als \texttt{String} im \textit{JSON} Format um. Übernimmt also das \textit{Marshalling}. Außerdem wird dieser \texttt{String} als Nachricht an alle Agenten gesendet. Als Performativ wird hier "'Informieren"', also \texttt{jade.lang.acl.ACLMessage.INFORM}, genutzt. Das \textit{PlanningBehaviour} ist ein \texttt{CyclicBehaviour} mit zwei Status. Die Statusm sind wie in \cite{book:jade} vorgeschlagen, als \texttt{switch case} umgesetzt. Befindet sich dieses Verhalten im ersten Status, dann wird darauf gewartet, dass von allen Agenten, auf die gewartet wird, eine Nachricht eintrifft. Im zweiten Status findet dann der Berechnungsschritt, also die Wegplanung, wie in \ref{chap:algorithmus} beschrieben, statt.
\end{sloppypar}

\paragraph{MessageReceiveBehaviour}
Auch das \textit{MessageReceiveBehaviour} ist ein \texttt{CyclicBehaviour}. Dieses kontrolliert regelmäßig, ob neue Nachrichten vorliegen und übernimmt dann das \textit{Unmarshalling}.

\paragraph{MovementBehaviour}
Dieses Behaviour ist ein \texttt{TickerBehaviour}. In regelmäßigen, durch eine Periode vorgegebenen, Abständen wird der aktuelle Bewegungsplan gelesen. Das MovementBehaviour steuert dann über den \textit{Hardware Abstraction Layer} (HAL), die nächste durchzuführende Bewegung. 

\paragraph{CoDyAgentDiscoveryBehaviour}
Das \textit{CoDyAgentDiscoveryBehaviour} ist ein \texttt{OneShotBehaviour}. Es sucht jene Agenten, die bei dem \texttt{DFService} einen CoDy-Service registriert haben und speichert diese in eine dem Agenten verfügbare Liste.

Ein Agent kann sich nicht selber ins Leben rufen. Diese Aufgabe wird von der Simulationsumgebung übernommen. Die Simulationsumgebung ist Teil eines \texttt{Controllers} des \textit{JavaFX} Projekts. Der \texttt{Controller} stellt den HAL, startet die Simulation und steuert die grafischen Bewegungen der Agenten. Um eine Simulation zu starten, wird im ersten Schritt eine JSON-Datei eingelesen, die die Karteninformationen, Parameter sowie die Start- und Zielpositionen der Agenten enthält. Daraufhin wird über die \texttt{jade.core.Runtime} die JADE Umgebung gestartet. Über diese Umgebung werden die Agenten initialisiert und gestartet. 
Der angesprochene \texttt{HAL} ist als innere Klasse implementiert und erfüllt das \texttt{CoDyAgentHAL} Interface. Die \texttt{SimulationCoDyHAL} ermöglicht dem \texttt{Controller} die Bewegungen der CoDy Agenten darzustellen. Außerdem werden die Bewegungen aufgezeichnet, so das sie nach der eigentlichen Simulation wiedergegeben werden können. Nachdem die Agenten initialisiert und gestartet sind, wartet der Simulator, bis alle Agenten ihre jeweiligen Ziele erreicht und schließt dann die JADE Umgebung. 

%
\chapter{Methodik}
\label{chap:methodik}
\section{Validierung}
\label{chap:validierung}
\input{chapter/methodik/validierung.tex}
%
\section{Simulationsumgebung}
\label{chap:simulator}
Um die Experimente ausführen zu können, wurde eine Multi-Agenten-Simulationsumgebung entwickelt. Auf eine bereits existierende Simulationsumgebung wurde nicht zurückgegriffen, da dort die Kommunikationsebene bereits vorgegeben ist. Entweder existieren die Agenten als Objekte auf einem großen Server und rufen die Methoden bei anderen Agenten direkt auf (zum Beispiel \cite{web:marsGroup}), oder die Simulationsumgegbung gibt bestimmte Nachrichtenformate und Verhaltensmuster für Agenten vor (zum Beispiel \cite{web:mass}). Da es in dieser Arbeit nicht nur darum geht Experimente für einen verteilten Algorithmus zu wiederholen, sondern auch Software für ein Multi-Roboter-System zu entwickeln, sind diese Einschränkungen groß genug, um eine eigene Simulationsumgebung zu entwickeln.

Der Simulator wurde dabei rudimentär entwickelt. Der Simulator startet die JADE Umgebung und erzeugt alle Agenten. Außerdem stellt die Umgebung den Agenten die Karte für das jeweilige Experiment zur Verfügung und beobachtet, wann ein Experiment erfolgreich beendet wurde. Zusätzlich bietet der Simulator die Möglichkeit, die von den Agenten erzeugten Daten als Dateien zu speichern und visualisiert die Bewegungen der Agenten.

Da JADE jedoch über Threads skaliert \cite{book:jade}, tut dies der Simulator auch. Für ein verteiltes System ist dieses Verhalten kein Problem, für einen einzelnen Rechner bedeutet dies aber, dass eine hohe Anzahl Agenten nicht simuliert werden kann. Hier wäre ein potenter Server nötig oder die Simulationsumgebung müsste dahingehend erweitert werden, dass diese verteilt ausgeführt werden kann. Dies stellt auch die größte Schwachstelle des Simulators dar.
%
\section{Metriken}
\label{chap:metriken}
\input{chapter/methodik/metriken.tex}
%
\chapter{Experimente}
\label{chap:experimente}
In diesem Kapitel wird eine Auswahl relevanter Experimente aus \cite{book:regele} beschrieben. Der Aufbau der Experimente wird durch Abbildungen und assistierend durch Text wiedergegeben. Auf den Abbildungen sind statische Hindernisse grau, freie Zellen weiß, die Startpositionen der Agenten blau und die Zielpositionen grün markiert. Die Experimente teilen sich in zwei Kapitel auf. Das erste Kapitel \ref{chap:allgemein} beschreibt jene Experimente, für die keine Messwerte vorliegen und Kapitel \ref{chap:mitMesswerten} diejenigen, für die Messwerte dokumentiert sind.

Für die Experimente sind die Agenten wie in \cite{book:regele} konfiguriert. Die Parameter werden folgend wiederholt.

Die Enfernungskarten, der Agenten, umfassen immer die gesamte Karte eines Experiments. Für jeden Bewegungsschritt führt ein Agent genau einen Berechnungsschritt durch. Ein Agent ist etwas kleiner als eine Zelle. Die Konfiguration für die Prioritätsangleichung ist wie folgt:\newline
\(BasePrio=0\), \(PrioNoBlock=1\), \(PrioBlock=10\) und \(PrioFullBlock=19\). Ein Prioritätsüberlauf, im Falle einer ungelösten Verklemmung, tritt erst ab einem Wert von \(PrioMax=400\) auf. Wenn ein solcher Überlauf eintritt, setzt sich der Prioritätswert auf eine zufällige Zahl zwischen null und zehn zurück. Jedes Experiment wird 30 Mal wiederholt. 
Der Durchmesser \((d)\) der Umgebungskarte variiert. Bei Experimenten mit Karten die 30 Felder breit sind, ist \(d=21\). Für die Experimente mit Messwerten beträgt \(d=13\). Für die anderen Experimente gilt \(d=9\). Die zeitliche Berechnungstiefe ist abhängig von der Dimension der Umgebungskarte und wird folgendermaßen bestimmt \(t\textsubscript{max}=0.75*d\).

Sofern nicht anders gekennzeichnet, sind \cite{book:regele} als Quelle für den Aufbau der Experimente, die Messwerte und die zu erwartenden Beobachtungen anzunehmen.

In den folgenden Unterkapiteln wird häufiger von Gruppen die Rede sein. Im Kontext des CoDy Algorithmus sind damit keine logisch zusammenhängenden Agenten gemeint. Eine Gruppe bezeichnet hier lediglich Agenten, die in Nähe zueinander starten.

\section{Allgemeine Experimente}
\label{chap:allgemein}
Für die folgenden Experimente liegen keine Messwerte vor. Sie eignen sich aber, dank der Tatsache, dass bestimmte Verhaltensmuster erwartet werden, zum Überprüfen, ob der CoDy Algorithmus richtig implementiert wurde.

Für jedes Experiment wird zuerst geklärt, was mit diesem gezeigt werden soll. Darauf folgend wird der Aufbau des Experiments beschrieben und die zu erwartenden Beobachtungen vorgestellt. Abschließend werden, für alle Experimente zusammenfassend, die eigenen Beobachtungen vorgestellt.

Für alle Experimente dieses Kapitel gilt die Erwartungshaltung, dass alle 30 Durchläufe erfolgreich absolviert werden.

%
\subsection{Kurze Engstelle}
\label{chap:engstelle}
Mit diesem Experiment wird gezeigt, dass Agenten Ausweichbewegungen durchführen und Wartephasen einlegen, um andere Agenten passieren zu lassen.

\textbf{Aufbau des Experiments}
\begin{figure}[H]
    \includegraphics[height=40mm]{images/one_slit.png}
    \centering
    \caption{Aufbau für das Durchfahren zweier Agenten durch eine kurze Engstelle}
    \label{fig:engstelle}
\end{figure}
Die Karte misst neun mal drei Felder. In der Mitte der Karte ist eine eins mal eins Feld breite Engstelle. Auf beiden Seiten dieser Engstelle stehen sich Agenten gegenüber, die die Engstelle durchfahren müssen, um ihre Zielposition zu erreichen.

\textbf{Erwartete Beobachtungen}\newline
Der Agent, der zuerst einen Weg plant, fährt direkt auf seine Zielposition. Der andere Agent fährt eine Ausweichposition in der unmittelbaren Nähe der Engstelle an. Nachdem der erste Agent die Engstelle passiert hat, wird sich der zweite Agent auf direkten Weg zu seinem Ziel machen.
%
\subsection{Umweg}
\label{chap:umweg}
Hier wird gezeigt wie ein Agent einen anderen verdrängt, aber auch das Agenten Umwege in Kauf nehmen.

\textbf{Aufbau des Experiments}
\begin{figure}[H]
    \includegraphics[height=40mm]{images/detour.png}
    \centering
    \caption{Aufbau für ein Szenario, in dem ein Agent einen Umweg in Kauf nimmt}
    \label{fig:umweg}
\end{figure}
Die Karte in Abbildung \ref{fig:umweg} misst elf mal drei Felder und wird mittig, horizontal durch einen Streifen aus sieben Feldern getrennt. Die Start- und Zielpositionen der Agenten sind so angeordnet, dass die nördliche Engstelle den kürzeren Weg darstellt und die südliche Engstelle den Umweg.

\textbf{Erwartete Beobachtungen}\newline
Für kleine Umgebungskarten, also solche bei denen sich die Umgebungskarten der beiden Agenten erst nach dem Annähern überlappen, werden sich beide Agenten in der nördlichen Engstelle annähern. Wenn sich die Umgebungskarten dann überlappen, wird einer der Agenten den Zuschlag erhalten und den anderen Agenten verdrängen. Dieser wird dann über die südliche Engstelle, also über den Umweg, sein Ziel anfahren.

Für größere Umgebungskarten wird der Agent, der zuerst seinen Weg plant, direkt und der andere über den Umweg sein Ziel anfahren.
%
\subsection{Tunnel}
\label{chap:tunnel}
Dieses Experiment testet, wie anfällig die Agenten für Verklemmungen sind.

\textbf{Aufbau des Experiments}
\begin{figure}[H]
    \includegraphics[width=\textwidth]{images/tunnel_2_groups.png}
    \centering
    \caption{Ausgangssituation für das Durchqueren eines Tunnels von zwei Gruppen, bestehend aus jeweils sechs Agenten}
    \label{fig:tunnel}
\end{figure}
In diesem Experiment durchqueren zwei Gruppen aus jeweils sechs Agenten eine Engstelle, die fünf Felder lang und ein Feld breit ist. Jeder Agent muss die Engstelle passieren um sein Ziel zu erreichen.

\textbf{Erwartete Beobachtungen}\newline
Im ersten Schritt nähern sich beide Gruppen der Engstelle. Während eine Gruppe anfängt die Engstelle zu durchqueren, fahren die Agenten der anderen Gruppe Ausweichpositionen an und verlängern damit die Engstelle. Im Laufe des Experiments wechseln die Gruppen ihre Rollen häufiger.
%
\subsection{Tunnel mit kleiner Ausweichbucht}
\label{chap:ausweichbucht}
Dieses Experiment dient zur Veranschaulichung der dynamischen Prioritäten. Diese Situation kann, dezentral, nämlich nicht durch feste Prioritäten gelöst werden.

\textbf{Aufbau des Experiments}
\begin{figure}[H]
    \includegraphics[height=32mm]{images/tunnel_turnout.png}
    \centering
    \caption{Aufbau für die Vorbeifahrt zweier Agenten in einem Tunnel mit einer kleinen Ausweichbucht}
    \label{fig:ausweichbucht}
\end{figure}
Zwei sich gegenüberstehende Agenten versuchen, in einer elf Felder langen und ein Feld schmalen Engstelle aneinander vorbei auf ihre Zielpositionen zu fahren. Der Tunnel hat in der Mitte ein zusätzliches freies Feld, das von den Agenten genutzt werden muss, um aneinander vorbei zu fahren.

\textbf{Erwartete Beobachtungen}\newline
Zuerst werden die beiden Agenten aufeinander zu fahren. Dann wird einer der beiden Agenten zurückgedrängt werden. Wenn der zurückgedrängte Agent sich nicht weiter zurückdrängen lässt, wird dieser seine Priorität erhöhen und den zu erst drängenden Agenten in die Ausweichbucht drängen, was dann beiden Agenten ermöglicht aneinander vorbei zu ihren Zielpositionen zu fahren. Es kann auch passieren, dass ein Agent die Ausweichbucht direkt anfährt und es zu keiner Verdrängung kommt.
%
\subsection{Durchfahren einer stehenden Menge}
\label{chap:menge}
In diesem Experiment wird gezeigt, dass obwohl ein Agent sein Ziel schon erreicht hat, immer noch aktiv an der passiven Kooperation teilnimmt.

\textbf{Aufbau des Experiments}
\begin{figure}[H]
    \includegraphics[height=56mm]{images/crowd_drive-through.png}
    \centering
    \caption{Aufbau für das Durchfahren eines Agenten durch ein Menge stehender Agenten}
    \label{fig:menge}
\end{figure}
Die Abbildung \ref{fig:menge} zeigt die Karte für dieses Experiment. Diese hat Dimensionen von elf mal fünf Felder. In der Mitte ist eine fünf Felder lange und drei Felder breite Engstelle. In dieser Engstelle stehen in zwei Reihen hintereinander sechs Agenten. Die Startpositionen dieser Agenten sind gleichzeitig ihre Zielpositionen. Ein weiterer Agent hat seine Startposition auf der linken Seite der Engstelle und seine Zielposition auf der Rechten. Er muss also durch die stehende Menge, um sein Ziel zu erreichen.

\textbf{Erwartete Beobachtungen}\newline
Die Agenten "'3"' und "'4"' werden von Agent "'0"' verdrängt, fahren also von ihren Zielpositionen weg, um Platz für Agent "'0"' zu machen. Es ist auch möglich, dass dabei andere Agenten von ihren Zielpositionen verdrängt werden. Nachdem Agent "'0"' die Menge durchquert hat, fahren alle Agenten wieder zurück zu ihren Zielpositionen.
%
\subsection{Kreuzung}
\label{chap:kreuzung}
Dieses Szenario dient der Beobachtung der Flexibilität der Agenten. Außerdem soll der Unterschied zu einem zentralen Ansatz deutlich gemacht werden und die Skalierbarkeit des Ansatzes gezeigt werden.

\textbf{Aufbau des Experiments}
\begin{figure}[H]
    \includegraphics[width=\textwidth]{images/junction.png}
    \centering
    \caption{Aufbau für das Passieren einer Kreuzung von vier Gruppen, bestehend aus jeweils vier Agenten}
    \label{fig:kreuzung}
\end{figure}
In diesem Experiment versuchen vier Gruppen von jeweils vier Agenten eine Kreuzung zu durchqueren. Ziel jeder Gruppe ist es, die gegenüberliegende Seite in gleicher Formation zu erreichen. Der Kreuzungsbereich besteht aus acht freien Feldern, bietet also nicht genügend Platz für alle Agenten. Normale Verkehrsregeln, zum Beispiel das Rechtsfahrgebot, Vorfahrtsregeln oder das Bilden von Fahrspuren, sind keine Lösungen, die ein konfliktfreies aneinander Vorbeifahren der Agenten ermöglichen. Die vier Fahrbahnen der Kreuzung sind jeweils zwei Felder breit und 14 Felder lang.

\textbf{Erwartete Beobachtungen}\newline
Wegen der dynamischem Prioritäten ist eine Vielzahl an Lösungen zu beobachten. Der Berechnungsaufwand pro Agent steigt, trotz der gestiegenen Teilnehmerzahl, nicht.
Ein möglicher zentraler Ansatz würde vermutlich zwei Gruppen blockieren und die übrigen gegenüberliegenden Gruppen durch die Kreuzung passieren lassen, um erst danach den blockierten Gruppen die Durchfahrt zu gewähren. Dies ist vergleichbar mit einer Ampel an einer Kreuzung im Straßenverkehr.


%
\subsection{Beobachtungen}
\label{chap:beobachtungen}
Alle zu erwartenden Beobachtungen konnten, so wie in \cite{book:regele} beschrieben, in den eigens ausgeführten Experimenten beobachtet werden.

\section{Experimente mit Messwerten}
\label{chap:mitMesswerten}
Die folgenden Experimente wurden zum Überprüfen der Leistungsfähigkeit des Algorithmus entwickelt. Im Kern stehen sich immer zwei gleich große Gruppen von Agenten gegenüber, deren Ziel es ist die Startpositionen der Agenten der anderen Gruppen zu erreichen. Die Anzahl der Agenten variiert zwischen den Experimenten und der freie Raum wird tendenziell immer kleiner. Für jedes Experiment ist es das Ziel, dass in allen 30 Wiederholung eine Lösung gefunden wird, also alle Agenten ihr Ziel erreichen.

Der Aufbau der Experimente und das zu erwartende Verhalten der Agenten werden vorgestellt. Im Kapitel \ref{chap:ergebnisse} folgen dann die Messwerte aus \cite{book:regele} und die eigenen.

\subsection{Sechs gegen sechs: Lockere Vorbeifahrt}
\label{chap:6x6_locker}
Das erste Experiment dieser Reihe bietet vergleichsweise viel Platz. Es ist vor allem für den späteren Vergleich interessant.

\textbf{Aufbau des Experiments}
\begin{figure}[H]
    \includegraphics[width=\textwidth]{images/6vs6_spacy.png}
    \centering
    \caption{Aufbau für die lockere Vorbeifahrt zweier Gruppen, bestehend aus jeweils sechs Agenten}
    \label{fig:6x6Locker}
\end{figure}
 Die Abbildung \ref{fig:6x6Locker} zeigt eine Karte die drei mal 30 Felder misst. Es stehen sich zwei Gruppen, bestehend aus jeweils sechs Agenten, gegenüber. Zum rechten beziehungsweise linken Rand sind für die Gruppen noch ein paar freie Felder vorhanden. Damit soll es den Agenten möglich sein, Ausweichbewegungen nach hinten hin ausführen zu können.
\subsection{Sechs gegen sechs: Enge Vorbeifahrt}
\label{chap:6x6_eng}
In diesem Experiment ist der Platz für Bewegungen sehr beschränkt. Zwölf Felder sind von Agenten belegt und lediglich neun sind frei. 

\textbf{Aufbau des Experiments}
\begin{figure}[H]
    \includegraphics[height=40mm]{images/6vs6_tight.png}
    \centering
    \caption{Aufbau für die enge Vorbeifahrt zweier Gruppen, bestehend aus jeweils sechs Agenten}
    \label{fig:6x6Eng}
\end{figure}
Die Karte für dieses Experiment ist sieben mal drei Felder groß. Es stehen sich zwei Gruppen aus jeweils sechs Agenten gegenüber. Wie in Abbildung \ref{fig:6x6Eng} zu erkennen, befindet sich zwischen den beiden Gruppen ein Block aus drei mal drei freien Feldern.
\subsection{Drei gegen drei: Enge Vorbeifahrt}
\label{chap:3x3_eng}
Dieses Experiment dient als Vergleich zum Vorangegangenen. Die Auswirkung die die Anzahl der Roboter hat, soll hiermit untersucht werden. 

\textbf{Aufbau des Experiments}
\begin{figure}[H]
    \includegraphics[height=40mm]{images/3vs3.png}
    \centering
    \caption{Aufbau für die enge Vorbeifahrt zweier Gruppen, bestehend aus jeweils drei Agenten}
    \label{fig:3x3}
\end{figure}
Die Karte für dieses Experiment ist sieben mal drei Felder groß. Es stehen sich zwei Gruppen aus jeweils drei Agenten gegenüber. Wie in Abbildung \ref{fig:3x3} zu erkennen, befindet sich zwischen den beiden Gruppen ein Block aus fünf mal drei freien Feldern.
\subsection{Vier gegen vier: Enge Vorbeifahrt}
\label{chap:4x4_eng}
Dieses Experiment schränkt den Platz der Agenten weiter ein. Die Zahl der belegten Felder ist hier größer als die Anzahl der freien Felder. Besonders für dieses Experiment ist, dass ein Graph vorliegt der die Prioritätswerte der Agenten über die Zeit zeigt (siehe Abbildung \ref{tab:resultsCoDy} und \ref{tab:myResults}). 

\textbf{Aufbau des Experiments}
\begin{figure}[H]
    \includegraphics[height=32mm]{images/4vs4_tight.png}
    \centering
    \caption{Aufbau für die enge Vorbeifahrt zweier Gruppen, bestehend aus jeweils vier Agenten}
    \label{fig:4x4}
\end{figure}
Die Karte für dieses Experiment misst sieben mal zwei Felder. Acht Agenten teilen sich in zwei gleich große Gruppen und stehen sich gegenüber.
\subsection{Messergebnisse}
\label{chap:ergebnisse}
\begin{table}[H]
    \centering
    \begin{tabular}{c|c|c|c|c|c|c}
       \textbf{Experiment} & \textbf{Gelöst} & \textbf{Prioritäts-Überlauf}
        & \(\textbf{s\textsubscript{opt}}\) & \(\textbf{s\textsubscript{CoDy}}\)
        & \(\textbf{t\textsubscript{opt}}\) & \(\textbf{t\textsubscript{CoDy}}\)\\ \hline
       \textbf{6-vs-6-locker} & 30 von 30 & 0 von 30
       & 19.5 & 20.97 & 22 & 24.33 \\ \hline
       \textbf{6-vs-6-eng} & 26 von 30 & 1 von 30
       & 5.45 & 11.5 & 10.35 & 21.89 \\ \hline
       \textbf{3-vs-3-eng} & 30 von 30 & 3 von 30
       & 5.5 & 6.35 & 6.33 & 7.68 \\ \hline
       \textbf{4-vs-4-eng} & 27 von 30 & 4 von 30
       & 5 & 10.85 & 10.5 & 24.96
    \end{tabular}
    \caption{Messwerte der Experimente aus \cite{book:regele}}
    \label{tab:resultsCoDy}
\end{table}

\begin{table}[H]
    \centering
    \begin{tabular}{c|c|c|c|c}
       \textbf{Experiment} & \textbf{Gelöst} & \textbf{Prioritäts-Überlauf}
        & [\textbf{\(\bar{s}\textsubscript{u}\)}, \textbf{\(\bar{s}\textsubscript{o}\)}]
        & [\textbf{\(\bar{t}\textsubscript{u}\)}, \textbf{\(\bar{t}\textsubscript{o}\)}]\\ \hline
       \textbf{6-vs-6-locker} & 30 von 30 & 0 von 30 & [20.01, 22.33] & [23.92, 26.4]\\ \hline
       \textbf{6-vs-6-eng} & 18 von 30 & 18 von 18 & [8.78, 9.81] & [19.52, 23.09]\\ \hline
       \textbf{3-vs-3-eng} & 27 von 30 & 0 von 27 & [7.3, 7.58] & [9.53, 9.97]\\ \hline
       \textbf{4-vs-4-eng} & 20 von 30 & 20 von 20 & [13.72, 25.2] & [23.04, 41.04]
    \end{tabular}
    \caption{Messwerte der selbst durchgeführten Experimente}
    \label{tab:myResults}
\end{table}

Die eigenen Messwerte weichen teils stark von den vorgegeben Messwerten ab. Lediglich das Experiment "'\ref{chap:6x6_locker} Sechs gegen sechs: Lockere Vorbeifahrt"' trifft die Erwartungen.

Für die eigenen Messungen ist anzumerken, dass für den Prioritätsüberlauf nicht immer "'von 30"' angegeben ist. Das hat damit zu tun, dass die Simulationsumgebung nur Daten erzeugt, wenn ein Experiment erfolgreich war.

\begin{figure}[H]
    \includegraphics[height=40mm]{images/6vs6_tight_full_block.png}
    \centering
    \caption{Beispielhafte Totalblockade für das Experiment "'\ref{chap:6x6_eng} Sechs gegen sechs: Enge Vorbeifahrt"'}
    \label{fig:6x6EngFullBlock}
\end{figure}
Für das Experiment "'\ref{chap:6x6_eng} Sechs gegen sechs: Enge Vorbeifahrt"' ist der zurückgelegte Strecke etwas besser als erwartet. Der Zeitbedarf trifft die Erwartung. Auffällig ist jedoch, dass nur 18 statt 26 Wiederholungen erfolgreich sind und es in allen, statt nur bei einer Wiederholung, zu Prioritätsüberläufen kam. Bei den zwölf ungelösten Wiederholungen ist eine Variation der in Abbildung \ref{fig:6x6EngFullBlock} gezeigten Situation eingetreten. 

Die Werte für "'\ref{chap:3x3_eng} Drei gegen drei: Enge Vorbeifahrt"'
weichen nur leicht von den erwarteten ab. Statt 30 werden nur 27 Wiederholungen gelöst. Dafür kommt es in den gelösten Wiederholungen aber nicht zu Prioritätsüberläufen. Die drei erwarteten Wiederholungen, die einen Prioritätsüberläufen haben, entstehen aus Situationen, in denen sich die Agenten in breiter Front aufeinander zu bewegen. In den eigenen Experimenten führten genau diese Situation zu den drei ungelösten Wiederholungen.

Die gemessenen Werte weichen für das Experiment "'\ref{chap:4x4_eng} Vier gegen vier: Enge Vorbeifahrt"' am stärksten ab. Es werden nur in 20 statt 27 Wiederholungen Lösungen gefunden. In jeder Wiederholung kommt es zu Prioritätsüberläufen. Auch das war nicht erwartet. Die Konfidenzintervalle für die durchschnittlich zurückgelegte Strecke und den durchschnittlichen Zeitbedarf sind im Vergleich besonders groß. Die erwartete Strecke ist nicht mal im Intervall enthalten. Wie Abbildungen \ref{fig:4x4_prio_2} deutlich zeigt, wird der erwartete Prioritätsverlauf verfehlt. Statt das die Priorität der Agenten über die Zeit stetig ansteigt und teilweise wieder zurückgeht, pendeln bei den eigenen Experimenten die Prioritätswerte zwischen minimalen und maximalen Werten hin und her.

\begin{figure}[H]
    \includegraphics[width=\textwidth]{images/4vs4_tight_prio_ref.png}
    \centering
    \caption{Prioritätsverlauf aller Agenten für das Experiment "'\ref{chap:4x4_eng} Vier gegen vier: Enge Vorbeifahrt"' aus \cite{book:regele}}
    \label{fig:4x4_prio_1}
\end{figure}

\begin{figure}[H]
    \includegraphics[width=\textwidth]{images/4vs4_tight_prio.png}
    \centering
    \caption{Prioritätsverlauf aller Agenten für das Experiment "'\ref{chap:4x4_eng} Vier gegen vier: Enge Vorbeifahrt"' der selbst durchgeführten Experimente}
    \label{fig:4x4_prio_2}
\end{figure}
%
\chapter{Diskussion}
\label{chap:diskussion}
Wie im Rest der Arbeit soll sich auch das Fazit an den drei Kern-Anforderungen entlanghangeln, die zu Beginn festgelegt wurden. Für jeden Aspekt soll einzeln erörtert werden, ob die Zielsetzung erreicht werden konnte, und welche Erfahrungen bei der Implementation gemacht werden konnten.

Webbasiert:
Trotz einiger Nachteile, die der Browser als Laufzeitumgebung mit sich bringt, erscheint es nur logisch, ein CRS webbasiert zu implementieren. Die technischen Anforderungen an das ausführende System sind relativ gering. Die Anforderungen an die User Experience und eine gut gestaltete Benutzeroberfläche im Gegensatz dazu eher hoch. Es erscheint nicht mehr zeitgemäß für eine solch simple (in Bezug auf die technischen Anforderungen) Aufgabe Software extra herunterladen und installieren zu müssen. Bei StuReSy war diese Entwurfs-Entscheidung nachvollziehbar, da dort auch Hardware-Abstimmungsgeräte und ein Plugin-System unterstützt werden sollten, die in einer Web-Anwendung deutlich schwerer zu integrieren gewesen wären.

Das React-Framework hat sich als gute Wahl für diese Aufgabe herausgestellet. Die bereitgestellten Konzepte und Bibliotheken waren relativ einfach zu verstehen und haben eine schnelle Entwicklung ermöglicht. Sicherlich hätte die Implementierung aber auch mit alternativen Frameworks erfolgreich funktioniert. Die Unterschiede der großen JavaScript-Frameworks (React, Vue) scheinen nur marginal zu sein.

Peer-2-Peer-Topologie: Die Implementierung einer P2P-Topologie auf Basis von WebRTC erwies sich als schwierig. Der WebRTC-Standard ist noch nicht fertiggestellt, und die Abweichungen zwischen den Browser-Herstellern sind relativ groß. WebRTC im Produktiveinsatz mit breiter Kompatibilität zu verwenden ist sehr aufwändig. Die Verwendung von WebRTC-Wrappern, die die Schnittstelle vereinfachen und das Signalling regeln erscheint gerade in kleinen Projekten notwendig zu sein. Leider gibt es in diesem Bereich nicht sehr viele, und nicht sehr aktuelle Open-Source-Bibliotheken.

Das fertige System funktioniert meistens einwandfrei. Allerdings kann es schnell zu Komplikationen kommen, zum Beispiel wenn die Verbindung der Teilnehmer unterbrochen wird, wenn Browser nicht unterstützt werden. Für den Produktiveinsatz sollte die Software an dieser Stelle noch robuster werden.

Code-Ausführung im Browser: Weclare zeigt, dass die Ausführung von Java-Quellcode im Browser grundsätzlich funktioniert.




%
\chapter{Fazit}
\label{chap:fazit}
Wie im Rest der Arbeit soll sich auch das Fazit an den drei Kern-Anforderungen entlanghangeln, die zu Beginn festgelegt wurden. Für jeden Aspekt soll einzeln erörtert werden, ob die Zielsetzung erreicht werden konnte, und welche Erfahrungen bei der Implementation gemacht werden konnten.

Webbasiert:
Trotz einiger Nachteile, die der Browser als Laufzeitumgebung mit sich bringt, erscheint es nur logisch, ein CRS webbasiert zu implementieren. Die technischen Anforderungen an das ausführende System sind relativ gering. Die Anforderungen an die User Experience und eine gut gestaltete Benutzeroberfläche im Gegensatz dazu eher hoch. Es erscheint nicht mehr zeitgemäß für eine solch simple (in Bezug auf die technischen Anforderungen) Aufgabe Software extra herunterladen und installieren zu müssen. Bei StuReSy war diese Entwurfs-Entscheidung nachvollziehbar, da dort auch Hardware-Abstimmungsgeräte und ein Plugin-System unterstützt werden sollten, die in einer Web-Anwendung deutlich schwerer zu integrieren gewesen wären.

Das React-Framework hat sich als gute Wahl für diese Aufgabe herausgestellet. Die bereitgestellten Konzepte und Bibliotheken waren relativ einfach zu verstehen und haben eine schnelle Entwicklung ermöglicht. Sicherlich hätte die Implementierung aber auch mit alternativen Frameworks erfolgreich funktioniert. Die Unterschiede der großen JavaScript-Frameworks (React, Vue) scheinen nur marginal zu sein.

Peer-2-Peer-Topologie: Die Implementierung einer P2P-Topologie auf Basis von WebRTC erwies sich als schwierig. Der WebRTC-Standard ist noch nicht fertiggestellt, und die Abweichungen zwischen den Browser-Herstellern sind relativ groß. WebRTC im Produktiveinsatz mit breiter Kompatibilität zu verwenden ist sehr aufwändig. Die Verwendung von WebRTC-Wrappern, die die Schnittstelle vereinfachen und das Signalling regeln erscheint gerade in kleinen Projekten notwendig zu sein. Leider gibt es in diesem Bereich nicht sehr viele, und nicht sehr aktuelle Open-Source-Bibliotheken.

Das fertige System funktioniert meistens einwandfrei. Allerdings kann es schnell zu Komplikationen kommen, zum Beispiel wenn die Verbindung der Teilnehmer unterbrochen wird, wenn Browser nicht unterstützt werden. Für den Produktiveinsatz sollte die Software an dieser Stelle noch robuster werden.

Code-Ausführung im Browser: Weclare zeigt, dass die Ausführung von Java-Quellcode im Browser grundsätzlich funktioniert.




%%%%%% Chapters End

%% bibliography and other stuff
\backmatter

\typeout{===== Section: literature}
%% read the documentation for customizing the style
\printbibliography
\typeout{===== Section: nomenclature}
%% uncomment if a TOC entry is needed
%% \addcontentsline{toc}{chapter}{Glossar}
\renewcommand{\nomname}{Glossar}
\clearpage
\markboth{\nomname}{\nomname} %% see nomencl doc, page 9, section 4.1
\printnomenclature

%% index
\typeout{===== Section: index}
\printindex

\HAWasurency

\end{document}
